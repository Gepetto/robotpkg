% $LAAS: getting.tex 2009/03/04 16:02:11 mallet $
%
% Copyright (c) 2009 LAAS/CNRS
% All rights reserved.
%
% Permission to use, copy, modify, and distribute this software for any purpose
% with or without   fee is hereby granted, provided   that the above  copyright
% notice and this permission notice appear in all copies.
%
% THE SOFTWARE IS PROVIDED "AS IS" AND THE AUTHOR DISCLAIMS ALL WARRANTIES WITH
% REGARD TO THIS  SOFTWARE INCLUDING ALL  IMPLIED WARRANTIES OF MERCHANTABILITY
% AND FITNESS. IN NO EVENT SHALL THE AUTHOR  BE LIABLE FOR ANY SPECIAL, DIRECT,
% INDIRECT, OR CONSEQUENTIAL DAMAGES OR  ANY DAMAGES WHATSOEVER RESULTING  FROM
% LOSS OF USE, DATA OR PROFITS, WHETHER IN AN ACTION OF CONTRACT, NEGLIGENCE OR
% OTHER TORTIOUS ACTION,   ARISING OUT OF OR IN    CONNECTION WITH THE USE   OR
% PERFORMANCE OF THIS SOFTWARE.
%
%                                             Anthony Mallet on Sat Jan 10 2009
%


\section{Where to get robotpkg and how to keep it up-to-date} % ------------
\label{section:getting}

Before you download and extract the files, you need to decide where you want to
extract  them.     {\tt   robotpkg}  is     by  default   installed   in   {\tt
/opt/openrobots}. Creating this directory  will probably require administration
privileges (``{\tt root}'') and if you don't have such privileges, you are free
to  install the   sources  and  binary   packages wherever you  want  in   your
filesystem. The pathname shall not contain white-space or other characters that
are interpreted specially  by the shell and some  other programs: a safe bet is
to use only letters, digits, underscores and dashes.  The rest of this document
will assume that you are using {\tt /opt/openrobots}  and you should adapt this
path to whatever location you choosed.

Please note that you should use a prefix which is dedicated to packages and not
shared with other programs (i.e., do not try  and use a  prefix of {\tt /usr}).
Also, you should not try to add  any of your own  files or directories (such as
{\tt src/})  below  the prefix tree.    This is to  prevent  possible conflicts
between programs and other files  installed by the  package system and whatever
else may have been installed there.

{\tt robotpkg}   will {\em never} require   administration privileges by itself
after the initial bootstrap phase.  We thus  recommend that you initially setup
the {\tt /opt/openrobots} directory readable and writable  by your regular user
and work only as this  user afterwards.  If  something ever goes really  wrong,
you might thank yourself later that you did so\ldots  Setting up this directory
can be done with the following commands in a shell:

\begin{verbatim}
% sudo mkdir -p /opt/openrobots
% sudo chown `id -u` /opt/openrobots
\end{verbatim}

If the {\tt sudo} command is not available or if you are not allowed to run it,
you should consider installing {\tt robotpkg} to a different location.


\subsection{Getting the binary bootstrap kit}

At the moment, the  binary bootstrap kit is not  available. Please get the {\tt
robotpkg} sources as described in the next section.


\subsection{Getting robotpkg for source compilation}

{\tt         robotpkg}    sources   are       distributed     {\em    via}  the
\href{http://git-scm.com/}{\tt git}  software  content management system.  {\tt
git} will probably be readily available on your system but if you don't have it
installed   or if you    are  unsure  about it,   contact  your  local   system
administrator.

There are two download methods: the anonymous one and the authenticated
one:

\begin{itemize}

  \item  Anonymous download is the  recommended method if  you  don't intend to
  work on  the   robotpkg infrastructure  itself,  nor commit   any  changes or
  packages  additions  back to  the  robotpkg main repository. Furthermore, the
  possibility to send contributions via patches is still open.

  As your regular user, simply run in a shell:

\begin{verbatim}
% cd /opt/openrobots
% git clone http://softs.laas.fr/git/robots/robotpkg.git
\end{verbatim}


  \item Authenticated   download requires a  valid  login on the  main robotpkg
  repository, and will give you full commit access  to this repository.  Simply
  run the following:

\begin{verbatim}
% cd /opt/openrobots
% git clone ssh://softs.laas.fr/git/robots/robotpkg
\end{verbatim}

\end{itemize}


\subsection{Keeping robotpkg up-to-date} % ---------------------------------

{\tt robotpkg} is   a   living thing:    updates  to  the packages are     made
perdiodicaly,  problems are fixed,   enhancements  are developed\ldots If   you
downloaded the robotpkg sources via git, you  should keep it up-to-date so that
you get the most  recent packages descriptions. This is done by running {\tt
git pull} in the robotpkg source directory:

\begin{verbatim}
% cd /opt/openrobots/robotpkg
% git pull
\end{verbatim}

When you update robotpkg, the git program will only  touch those files that are
registered in the git repository. That means that any packages that you created
on your own will stay unmodified. If you change  files that are managed by git,
later updates will try to merge your changes with those that  have been done by
others. See the {\tt git-pull} manual for details.

If you want  to be informed  of package additions  and other  updates, a public
mailing    list  is   available    for   your    reading   pleasure.  Go     to
\url{https://sympa.laas.fr/sympa/info/robotpkg}    for   more  information  and
subscription.
