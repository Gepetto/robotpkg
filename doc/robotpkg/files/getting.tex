% $LAAS: getting.tex 2009/01/11 11:43:51 tho $
%
% Copyright (c) 2009 LAAS/CNRS
% All rights reserved.
%
% Permission to use, copy, modify, and distribute this software for any purpose
% with or without   fee is hereby granted, provided   that the above  copyright
% notice and this permission notice appear in all copies.
%
% THE SOFTWARE IS PROVIDED "AS IS" AND THE AUTHOR DISCLAIMS ALL WARRANTIES WITH
% REGARD TO THIS  SOFTWARE INCLUDING ALL  IMPLIED WARRANTIES OF MERCHANTABILITY
% AND FITNESS. IN NO EVENT SHALL THE AUTHOR  BE LIABLE FOR ANY SPECIAL, DIRECT,
% INDIRECT, OR CONSEQUENTIAL DAMAGES OR  ANY DAMAGES WHATSOEVER RESULTING  FROM
% LOSS OF USE, DATA OR PROFITS, WHETHER IN AN ACTION OF CONTRACT, NEGLIGENCE OR
% OTHER TORTIOUS ACTION,   ARISING OUT OF OR IN    CONNECTION WITH THE USE   OR
% PERFORMANCE OF THIS SOFTWARE.
%
%                                             Anthony Mallet on Sat Jan 10 2009
%

\section{Where to get robotpkg and how to keep it up-to-date} % ------------
\label{section:getting}

Before you download and extract the files, you need to decide where you want to
extract  them.  robotpkg is  usually installed  in  {\tt /opt/openrobots},  but
creating this directory will probably   require administration privileges.   If
you don't have such privileges, you are free to  install the sources and binary
packages wherever you want in your filesystem, provided  that the pathname does
not contain white-space or other  characters that are interpreted specially  by
the shell and some other programs.  A safe bet is to  use only letters, digits,
underscores  and dashes. The rest of  this  document will  assume  that you are
using {\tt  /opt/openrobots}.  You should adapt  this path to whatever location
you choosed.


\subsection{Getting robotpkg for the first time} % -------------------------

robotpkg  will {\em never} require administration  privileges by itself.  So we
recommend that you set up the {\tt  /opt/openrobots} directory with read, write
and execute permissions for your regular user  name and then  only work as this
user afterwards. If something ever goes really  wrong, you might thank yourself
later that you did so\ldots This  can be done with  the following commands in a
shell:

\begin{verbatim}
% sudo mkdir -p /opt/openrobots
% sudo chown `id -u` /opt/openrobots
\end{verbatim}

At  the    moment,  robotpkg   is      only   distributed    {\em  via}     the
\href{http://git-scm.com/}{\tt git}  software content  management  system. {\tt
git} will probably be available on your system but if you don't have it readily
installed   or if  you  are   unsure  about  it,   contact your  local   system
administrator.

There are two download methods: the anonymous one and the authenticated
one. The two methods are described here.


\subsubsection{The anonymous download}

Anonymous  download is perfect  if  you don't intend  to  work on  the robotpkg
infrastructure itself, nor commit any changes or packages additions back to the
robotpkg main repository.  This is the recommended way  to go: it will fit most
users' usage while still leaving the possibility to send feedback via patches.

As your regular user, simply run in a shell:

\begin{verbatim}
% cd /opt/openrobots
% git clone http://softs.laas.fr/git/robots/robotpkg.git
\end{verbatim}


\subsubsection{The authenticated download}

Authenticated download requires a valid login  on the main robotpkg repository,
and  will give you  full commit access to this   repository. Assuming your user
name is ``{\tt user}'', run the following:

\begin{verbatim}
% cd /opt/openrobots
% git clone ssh://user@softs.laas.fr/git/robots/robotpkg
\end{verbatim}


\subsection{Keeping robotpkg up-to-date} % ---------------------------------

robotpkg is  a living  thing: updates  to the packages  are made  perdiodicaly,
problems are fixed,  enhancements are developed\ldots  In order to get the most
recent packages descriptions, you should keep your robotpkg copy up-to-date by
regularly running {\tt git pull}:

\begin{verbatim}
% cd /opt/openrobots/robotpkg
% git pull
\end{verbatim}

When you update robotpkg, the git program will only  touch those files that are
registered in the git repository. That means that any packages that you created
on your own will stay unmodified. If you change files that  are managed by git,
later updates will try to merge your changes with  those that have been done by
others. See the git-pull manual for details.

If you want  to be informed  of package additions  and other  updates, a public
mailing    list  is   available    for   your    reading   pleasure.  Go     to
\url{https://sympa.laas.fr/sympa/info/robotpkg}    for   more  information  and
subscription.
