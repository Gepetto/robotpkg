% $LAAS: introduction.tex 2010/11/17 17:56:41 mallet $
%
% Copyright (c) 2009-2010 LAAS/CNRS
% All rights reserved.
%
% Permission to use, copy, modify, and distribute this software for any purpose
% with or without   fee is hereby granted, provided   that the above  copyright
% notice and this permission notice appear in all copies.
%
% THE SOFTWARE IS PROVIDED "AS IS" AND THE AUTHOR DISCLAIMS ALL WARRANTIES WITH
% REGARD TO THIS  SOFTWARE INCLUDING ALL  IMPLIED WARRANTIES OF MERCHANTABILITY
% AND FITNESS. IN NO EVENT SHALL THE AUTHOR  BE LIABLE FOR ANY SPECIAL, DIRECT,
% INDIRECT, OR CONSEQUENTIAL DAMAGES OR  ANY DAMAGES WHATSOEVER RESULTING  FROM
% LOSS OF USE, DATA OR PROFITS, WHETHER IN AN ACTION OF CONTRACT, NEGLIGENCE OR
% OTHER TORTIOUS ACTION,   ARISING OUT OF OR IN    CONNECTION WITH THE USE   OR
% PERFORMANCE OF THIS SOFTWARE.
%
%                                             Anthony Mallet on Sat Jan 10 2009
%

\section{What is robotpkg?} % ----------------------------------------------

The robotics research  community has always been developing  a lot of software,
in order  to illustrate theoretical concepts and  validate algorithms  on board
real robots.  A great amount of this software was  made freely available to the
community, especially for Unix-based systems,  and is usually available in form
of the source code. Therefore, before such software can be used, it needs to be
configured to  the local system, compiled and  installed.  This is exactly what
The Robotics Packages Collection (robotpkg) does.  robotpkg also has some basic
commands  to handle binary packages,  so that not  every user  has to build the
packages for himself, which is a time-costly, cumbersome and error-prone task.

The robotpkg project was initiated in the \href{http://www.laas.fr/}{Laboratory
for Analysis and Architecture of  Systems} (CNRS/LAAS), France.  The motivation
was, on the one hand,  to ease the software   maintenance tasks for the  robots
that are used there.   On the other  hand, roboticists at CNRS/LAAS have always
fostered  an  open-source  development   model  for   the   software they  were
developing.  In order to  help people  working with the  laboratory to  get the
LAAS software  running outside the laboratory,  a package management system was
necessary.

Although  robotpkg was an  innovative   project in  the robotics community  (it
started in 2006), a lot of general-purpose software packages management systems
were readily available at this time for  a great variety of Unix-based systems.
The main requirements that we wanted  robotpkg to fullfill  were listed and the
best existing package management system  was chosen as  a starting point.   The
biggest requirement was the  capacity of the system to  adapt to the  nature of
the robotic software,  being available mostly in form  of source code  only (no
binary packages),  with unfrequent stable  releases.  robotpkg had thus to deal
mostly with  source code  and automate the  compilation of  the  packages.  The
system chosen  as a starting  point was \href{http://www.pkgsrc.org}{The NetBSD
Packages  Collection} (pkgsrc).  robotpkg  can be considered as  a fork of this
project and  it is still very similar  to pkgsrc in  many points, although some
simplifications were made in order to provide  a tool geared toward people that
are not computer scientists but roboticists.

Due to its  origins, robotpkg provides many packages  developed at LAAS.  It is
however not  limited to such  packages and contains, in  fact, quite some other
software useful to  roboticists.  Of  course, robotpkg  is  not meant to  be  a
general purpose  packaging system   (although  there  would  be   no  technical
restriction to this) and will never  contain widely available packages that can
be found  on  any modern  Unix  distribution. Yet, robotpkg currently  contains
roughly one hundred and fifty packages, including:

\begin{itemize}
   \item architecture/genom - The LAAS Generator of Robotic Components

   \item simulation/openhrp - The Open Architecture Humanoid Robotics
   Platform from AIST, Japan

   \item architecture/openrtm - The robotic distributed middleware from AIST, Japan

   \item middleware/yarp - The ``other'', yet famous, robot platform

   \item ...just to name a few.
\end{itemize}


\section{Why robotpkg?} % --------------------------------------------------

robotpkg provides the following key features:

\begin{itemize}

   \item Easy building of software  from  source as well   as the creation  and
   installation of binary packages. The source and latest patches are retrieved
   from a master download site, checksum verified, then built on your system.

   \item All  packages are installed in a  consistent directory tree, including
   binaries, libraries, man pages and other documentation.

   \item  Package dependencies, including  when performing package updates, are
   handled automatically.

   \item The installation prefix, acceptable  software licenses and  build-time
   options  for a large  number of packages  are all set  in  a simple, central
   configuration file.

   \item The  entire framework source  (not including the  package distribution
   files themselves) is freely available under a BSD license, so you may extend
   and adapt robotpkg to your needs, like robotpkg was adapted from pkgsrc.

\end{itemize}


One question often asked by people is ``why was robotpkg forked from pkgsrc
instead of integrating the packages into pkgsrc?''. This is indeed a very good
question and the following paragraphs try to answer it.

First,  robotpkg is  not meant  to be  a replacement  for the  system's package
management tool (it does not  superseeds pkgsrc, dpkg, macports etc.). The goal
is to package software that is not widely available on a platform, and which is
mostly  "lab  software" (generally  of  lesser  quality  than widely  available
software).    Those   packages   change   (a   lot)  more   often,   and   more
drastically. Thus, robotpkg is a little bit closer to a "development" tool than
pkgsrc.  Other  ``system  packages''  are  correctly handled  by  a  number  of
packaging tools, and there is no need for a new tool.

Currently, pkgsrc mixes both infrastructure and packages descriptions
themselves. For someone working on e.g. Linux, checking-out
the whole pkgsrc tree would be cumbersome: it would be redundant with the base
Linux package system, plus it would be difficult to isolate the specific
robotic packages from the rest (the rest usually being available in the base
system). robotpkg currently suffers from the same symptom: this may change in
the future if the need for several package repositories becomes blatant.

robotpkg provides a number of features not available in pkgsrc (and probably
not really useful to pkgsrc either). The most important feature is to be able
to detect "system packages", that are considered as "external software not in
robotpkg but usually available on a unix system". pkgsrc has a similar system
but much more limited -- to a few base packages only. This is so because pkgsrc
is a full-fledged package system. Thus, it aims at being self contained, while
robotpkg does not.

Finally, there are a number of additions/changes to the pkgsrc infrastructure
that correspond to legitimate users requests and the specifc workflow in which
robotpkg is used. For instance, robotpkg provides the possibility to generate
an archive of a package from a specific tag in a source repository ``on the
fly'' or just bypass the archive generation and work directly from the source
repository to install the software. This later workflow is not encouraged, but
it is convenient to quickly test a -current version of some software to see if
it causes any problem. Those features could be ported back to pkgsrc if the
pkgsrc team would find them useful. In the meantime, robotpkg provides a
good testbed for them.

Still, robotpkg directly uses many of the pkgsrc tools unchanged and the binary
packages are fully compatible.


\section{Supported platforms} % --------------------------------------------

robotpkg consists of  a   source distribution. After retrieving    the required
source, you can be up and running with robotpkg in just minutes!

robotpkg  does not have much requirements  by itself and it  can work on a wide
variety of systems  as  long as they   provide a  GNU-make utility, a   working
C-compiler and a small, reasonably standard subset  of Unix commands (like sed,
awk, find,  grep ...).  However, individual packages  might have their specific
requirements.  The   following platforms  have been  reported  to  be supported
reasonably well:

\begin{center}\begin{tabular}{|c|c|}
\hline
Platform & Version
\doublehline
Fedora & 5 -- 13\\
Ubuntu & 7.10 -- 9.10\\
Debian & 5.03\\
CentOS & 5\\
NetBSD & 4 -- 5\\
Darwin & Partial support - infrastructure works, individual packages may not\\
\hline
\end{tabular}\end{center}


\section{Overview} % -------------------------------------------------------

This document is divided  into three parts.  \xref{chapter:user}{The first one}
describes how  one  can  use  one of   the  packages  in the  Robotics  Package
Collection, either  by installing a precompiled binary  package, or by building
one's own  copy  using  robotpkg.   \xref{chapter:developer}{The  second  part}
explains how  to prepare a package so  it can be  easily  built by  other users
without     knowing     about     the     package's    building        details.
\xref{chapter:internal}{The   third part} is  intended for  those   who want to
understand how robotpkg is implemented.


\section{Terminology} % ----------------------------------------------------

Here is a description of all the terminology used within this document.

\begin{description}
   \item[Package] A set of files and building instructions that describe what's
   necessary to build a certain piece  of software using robotpkg. Packages are
   traditionally stored under {\tt /opt/robotpkg}.

   \item[robotpkg]  This is  the The Robotics   Package Collection.  It handles
   building (compiling), installing, and removing of packages.

   \item[Distfile] This  term describes the file  or files that are provided by
   the author of the piece of software to distribute  his work. All the changes
   necessary to  build are reflected  in the corresponding package. Usually the
   distfile is in  the form of a  compressed  tar-archive, but other  types are
   possible,     too.    Distfiles   are      usually   stored    below    {\tt
   /opt/robotpkg/distfiles}.

   \item[Precompiled/binary package] A set of binaries built with robotpkg from
   a distfile  and stuffed together in a  single {\tt .tgz} file   so it can be
   installed  on machines of the same  machine architecture without the need to
   recompile. Packages are usually generated in {\tt /opt/robotpkg/packages}.

   Sometimes, this is  referred to by the  term ``package''  too, especially in
   the context of precompiled packages.

   \item[Program]  The  piece  of  software to  be  installed  which  will   be
   constructed from all the files in the distfile by the actions defined in the
   corresponding package.

\end{description}


\section{Roles involved in robotpkg} % -------------------------------------

\begin{description}
   \item[robotpkg users] The  robotpkg users  are people  who  use the packages
   provided by robotpkg.  Typically they are student  working  in robotics. The
   usage  of the software  that is {\em inside} the  packages is not covered by
   the robotpkg guide.

   There are two  kinds of robotpkg users:  Some only want to install pre-built
   binary packages.  Others build the robotpkg packages from source, either for
   installing them  directly or for building binary   packages themselves.  For
   robotpkg users, \xref{chapter:user}{Part~\ref{chapter:user}}  should provide
   all necessary documentation.

   \item[package  maintainers]   A   package maintainer  creates  packages   as
   described in \xref{chapter:developer}{Part~\ref{chapter:developer}}.

   \item[infrastructure  developers]  These people are    involved in all those
   files that live  in the {\tt mk/} directory   and below.  Only  these people
   should             need          to               read               through
   \xref{chapter:internal}{Part~\ref{chapter:internal}}, though others might be
   curious, too.

\end{description}


\section{Typography} % -----------------------------------------------------

When giving examples for  commands,  shell prompts  are  used  to show if   the
command  should/can be issued  as  root, or if  ``normal''  user privileges are
sufficient. We use  a {\tt \#}  for  root's shell  prompt, and  a {\tt \%}  for
users' shell prompt, assuming they use the C-shell or tcsh.
