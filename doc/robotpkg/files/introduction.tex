% $LAAS: introduction.tex 2009/01/16 23:58:51 tho $
%
% Copyright (c) 2009 LAAS/CNRS
% All rights reserved.
%
% Permission to use, copy, modify, and distribute this software for any purpose
% with or without   fee is hereby granted, provided   that the above  copyright
% notice and this permission notice appear in all copies.
%
% THE SOFTWARE IS PROVIDED "AS IS" AND THE AUTHOR DISCLAIMS ALL WARRANTIES WITH
% REGARD TO THIS  SOFTWARE INCLUDING ALL  IMPLIED WARRANTIES OF MERCHANTABILITY
% AND FITNESS. IN NO EVENT SHALL THE AUTHOR  BE LIABLE FOR ANY SPECIAL, DIRECT,
% INDIRECT, OR CONSEQUENTIAL DAMAGES OR  ANY DAMAGES WHATSOEVER RESULTING  FROM
% LOSS OF USE, DATA OR PROFITS, WHETHER IN AN ACTION OF CONTRACT, NEGLIGENCE OR
% OTHER TORTIOUS ACTION,   ARISING OUT OF OR IN    CONNECTION WITH THE USE   OR
% PERFORMANCE OF THIS SOFTWARE.
%
%                                             Anthony Mallet on Sat Jan 10 2009
%

\section{What is robotpkg?} % ----------------------------------------------

The robotics  research community has always been  developing a lot of software,
mostly to  illustrate theoretical concepts and  validatate algorithms  on board
real robots. A  great amount of this software  was made freely available to the
community, especially for Unix-based systems, and  is usually available in form
of the source code. Therefore, before such software can be used, it needs to be
configured to the local  system, compiled and installed.   This is exactly what
The Robotics Packages Collection (robotpkg) does.  robotpkg also has some basic
commands to handle binary  packages, so that not   every user has to  build the
packages for himself, which is a time-costly, cumbersome and error-prone task.

The  robotpkg   project was  initiated   in the   Laboratory for  Analysis  and
Architecture of Systems (CNRS/LAAS), France.   The motivation  was, on the  one
hand, to ease the software  maintenance tasks  for  the great amount of  robots
that are used there.  On  the other hand,  roboticists at CNRS/LAAS have always
fostered  an    open-source  development model   for  the   software they  were
developing.  In order to help people collaborating  with the laboratory, in the
context of research project,   to get the   LAAS software running  outside  the
laboratory, a package management system was necessary.

Although  robotpkg was an   innovative project in  the  robotics community  (it
started in 2006),  a lot of  generic software packages  management systems were
readily available at this time for a great variety  of Unix-based systems.  The
main requirements that we wanted robotpkg to fullfill  were listed and the best
existing package management system was chosen as a starting point.  The biggest
requirement  was  the capacity of  the  system to adapt to    the nature of the
robotic software, being available mostly in form of source code only (no binary
packages), with  unfrequent stable releases.  robotpkg had thus to  deal mostly
with source   code and automate  the compilation  of the packages.   The system
chosen  as   a starting point   was  The NetBSD  Packages  Collection (pkgsrc).
robotpkg can be  considered as a   fork of this  project and  it  is still very
similar to pkgsrc  in many points, although  some simplifications were made  in
robotpkg in order to provide a tool geared  toward people that are not computer
scientists but roboticists.

Due to its  origins, robotpkg provides many  packages developed at LAAS.  It is
however not limited to  such packages and  contains, in fact, quite some  other
software useful  to roboticists.   Of  course, robotpkg is  not  meant to be  a
general  purpose  packaging  system  (although there   would  be   no technical
restriction to this) and will never contain  widely available packages that can
be  found on any modern  Unix  distribution. Yet,  robotpkg currently  contains
roughly a hundred of packages, including:

\begin{itemize}
   \item architecture/genom - The LAAS Generator of Robotic Components

   \item architecture/openhrp - The Open Architecture Humanoid Robotics
   Platform (AIST, Japan)

   \item architecture/openrtm - The robotic distributed middleware from AIST

   \item ...just to name a few.
\end{itemize}


\section{Why robotpkg?} % --------------------------------------------------

robotpkg provides the following key features:

\begin{itemize}

   \item Easy building of software  from  source as well   as the creation  and
   installation of binary packages. The source and latest patches are retrieved
   from a master download site, checksum verified, then built on your system.

   \item All  packages are installed in a  consistent directory tree, including
   binaries, libraries, man pages and other documentation.

   \item  Package dependencies, including  when performing package updates, are
   handled automatically.

   \item The installation prefix, acceptable  software licenses and  build-time
   options  for a large  number of packages  are all set  in  a simple, central
   configuration file.

   \item   The entire  source (not   including  the package distribution  files
   themselves) is freely available  under a BSD license,  so you may extend and
   adapt robotpkg to your needs like robotpkg was adapted from pkgsrc.

\end{itemize}

\section{Supported platforms} % --------------------------------------------

robotpkg consists of  a   source distribution. After retrieving    the required
source, you can be up and running with robotpkg in just minutes!

robotpkg does  not have much requirements by  itself, so it  can work on a wide
variety of  systems  as long as  they  provide  a GNU-make   utility, a working
C-compiler chain and a reasonably standard small  subset of Unix commands (like
sed,  awk, find, ...).  Individual  packages might  have specific requirements,
however,  and the following  platforms  have   been reported  to  be  supported
reasonably well:

\begin{center}\begin{tabular}{cc}
\hline
Platform & Version \\
\hline
\hline
Fedora & 3, 5 -- 10\\
Ubuntu & 7.10 -- 8.10\\
CentOS & 5\\
NetBSD & 4, 5\\
\hline
\end{tabular}\end{center}


\section{Overview} % -------------------------------------------------------

This document is divided  into three parts.  \xref{chapter:user}{The first one}
describes how  one  can  use  one of   the  packages  in the  Robotics  Package
Collection, either  by installing a precompiled binary  package, or by building
one's own  copy  using  robotpkg.   \xref{chapter:developer}{The  second  part}
explains how  to prepare a package so  it can be  easily  built by  other users
without     knowing     about     the     package's    building        details.
\xref{chapter:internal}{The   third part} is  intended for  those   who want to
understand how robotpkg is implemented.


\section{Terminology} % ----------------------------------------------------

Here is a description of all the terminology used within this document.

\begin{description}
   \item[Package] A set of files and building instructions that describe what's
   necessary to build a certain piece  of software using robotpkg. Packages are
   traditionally stored under {\tt /opt/robotpkg}.

   \item[robotpkg]  This is  the The Robotics   Package Collection.  It handles
   building (compiling), installing, and removing of packages.

   \item[Distfile] This  term describes the file  or files that are provided by
   the author of the piece of software to distribute  his work. All the changes
   necessary to  build are reflected  in the corresponding package. Usually the
   distfile is in  the form of a  compressed  tar-archive, but other  types are
   possible,     too.    Distfiles   are      usually   stored    below    {\tt
   /opt/robotpkg/distfiles}.

   \item[Precompiled/binary package] A set of binaries built with pkgsrc from a
   distfile and  stuffed  together in a   single {\tt .tgz}  file so  it can be
   installed on machines of the  same machine architecture  without the need to
   recompile. Packages are usually generated in {\tt /opt/robotpkg/packages}.

   Sometimes, this is  referred to by the  term ``package''  too, especially in
   the context of precompiled packages.

   \item[Program]  The  piece  of  software to  be  installed  which  will   be
   constructed from all the files in the distfile by the actions defined in the
   corresponding package.

\end{description}


\section{Roles involved in robotpkg} % -------------------------------------

\begin{description}
   \item[robotpkg users] The  robotpkg users  are people  who  use the packages
   provided by robotpkg.  Typically they are student  working  in robotics. The
   usage  of the software  that is {\em inside} the  packages is not covered by
   the robotpkg guide.

   There are two  kinds of robotpkg users:  Some only want to install pre-built
   binary packages.  Others build the robotpkg packages from source, either for
   installing them  directly or for building binary   packages themselves.  For
   robotpkg users, \xref{chapter:user}{Part~\ref{chapter:user}}  should provide
   all necessary documentation.

   \item[package  maintainers]   A   package maintainer  creates  packages   as
   described in \xref{chapter:developer}{Part~\ref{chapter:developer}}.

   \item[infrastructure  developers]  These people are    involved in all those
   files that live  in the {\tt mk/} directory   and below.  Only  these people
   should             need          to               read               through
   \xref{chapter:internal}{Part~\ref{chapter:internal}}, though others might be
   curious, too.

\end{description}


\section{Typography} % -----------------------------------------------------

When giving examples for  commands,  shell prompts  are  used  to show if   the
command  should/can be issued  as  root, or if  ``normal''  user privileges are
sufficient. We use  a {\tt \#}  for  root's shell  prompt, and  a {\tt \%}  for
users' shell prompt, assuming they use the C-shell or tcsh.
