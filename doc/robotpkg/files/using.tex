% $LAAS: using.tex 2010/06/28 10:24:09 mallet $
%
% Copyright (c) 2009-2010 LAAS/CNRS
% All rights reserved.
%
% Permission to use, copy, modify, and distribute this software for any purpose
% with or without   fee is hereby granted, provided   that the above  copyright
% notice and this permission notice appear in all copies.
%
% THE SOFTWARE IS PROVIDED "AS IS" AND THE AUTHOR DISCLAIMS ALL WARRANTIES WITH
% REGARD TO THIS  SOFTWARE INCLUDING ALL  IMPLIED WARRANTIES OF MERCHANTABILITY
% AND FITNESS. IN NO EVENT SHALL THE AUTHOR  BE LIABLE FOR ANY SPECIAL, DIRECT,
% INDIRECT, OR CONSEQUENTIAL DAMAGES OR  ANY DAMAGES WHATSOEVER RESULTING  FROM
% LOSS OF USE, DATA OR PROFITS, WHETHER IN AN ACTION OF CONTRACT, NEGLIGENCE OR
% OTHER TORTIOUS ACTION,   ARISING OUT OF OR IN    CONNECTION WITH THE USE   OR
% PERFORMANCE OF THIS SOFTWARE.
%
%                                             Anthony Mallet on Sun Jan 11 2009
%

\section{Using robotpkg} % -------------------------------------------------

After obtaining \robotpkg, the  {\tt robotpkg} directory now  contains a set of
packages, organized   into  categories.  You can   browse  the online  index of
packages, or run {\tt  make index} from the {\tt  robotpkg} directory to  build
local {\tt index.html}  files for all  packages, viewable with any web  browser
such as {\tt lynx} or {\tt firefox}.


\subsection{Building packages from source} % -------------------------------

The  first step for  building  a  package  is  downloading the {\em  distfiles}
(i.e. the unmodified  source). If they have not  yet been downloaded, \robotpkg
will  fetch them automatically  and place them  in the {\tt robotpkg/distfiles}
directory.

Once the software  has  been downloaded,  any  patches will be applied  and the
package will  be compiled for  you.  This may  take some time depending on your
computer, and how many other packages the software depends on and their compile
time.

For  example,  type the following  commands  at the shell   prompt to build the
robotpkg documentation package:

\begin{verbatim}
% cd /opt/openrobots/robotpkg
% cd doc/robotpkg
% make
\end{verbatim}

The next  stage is  to  actually install the newly   compiled package onto your
system. While you   are still in  the directory  for whatever package  you  are
installing, you can do this by entering:

\begin{verbatim}
% make install
\end{verbatim}

Installing the package on your system does  not require you  to be root (except
for a few specific  packages). However, if   you bootstraped with a  prefix for
which   you  don't   have  writing   permissions,    \robotpkg   has a     {\rm
just-in-time-sudo}  feature,  which allows you to  become  {\tt  root}  for the
actual installation step.

That's it, the software should now be installed   under  the prefix  of the
packages tree --- {\tt /opt/openrobots} by default --- and setup for use.

You can now enter:

\begin{verbatim}
% make clean
\end{verbatim}

to remove the compiled files in the work  directory, as you shouldn't need them
any more. If  other packages were also  added to your system (dependencies)  to
allow your program to compile, you can also tidy these up with the command:

\begin{verbatim}
% make clean-depends
\end{verbatim}

Since  the three tasks of building,  installing and  cleaning correspond to the
typical usage of \robotpkg, a helper target doing all these tasks exists and is
called {\tt update}. Thus,  to intall a package  with a single command, you can
simply run:

\begin{verbatim}
% make update
\end{verbatim}

In addition, {\tt  make update} will  also recompile all the installed packages
that were depending on the package that you are updating. This can be quite
time consuming if you are updating a low-level package. Also, note that all
packages that depend on the package you are updating will be deinstalled
first and unavailable in your system until all packages are recompiled and
reinstalled.

%
%    <para>Some packages look in &mk.conf; to
%    alter some configuration options at build time.  Have a look at
%    <filename>pkgsrc/mk/defaults/mk.conf</filename> to get an overview
%    of what will be set there by default.  Environment variables such
%    as <varname>LOCALBASE</varname> can be set in
%    &mk.conf; to save having to remember to
%    set them each time you want to use pkgsrc.</para>
%

Occasionally, people want to ``look under the covers'' to see  what is going on
when a  package  is building  or being  installed.  This may  be for  debugging
purposes, or  out  of simple curiosity. A  number  of utility values have  been
added to help with this.

\begin{enumerate}

\item If you invoke the {\tt make} command with {\tt PKG\_DEBUG\_LEVEL=1}, then
      a huge amount of information will be displayed. For example,

\begin{verbatim}
% make patch PKG_DEBUG_LEVEL=1
\end{verbatim}

      will show all the commands that are invoked, up to and including the
      ``patch'' stage. Using {\tt PKG\_DEBUG\_LEVEL=2} will give you even
      more details.

\item If you want to know the value of a certain {\tt make} definition, then
   the {\tt VARNAME} variable   should be used,  in  conjunction with the  {\tt
   show-var} target.  e.g.  to show the  expansion  of the  {\tt make} variable
   {\tt LOCALBASE}:

\begin{verbatim}
% make show-var VARNAME=LOCALBASE
\end{verbatim}

\end{enumerate}

%    <para>If you want to install a binary package that you've either
%    created yourself (see next section), that you put into
%    pkgsrc/packages manually or that is located on a remote FTP
%    server, you can use the "bin-install" target. This target will
%    install a binary package - if available - via &man.pkg.add.1;,
%    else do a <command>make package</command>.  The list of remote FTP
%    sites searched is kept in the variable
%    <varname>BINPKG_SITES</varname>, which defaults to
%    ftp.NetBSD.org. Any flags that should be added to &man.pkg.add.1;
%    can be put into <varname>BIN_INSTALL_FLAGS</varname>.  See
%    <filename>pkgsrc/mk/defaults/mk.conf</filename> for more
%    details.</para>


%    <para>A final word of warning: If you set up a system that has a
%    non-standard setting for <varname>LOCALBASE</varname>, be sure to
%    set that before any packages are installed, as you cannot use
%    several directories for the same purpose. Doing so will result in
%    pkgsrc not being able to properly detect your installed packages,
%    and fail miserably. Note also that precompiled binary packages are
%    usually built with the default <varname>LOCALBASE</varname> of
%    <filename>/usr/pkg</filename>, and that you should
%    <emphasis>not</emphasis> install any if you use a non-standard
%    <varname>LOCALBASE</varname>.</para>


\subsection{Installing binary packages} % ----------------------------------

At the moment, installing binary packages is not documented.


\subsection{Removing packages} % -------------------------------------------

To deinstall a package, it does not matter whether it was installed from source
code or  from a  binary package.  The  {\tt robotpkg\_delete} command  does not
know it  anyway.  To delete a  package, you can just  run {\tt robotpkg\_delete
<package-name>}.  The package name can be given with or without version number.
Wildcards can  also be used  to deinstall a  set of packages, for  example {\tt
*genom*} all  packages whose  name contain  the word {\tt  genom}.  Be  sure to
include them  in quotes,  so that the  shell does  not expand them  before {\tt
robotpkg\_delete} sees them.

The {\tt -r} option is very powerful: it  removes all the packages that require
the package in question and then removes the package itself. For example:

\begin{verbatim}
% robotpkg_delete -r genom
\end{verbatim}

will remove genom and all the packages that used it; this allows
upgrading the {\tt genom} package.


\subsection{Getting information about installed packages} % ----------------

The {\tt  robotpkg\_info} shows information about installed  packages or binary
package files.


\subsection{Other administrative functions} % ------------------------------

The  {\tt robotpkg\_admin}  executes  various administrative  functions on  the
package system.
