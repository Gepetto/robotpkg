% $LAAS: pkgvars.tex 2010/10/06 18:45:43 mallet $
%
% Copyright (c) 2010 LAAS/CNRS
% All rights reserved.
%
% Permission to use, copy, modify, and distribute this software for any purpose
% with or without   fee is hereby granted, provided   that the above  copyright
% notice and this permission notice appear in all copies.
%
% THE SOFTWARE IS PROVIDED "AS IS" AND THE AUTHOR DISCLAIMS ALL WARRANTIES WITH
% REGARD TO THIS  SOFTWARE INCLUDING ALL  IMPLIED WARRANTIES OF MERCHANTABILITY
% AND FITNESS. IN NO EVENT SHALL THE AUTHOR  BE LIABLE FOR ANY SPECIAL, DIRECT,
% INDIRECT, OR CONSEQUENTIAL DAMAGES OR  ANY DAMAGES WHATSOEVER RESULTING  FROM
% LOSS OF USE, DATA OR PROFITS, WHETHER IN AN ACTION OF CONTRACT, NEGLIGENCE OR
% OTHER TORTIOUS ACTION,   ARISING OUT OF OR IN    CONNECTION WITH THE USE   OR
% PERFORMANCE OF THIS SOFTWARE.
%
%                                             Anthony Mallet on Wed Oct  6 2010
%
\section{Package files, directories and contents} % ------------------------
\label{section:pkgvars}

Whenever you're preparing a package, there are a number of files involved which
are described in the following sections.

\subsection{Makefile} % ----------------------------------------------------
\label{subsection:makefile}

Building, installation and creation of a package are all controlled by the
package's Makefile. The Makefile describes various things about a package,
for example from where to get it, how to configure, build, and install it.

A package Makefile contains several sections that describe the package.

In the first section there are the following variables, which should appear
exactly in the order given here. The order and grouping of the variables is
mostly historical and has no further meaning.

\begin{description}
   \item[MASTER\_SITES] In simple cases, {\tt MASTER\_SITES}  defines all URLs
   from where the distfile, whose name is derived from the {\tt DISTNAME}
   variable, is fetched.

   When actually fetching the distfiles, each item from {\tt MASTER\_SITES}
   gets the name of each distfile appended to it, without an intermediate
   slash. Therefore, all site values have to end with a slash or other
   separator character. This allows for example to set {\tt MASTER\_SITES} to a
   URL of a CGI script that gets the name of the distfile as a parameter. In
   this case, the definition would look like:
   \begin{quote}
      {\tt MASTER\_SITES=   http://www.example.com/download.cgi?file=}
   \end{quote}

   There are some predefined values for {\tt MASTER\_SITES}, which can be used
   in packages. The names of the variables should speak for themselves.
   \begin{quote}\tt
      \$\{MASTER\_SITE\_SOURCEFORGE\}\\
      \$\{MASTER\_SITE\_GNU\}\\
      \$\{MASTER\_SITE\_OPENROBOTS\}
   \end{quote}

   If you choose one of these predefined sites, you may want to specify a
   subdirectory of that site. Since these macros may expand to more than one
   actual site, {\em you must} use the following construct to specify a
   subdirectory:
   \begin{quote}\tt
      MASTER\_SITES=~\$\{MASTER\_SITE\_SOURCEFORGE:=project\_name/\}
   \end{quote}
   Note the trailing slash after the subdirectory name.

   \smallbreak
   \item[FETCH\_METHOD] This is the method used to download the distfile from
   {\tt MASTER\_SITES}. It defaults to '{\tt archive}' which corresponds to the
   normal situation where distfile is an archive available from {\tt
   MASTER\_SITES}, so it normally needs not to be set.

   However, it can happen that a software provider does not provide any archive
   available for download but has only a public repository. In this case, {\tt
   FETCH\_METHOD} can be set to {\tt cvs}, {\tt git} or {\tt svn} according to
   the kind of repository available. {\tt MASTER\_SITES} is then interpreted as
   a repository of the form {\tt url[@revision[+module]]}, where the bits
   between square brackets are optional and refer to a particular revision and
   module in the repository located at {\tt url}. {\tt url} can take any form
   supported by the underlying fetch tool ({\tt cvs}, {\tt git} or {\tt
   svn}). It is {\em strongly} advised to define at least a specific revision
   to be checked out, so that the package can be reproducibly installed in a
   known state.

\end{description}
