% $LAAS: pkgvars.tex 2010/10/29 17:22:33 mallet $
%
% Copyright (c) 2010 LAAS/CNRS
% All rights reserved.
%
% Permission to use, copy, modify, and distribute this software for any purpose
% with or without   fee is hereby granted, provided   that the above  copyright
% notice and this permission notice appear in all copies.
%
% THE SOFTWARE IS PROVIDED "AS IS" AND THE AUTHOR DISCLAIMS ALL WARRANTIES WITH
% REGARD TO THIS  SOFTWARE INCLUDING ALL  IMPLIED WARRANTIES OF MERCHANTABILITY
% AND FITNESS. IN NO EVENT SHALL THE AUTHOR  BE LIABLE FOR ANY SPECIAL, DIRECT,
% INDIRECT, OR CONSEQUENTIAL DAMAGES OR  ANY DAMAGES WHATSOEVER RESULTING  FROM
% LOSS OF USE, DATA OR PROFITS, WHETHER IN AN ACTION OF CONTRACT, NEGLIGENCE OR
% OTHER TORTIOUS ACTION,   ARISING OUT OF OR IN    CONNECTION WITH THE USE   OR
% PERFORMANCE OF THIS SOFTWARE.
%
%                                             Anthony Mallet on Wed Oct 29 2010
%

\section{Making a package work} % ------------------------------------------
\label{section:fixing}

\subsection{Incrementing versions when fixing an existing package} % -------

When making fixes to an existing package it can be useful to change the version
number in {\tt PKGNAME}. To avoid conflicting with future versions by the
original author, a "r1", "r2", ... suffix can be used on package versions by
setting {\tt PKGREVISION=1} ({\tt 2}, ...) in the package Makefile. E.g.
\begin{quote}
   DISTNAME=             foo-17.42\\
   PKGREVISION=          9
\end{quote}
will result in a {\tt PKGNAME} of "foo-17.42r9". The "r" is treated like a "."
by the package tools.

{\tt PKGREVISION} should be incremented for any non-trivial change in the
resulting binary package. Without a {\tt PKGREVISION} bump, someone with the
previous version installed has no way of knowing that their package is out
of date. Thus, changes without increasing {\tt PKGREVISION} are essentially
labeled "this is so trivial that no reasonable person would want to
upgrade", and this is the rough test for when increasing {\tt PKGREVISION}
is appropriate. Examples of changes that do not merit increasing {\tt
PKGREVISION} are:
\begin{itemize}
   \item Changing {\tt HOMEPAGE}, {\tt MAINTAINER} or comments in Makefile.
   \item Changing build variables if the resulting binary package is the same.
   \item Changing {\tt DESCR}.
   \item Adding {\tt PKG\_OPTIONS} if the default options don't change.
\end{itemize}

Examples of changes that do merit an increase to {\tt PKGREVISION} include:
\begin{itemize}
   \item Security fixes
   \item Changes or additions to a patch file
   \item Changes to the {\tt PLIST}
   \item A dependency is changed or renamed.
\end{itemize}

{\tt PKGREVISION} must also be incremented when dependencies have ABI changes.

When a new release of the package is released, the {\tt PKGREVISION} must be
removed.
