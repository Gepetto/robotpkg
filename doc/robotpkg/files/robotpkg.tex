% $LAAS: robotpkg.tex 2009/01/18 23:13:09 tho $
%
% Copyright (c) 2009 LAAS/CNRS
% All rights reserved.
%
% Permission to use, copy, modify, and distribute this software for any purpose
% with or without   fee is hereby granted, provided   that the above  copyright
% notice and this permission notice appear in all copies.
%
% THE SOFTWARE IS PROVIDED "AS IS" AND THE AUTHOR DISCLAIMS ALL WARRANTIES WITH
% REGARD TO THIS  SOFTWARE INCLUDING ALL  IMPLIED WARRANTIES OF MERCHANTABILITY
% AND FITNESS. IN NO EVENT SHALL THE AUTHOR  BE LIABLE FOR ANY SPECIAL, DIRECT,
% INDIRECT, OR CONSEQUENTIAL DAMAGES OR  ANY DAMAGES WHATSOEVER RESULTING  FROM
% LOSS OF USE, DATA OR PROFITS, WHETHER IN AN ACTION OF CONTRACT, NEGLIGENCE OR
% OTHER TORTIOUS ACTION,   ARISING OUT OF OR IN    CONNECTION WITH THE USE   OR
% PERFORMANCE OF THIS SOFTWARE.
%
%                                             Anthony Mallet on Sat Jan 10 2009
%

\documentclass[a4paper,11pt]{book}
\newif\iftth\iftth
   % $LAAS: robotpkg.tex 2010/10/08 15:39:36 mallet $
%
% Copyright (c) 2009-2010 LAAS/CNRS
% All rights reserved.
%
% Permission to use, copy, modify, and distribute this software for any purpose
% with or without   fee is hereby granted, provided   that the above  copyright
% notice and this permission notice appear in all copies.
%
% THE SOFTWARE IS PROVIDED "AS IS" AND THE AUTHOR DISCLAIMS ALL WARRANTIES WITH
% REGARD TO THIS  SOFTWARE INCLUDING ALL  IMPLIED WARRANTIES OF MERCHANTABILITY
% AND FITNESS. IN NO EVENT SHALL THE AUTHOR  BE LIABLE FOR ANY SPECIAL, DIRECT,
% INDIRECT, OR CONSEQUENTIAL DAMAGES OR  ANY DAMAGES WHATSOEVER RESULTING  FROM
% LOSS OF USE, DATA OR PROFITS, WHETHER IN AN ACTION OF CONTRACT, NEGLIGENCE OR
% OTHER TORTIOUS ACTION,   ARISING OUT OF OR IN    CONNECTION WITH THE USE   OR
% PERFORMANCE OF THIS SOFTWARE.
%
%                                             Anthony Mallet on Sat Jan 10 2009
%

\documentclass[a4paper,11pt]{book}
\newif\iftth\iftth
   % $LAAS: robotpkg.tex 2010/10/08 15:39:36 mallet $
%
% Copyright (c) 2009-2010 LAAS/CNRS
% All rights reserved.
%
% Permission to use, copy, modify, and distribute this software for any purpose
% with or without   fee is hereby granted, provided   that the above  copyright
% notice and this permission notice appear in all copies.
%
% THE SOFTWARE IS PROVIDED "AS IS" AND THE AUTHOR DISCLAIMS ALL WARRANTIES WITH
% REGARD TO THIS  SOFTWARE INCLUDING ALL  IMPLIED WARRANTIES OF MERCHANTABILITY
% AND FITNESS. IN NO EVENT SHALL THE AUTHOR  BE LIABLE FOR ANY SPECIAL, DIRECT,
% INDIRECT, OR CONSEQUENTIAL DAMAGES OR  ANY DAMAGES WHATSOEVER RESULTING  FROM
% LOSS OF USE, DATA OR PROFITS, WHETHER IN AN ACTION OF CONTRACT, NEGLIGENCE OR
% OTHER TORTIOUS ACTION,   ARISING OUT OF OR IN    CONNECTION WITH THE USE   OR
% PERFORMANCE OF THIS SOFTWARE.
%
%                                             Anthony Mallet on Sat Jan 10 2009
%

\documentclass[a4paper,11pt]{book}
\newif\iftth\iftth
   % $LAAS: robotpkg.tex 2010/10/08 15:39:36 mallet $
%
% Copyright (c) 2009-2010 LAAS/CNRS
% All rights reserved.
%
% Permission to use, copy, modify, and distribute this software for any purpose
% with or without   fee is hereby granted, provided   that the above  copyright
% notice and this permission notice appear in all copies.
%
% THE SOFTWARE IS PROVIDED "AS IS" AND THE AUTHOR DISCLAIMS ALL WARRANTIES WITH
% REGARD TO THIS  SOFTWARE INCLUDING ALL  IMPLIED WARRANTIES OF MERCHANTABILITY
% AND FITNESS. IN NO EVENT SHALL THE AUTHOR  BE LIABLE FOR ANY SPECIAL, DIRECT,
% INDIRECT, OR CONSEQUENTIAL DAMAGES OR  ANY DAMAGES WHATSOEVER RESULTING  FROM
% LOSS OF USE, DATA OR PROFITS, WHETHER IN AN ACTION OF CONTRACT, NEGLIGENCE OR
% OTHER TORTIOUS ACTION,   ARISING OUT OF OR IN    CONNECTION WITH THE USE   OR
% PERFORMANCE OF THIS SOFTWARE.
%
%                                             Anthony Mallet on Sat Jan 10 2009
%

\documentclass[a4paper,11pt]{book}
\newif\iftth\iftth
   \input{share/robotpkg.tth}
\else
   \usepackage[T1]{fontenc}
   \usepackage{robotpkg}
\fi

\title{A guide to robotpkg}
\author{
   Anthony Mallet --- {\tt anthony.mallet@laas.fr}\\[1em]
   Copyright 2006-2010 \copyright LAAS/CNRS
}
\date{\today}

\def\robotpkg{{\tt robotpkg} }

\begin{document} % ---------------------------------------------------------

\frontmatter
\maketitle
\tableofcontents
\mainmatter

\chapter{Introduction}
\label{chapter:introduction}
\input{introduction}

\chapter{The robotpkg user's guide}
\label{chapter:user}

Basically, there are two ways of using robotpkg.  The  first is to only install
the  package tools and to  use binary packages that  someone else has prepared.
The second way is  to install the  programs from source. Then  you are  able to
build your own packages,  and you can  still use  binary packages from  someone
else. Sections in this document will detail both approaches where appropriate.

\input{getting}
\input{bootstrapping}
\input{using}
\input{configuring}

\chapter{The robotpkg developer's guide}
\label{chapter:developer}

This part of the documentation deals with creating and modifying packages.

\input{pkgvars}

\chapter{The robotpkg infrastructure internals}
\label{chapter:internal}

\end{document} % -----------------------------------------------------------

\else
   \usepackage[T1]{fontenc}
   \usepackage{robotpkg}
\fi

\title{A guide to robotpkg}
\author{
   Anthony Mallet --- {\tt anthony.mallet@laas.fr}\\[1em]
   Copyright 2006-2010 \copyright LAAS/CNRS
}
\date{\today}

\def\robotpkg{{\tt robotpkg} }

\begin{document} % ---------------------------------------------------------

\frontmatter
\maketitle
\tableofcontents
\mainmatter

\chapter{Introduction}
\label{chapter:introduction}
% $LAAS: introduction.tex 2010/11/17 17:56:41 mallet $
%
% Copyright (c) 2009-2010 LAAS/CNRS
% All rights reserved.
%
% Permission to use, copy, modify, and distribute this software for any purpose
% with or without   fee is hereby granted, provided   that the above  copyright
% notice and this permission notice appear in all copies.
%
% THE SOFTWARE IS PROVIDED "AS IS" AND THE AUTHOR DISCLAIMS ALL WARRANTIES WITH
% REGARD TO THIS  SOFTWARE INCLUDING ALL  IMPLIED WARRANTIES OF MERCHANTABILITY
% AND FITNESS. IN NO EVENT SHALL THE AUTHOR  BE LIABLE FOR ANY SPECIAL, DIRECT,
% INDIRECT, OR CONSEQUENTIAL DAMAGES OR  ANY DAMAGES WHATSOEVER RESULTING  FROM
% LOSS OF USE, DATA OR PROFITS, WHETHER IN AN ACTION OF CONTRACT, NEGLIGENCE OR
% OTHER TORTIOUS ACTION,   ARISING OUT OF OR IN    CONNECTION WITH THE USE   OR
% PERFORMANCE OF THIS SOFTWARE.
%
%                                             Anthony Mallet on Sat Jan 10 2009
%

\section{What is robotpkg?} % ----------------------------------------------

The robotics research  community has always been developing  a lot of software,
in order  to illustrate theoretical concepts and  validate algorithms  on board
real robots.  A great amount of this software was  made freely available to the
community, especially for Unix-based systems,  and is usually available in form
of the source code. Therefore, before such software can be used, it needs to be
configured to  the local system, compiled and  installed.  This is exactly what
The Robotics Packages Collection (robotpkg) does.  robotpkg also has some basic
commands  to handle binary packages,  so that not  every user  has to build the
packages for himself, which is a time-costly, cumbersome and error-prone task.

The robotpkg project was initiated in the \href{http://www.laas.fr/}{Laboratory
for Analysis and Architecture of  Systems} (CNRS/LAAS), France.  The motivation
was, on the one hand,  to ease the software   maintenance tasks for the  robots
that are used there.   On the other  hand, roboticists at CNRS/LAAS have always
fostered  an  open-source  development   model  for   the   software they  were
developing.  In order to  help people  working with the  laboratory to  get the
LAAS software  running outside the laboratory,  a package management system was
necessary.

Although  robotpkg was an  innovative   project in  the robotics community  (it
started in 2006), a lot of general-purpose software packages management systems
were readily available at this time for  a great variety of Unix-based systems.
The main requirements that we wanted  robotpkg to fullfill  were listed and the
best existing package management system  was chosen as  a starting point.   The
biggest requirement was the  capacity of the system to  adapt to the  nature of
the robotic software,  being available mostly in form  of source code  only (no
binary packages),  with unfrequent stable  releases.  robotpkg had thus to deal
mostly with  source code  and automate the  compilation of  the  packages.  The
system chosen  as a starting  point was \href{http://www.pkgsrc.org}{The NetBSD
Packages  Collection} (pkgsrc).  robotpkg  can be considered as  a fork of this
project and  it is still very similar  to pkgsrc in  many points, although some
simplifications were made in order to provide  a tool geared toward people that
are not computer scientists but roboticists.

Due to its  origins, robotpkg provides many packages  developed at LAAS.  It is
however not  limited to such  packages and contains, in  fact, quite some other
software useful to  roboticists.  Of  course, robotpkg  is  not meant to  be  a
general purpose  packaging system   (although  there  would  be   no  technical
restriction to this) and will never  contain widely available packages that can
be found  on  any modern  Unix  distribution. Yet, robotpkg currently  contains
roughly one hundred and fifty packages, including:

\begin{itemize}
   \item architecture/genom - The LAAS Generator of Robotic Components

   \item simulation/openhrp - The Open Architecture Humanoid Robotics
   Platform from AIST, Japan

   \item architecture/openrtm - The robotic distributed middleware from AIST, Japan

   \item middleware/yarp - The ``other'', yet famous, robot platform

   \item ...just to name a few.
\end{itemize}


\section{Why robotpkg?} % --------------------------------------------------

robotpkg provides the following key features:

\begin{itemize}

   \item Easy building of software  from  source as well   as the creation  and
   installation of binary packages. The source and latest patches are retrieved
   from a master download site, checksum verified, then built on your system.

   \item All  packages are installed in a  consistent directory tree, including
   binaries, libraries, man pages and other documentation.

   \item  Package dependencies, including  when performing package updates, are
   handled automatically.

   \item The installation prefix, acceptable  software licenses and  build-time
   options  for a large  number of packages  are all set  in  a simple, central
   configuration file.

   \item The  entire framework source  (not including the  package distribution
   files themselves) is freely available under a BSD license, so you may extend
   and adapt robotpkg to your needs, like robotpkg was adapted from pkgsrc.

\end{itemize}


One question often asked by people is ``why was robotpkg forked from pkgsrc
instead of integrating the packages into pkgsrc?''. This is indeed a very good
question and the following paragraphs try to answer it.

First,  robotpkg is  not meant  to be  a replacement  for the  system's package
management tool (it does not  superseeds pkgsrc, dpkg, macports etc.). The goal
is to package software that is not widely available on a platform, and which is
mostly  "lab  software" (generally  of  lesser  quality  than widely  available
software).    Those   packages   change   (a   lot)  more   often,   and   more
drastically. Thus, robotpkg is a little bit closer to a "development" tool than
pkgsrc.  Other  ``system  packages''  are  correctly handled  by  a  number  of
packaging tools, and there is no need for a new tool.

Currently, pkgsrc mixes both infrastructure and packages descriptions
themselves. For someone working on e.g. Linux, checking-out
the whole pkgsrc tree would be cumbersome: it would be redundant with the base
Linux package system, plus it would be difficult to isolate the specific
robotic packages from the rest (the rest usually being available in the base
system). robotpkg currently suffers from the same symptom: this may change in
the future if the need for several package repositories becomes blatant.

robotpkg provides a number of features not available in pkgsrc (and probably
not really useful to pkgsrc either). The most important feature is to be able
to detect "system packages", that are considered as "external software not in
robotpkg but usually available on a unix system". pkgsrc has a similar system
but much more limited -- to a few base packages only. This is so because pkgsrc
is a full-fledged package system. Thus, it aims at being self contained, while
robotpkg does not.

Finally, there are a number of additions/changes to the pkgsrc infrastructure
that correspond to legitimate users requests and the specifc workflow in which
robotpkg is used. For instance, robotpkg provides the possibility to generate
an archive of a package from a specific tag in a source repository ``on the
fly'' or just bypass the archive generation and work directly from the source
repository to install the software. This later workflow is not encouraged, but
it is convenient to quickly test a -current version of some software to see if
it causes any problem. Those features could be ported back to pkgsrc if the
pkgsrc team would find them useful. In the meantime, robotpkg provides a
good testbed for them.

Still, robotpkg directly uses many of the pkgsrc tools unchanged and the binary
packages are fully compatible.


\section{Supported platforms} % --------------------------------------------

robotpkg consists of  a   source distribution. After retrieving    the required
source, you can be up and running with robotpkg in just minutes!

robotpkg  does not have much requirements  by itself and it  can work on a wide
variety of systems  as  long as they   provide a  GNU-make utility, a   working
C-compiler and a small, reasonably standard subset  of Unix commands (like sed,
awk, find,  grep ...).  However, individual packages  might have their specific
requirements.  The   following platforms  have been  reported  to  be supported
reasonably well:

\begin{center}\begin{tabular}{|c|c|}
\hline
Platform & Version
\doublehline
Fedora & 5 -- 13\\
Ubuntu & 7.10 -- 9.10\\
Debian & 5.03\\
CentOS & 5\\
NetBSD & 4 -- 5\\
Darwin & Partial support - infrastructure works, individual packages may not\\
\hline
\end{tabular}\end{center}


\section{Overview} % -------------------------------------------------------

This document is divided  into three parts.  \xref{chapter:user}{The first one}
describes how  one  can  use  one of   the  packages  in the  Robotics  Package
Collection, either  by installing a precompiled binary  package, or by building
one's own  copy  using  robotpkg.   \xref{chapter:developer}{The  second  part}
explains how  to prepare a package so  it can be  easily  built by  other users
without     knowing     about     the     package's    building        details.
\xref{chapter:internal}{The   third part} is  intended for  those   who want to
understand how robotpkg is implemented.


\section{Terminology} % ----------------------------------------------------

Here is a description of all the terminology used within this document.

\begin{description}
   \item[Package] A set of files and building instructions that describe what's
   necessary to build a certain piece  of software using robotpkg. Packages are
   traditionally stored under {\tt /opt/robotpkg}.

   \item[robotpkg]  This is  the The Robotics   Package Collection.  It handles
   building (compiling), installing, and removing of packages.

   \item[Distfile] This  term describes the file  or files that are provided by
   the author of the piece of software to distribute  his work. All the changes
   necessary to  build are reflected  in the corresponding package. Usually the
   distfile is in  the form of a  compressed  tar-archive, but other  types are
   possible,     too.    Distfiles   are      usually   stored    below    {\tt
   /opt/robotpkg/distfiles}.

   \item[Precompiled/binary package] A set of binaries built with robotpkg from
   a distfile  and stuffed together in a  single {\tt .tgz} file   so it can be
   installed  on machines of the same  machine architecture without the need to
   recompile. Packages are usually generated in {\tt /opt/robotpkg/packages}.

   Sometimes, this is  referred to by the  term ``package''  too, especially in
   the context of precompiled packages.

   \item[Program]  The  piece  of  software to  be  installed  which  will   be
   constructed from all the files in the distfile by the actions defined in the
   corresponding package.

\end{description}


\section{Roles involved in robotpkg} % -------------------------------------

\begin{description}
   \item[robotpkg users] The  robotpkg users  are people  who  use the packages
   provided by robotpkg.  Typically they are student  working  in robotics. The
   usage  of the software  that is {\em inside} the  packages is not covered by
   the robotpkg guide.

   There are two  kinds of robotpkg users:  Some only want to install pre-built
   binary packages.  Others build the robotpkg packages from source, either for
   installing them  directly or for building binary   packages themselves.  For
   robotpkg users, \xref{chapter:user}{Part~\ref{chapter:user}}  should provide
   all necessary documentation.

   \item[package  maintainers]   A   package maintainer  creates  packages   as
   described in \xref{chapter:developer}{Part~\ref{chapter:developer}}.

   \item[infrastructure  developers]  These people are    involved in all those
   files that live  in the {\tt mk/} directory   and below.  Only  these people
   should             need          to               read               through
   \xref{chapter:internal}{Part~\ref{chapter:internal}}, though others might be
   curious, too.

\end{description}


\section{Typography} % -----------------------------------------------------

When giving examples for  commands,  shell prompts  are  used  to show if   the
command  should/can be issued  as  root, or if  ``normal''  user privileges are
sufficient. We use  a {\tt \#}  for  root's shell  prompt, and  a {\tt \%}  for
users' shell prompt, assuming they use the C-shell or tcsh.


\chapter{The robotpkg user's guide}
\label{chapter:user}

Basically, there are two ways of using robotpkg.  The  first is to only install
the  package tools and to  use binary packages that  someone else has prepared.
The second way is  to install the  programs from source. Then  you are  able to
build your own packages,  and you can  still use  binary packages from  someone
else. Sections in this document will detail both approaches where appropriate.

% $LAAS: getting.tex 2009/01/11 11:43:51 tho $
%
% Copyright (c) 2009 LAAS/CNRS
% All rights reserved.
%
% Permission to use, copy, modify, and distribute this software for any purpose
% with or without   fee is hereby granted, provided   that the above  copyright
% notice and this permission notice appear in all copies.
%
% THE SOFTWARE IS PROVIDED "AS IS" AND THE AUTHOR DISCLAIMS ALL WARRANTIES WITH
% REGARD TO THIS  SOFTWARE INCLUDING ALL  IMPLIED WARRANTIES OF MERCHANTABILITY
% AND FITNESS. IN NO EVENT SHALL THE AUTHOR  BE LIABLE FOR ANY SPECIAL, DIRECT,
% INDIRECT, OR CONSEQUENTIAL DAMAGES OR  ANY DAMAGES WHATSOEVER RESULTING  FROM
% LOSS OF USE, DATA OR PROFITS, WHETHER IN AN ACTION OF CONTRACT, NEGLIGENCE OR
% OTHER TORTIOUS ACTION,   ARISING OUT OF OR IN    CONNECTION WITH THE USE   OR
% PERFORMANCE OF THIS SOFTWARE.
%
%                                             Anthony Mallet on Sat Jan 10 2009
%

\section{Where to get robotpkg and how to keep it up-to-date} % ------------
\label{section:getting}

Before you download and extract the files, you need to decide where you want to
extract  them.  robotpkg is  usually installed  in  {\tt /opt/openrobots},  but
creating this directory will probably   require administration privileges.   If
you don't have such privileges, you are free to  install the sources and binary
packages wherever you want in your filesystem, provided  that the pathname does
not contain white-space or other  characters that are interpreted specially  by
the shell and some other programs.  A safe bet is to  use only letters, digits,
underscores  and dashes. The rest of  this  document will  assume  that you are
using {\tt  /opt/openrobots}.  You should adapt  this path to whatever location
you choosed.


\subsection{Getting robotpkg for the first time} % -------------------------

robotpkg  will {\em never} require administration  privileges by itself.  So we
recommend that you set up the {\tt  /opt/openrobots} directory with read, write
and execute permissions for your regular user  name and then  only work as this
user afterwards. If something ever goes really  wrong, you might thank yourself
later that you did so\ldots This  can be done with  the following commands in a
shell:

\begin{verbatim}
% sudo mkdir -p /opt/openrobots
% sudo chown `id -u` /opt/openrobots
\end{verbatim}

At  the    moment,  robotpkg   is      only   distributed    {\em  via}     the
\href{http://git-scm.com/}{\tt git}  software content  management  system. {\tt
git} will probably be available on your system but if you don't have it readily
installed   or if  you  are   unsure  about  it,   contact your  local   system
administrator.

There are two download methods: the anonymous one and the authenticated
one. The two methods are described here.


\subsubsection{The anonymous download}

Anonymous  download is perfect  if  you don't intend  to  work on  the robotpkg
infrastructure itself, nor commit any changes or packages additions back to the
robotpkg main repository.  This is the recommended way  to go: it will fit most
users' usage while still leaving the possibility to send feedback via patches.

As your regular user, simply run in a shell:

\begin{verbatim}
% cd /opt/openrobots
% git clone http://softs.laas.fr/git/robots/robotpkg.git
\end{verbatim}


\subsubsection{The authenticated download}

Authenticated download requires a valid login  on the main robotpkg repository,
and  will give you  full commit access to this   repository. Assuming your user
name is ``{\tt user}'', run the following:

\begin{verbatim}
% cd /opt/openrobots
% git clone ssh://user@softs.laas.fr/git/robots/robotpkg
\end{verbatim}


\subsection{Keeping robotpkg up-to-date} % ---------------------------------

robotpkg is  a living  thing: updates  to the packages  are made  perdiodicaly,
problems are fixed,  enhancements are developed\ldots  In order to get the most
recent packages descriptions, you should keep your robotpkg copy up-to-date by
regularly running {\tt git pull}:

\begin{verbatim}
% cd /opt/openrobots/robotpkg
% git pull
\end{verbatim}

When you update robotpkg, the git program will only  touch those files that are
registered in the git repository. That means that any packages that you created
on your own will stay unmodified. If you change files that  are managed by git,
later updates will try to merge your changes with  those that have been done by
others. See the git-pull manual for details.

If you want  to be informed  of package additions  and other  updates, a public
mailing    list  is   available    for   your    reading   pleasure.  Go     to
\url{https://sympa.laas.fr/sympa/info/robotpkg}    for   more  information  and
subscription.

% $LAAS: bootstrapping.tex 2010/06/23 15:04:49 mallet $
%
% Copyright (c) 2009-2010 LAAS/CNRS
% All rights reserved.
%
% Permission to use, copy, modify, and distribute this software for any purpose
% with or without   fee is hereby granted, provided   that the above  copyright
% notice and this permission notice appear in all copies.
%
% THE SOFTWARE IS PROVIDED "AS IS" AND THE AUTHOR DISCLAIMS ALL WARRANTIES WITH
% REGARD TO THIS  SOFTWARE INCLUDING ALL  IMPLIED WARRANTIES OF MERCHANTABILITY
% AND FITNESS. IN NO EVENT SHALL THE AUTHOR  BE LIABLE FOR ANY SPECIAL, DIRECT,
% INDIRECT, OR CONSEQUENTIAL DAMAGES OR  ANY DAMAGES WHATSOEVER RESULTING  FROM
% LOSS OF USE, DATA OR PROFITS, WHETHER IN AN ACTION OF CONTRACT, NEGLIGENCE OR
% OTHER TORTIOUS ACTION,   ARISING OUT OF OR IN    CONNECTION WITH THE USE   OR
% PERFORMANCE OF THIS SOFTWARE.
%
%                                             Anthony Mallet on Sun Jan 11 2009
%

\section{Bootstrapping robotpkg} % -----------------------------------------
\label{section:bootstrapping}

Once you have  downloaded the robotpkg sources  or the binary bootstrap kit  as
described  in  \xref{section:getting}{Section~\ref{section:getting}}, a minimal
set  of  the administrative package management  utilities  must be installed on
your system  before you  can  use robotpkg.   This  is  called the  ``bootstrap
phase'' and  should   be done only   once,  the very  first  time you  download
robotpkg.


\subsection{Bootstrapping via the binary kit} % ----------------------------

At the moment, the binary bootstrap kit is not available. Please bootstrap {\tt
robotpkg} as described in the next section.


\subsection{Bootstrapping from source} % -----------------------------------

You will  need a working C compiler  and the GNU-make   utility version 3.81 or
later.    If you have  extracted  the  robotpkg  archive  into  the standard {\tt
/opt/openrobots/robotpkg} location, installing the   bootstrap kit from  source
should then be as simple as:

\begin{verbatim}
% cd /opt/openrobots/robotpkg/bootstrap
% ./bootstrap
\end{verbatim}

This will  install various utilities   into {\tt /opt/openrobots/sbin}.

Should you prefer another installation path, you could use the {\tt -{}-prefix}
option to  change the default  installation prefix.  For  instance, configuring
robotpkg  to  install programs  into  the  openrobots  directory in  your  home
directory can be done like this:

\begin{verbatim}
% cd robotpkg/bootstrap
% ./bootstrap --prefix=${HOME}/openrobots
\end{verbatim}

{\bf  After the  bootstrap script  has run,  a message  indicating  the success
should be  displayed.  If  you choosed a  non-standard installation  path, read
this message carefuly}, as it contains  instructions that you have to follow in
order  to  setup your  shell  environment  correctly.   These instructions  are
described in the next section.


\subsubsection{Configuring your environment} % -----------------------------

If  you configured robotpkg,   during the bootstrap  phase,  to install to some
other location   than {\tt /opt/openrobots}, you  have   to setup manually your
shell environment so that it contains a few  variables holding the installation
path.  Assuming  you invoked bootstrap with {\tt --prefix=/path/to/openrobots},
you have two options that are compatible with each other:

\begin{itemize}
   \item Add  the directory {\tt  /path/to/openrobots/sbin}  to your {\tt PATH}
   variable. robotpkg will    then be able  to find    its administrative tools
   automatically and from that recover other configuration information. This is
   the preferred method.

   \item Create the environment variable {\tt ROBOTPKG\_BASE} and set its value
   to {\tt /path/to/openrobots}.  robotpkg will  look for this variable  first,
   so it takes precedence over the  first method.  This is  the method you have
   to choose  if  you have  configured  several instances  of robotpkg  in your
   system. This is ony useful in some circumstances and is not normally needed.
\end{itemize}

If  you  don't know  how  to setup   environment variables  permanently in your
system,  please  refer  to  your shell's  manual  or contact  your local system
administrator.


\subsubsection{The bootstrap script usage} % -------------------------------

The {\tt bootstrap} script will by default install the package administrative
tools in {\tt /opt/openrobots/sbin}, use {\tt gcc} as the C compiler and {\tt
make} as the GNU-make program. This behaviour can be fine-tuned by using the
following options:

\begin{description}
   \item[\tt   -{}-prefix <path>]   will   select the  prefix  location where
   programs will be installed in.

   \item[\tt -{}-sysconfdir <path>] defaults to {\tt <prefix>/etc}. This is the
   path to the robotpkg configuration file.  Other packages configuration files
   (if any) will also be stored in this directory.

   \item[\tt -{}-pkgdbdir  <path>] defaults to {\tt  <prefix>/var/db/pkg}. This
   is the path  to the package database  directory  where robotpkg will  do its
   internal bookkeeping.

   \item[\tt -{}-compiler <program>] defaults to {\tt gcc}.  Use this option if
   you want to use a different C compiler.

   \item[\tt -{}-make <program>] defaults to {\tt make}. Use this option if you
   want to use a different make program. This program should be compatible with
   GNU-make.

   \item[\tt -{}-help]  displays  the {\tt bootstrap} usage.  The comprehensive
   list of recognized options will be displayed.
\end{description}

%
% Copyright (c) 2009-2010,2013 LAAS/CNRS
% All rights reserved.
%
% Permission to use, copy, modify, and distribute this software for any purpose
% with or without   fee is hereby granted, provided   that the above  copyright
% notice and this permission notice appear in all copies.
%
% THE SOFTWARE IS PROVIDED "AS IS" AND THE AUTHOR DISCLAIMS ALL WARRANTIES WITH
% REGARD TO THIS  SOFTWARE INCLUDING ALL  IMPLIED WARRANTIES OF MERCHANTABILITY
% AND FITNESS. IN NO EVENT SHALL THE AUTHOR  BE LIABLE FOR ANY SPECIAL, DIRECT,
% INDIRECT, OR CONSEQUENTIAL DAMAGES OR  ANY DAMAGES WHATSOEVER RESULTING  FROM
% LOSS OF USE, DATA OR PROFITS, WHETHER IN AN ACTION OF CONTRACT, NEGLIGENCE OR
% OTHER TORTIOUS ACTION,   ARISING OUT OF OR IN    CONNECTION WITH THE USE   OR
% PERFORMANCE OF THIS SOFTWARE.
%
%                                             Anthony Mallet on Sun Jan 11 2009
%

\section{Using robotpkg} % -------------------------------------------------

After obtaining \robotpkg, the  {\tt robotpkg} directory now  contains a set of
packages, organized   into  categories.  You can   browse  the online  index of
packages, or run {\tt  make index} from the {\tt  robotpkg} directory to  build
local {\tt index.html}  files for all  packages, viewable with any web  browser
such as {\tt lynx} or {\tt firefox}.

\robotpkg is  essentially based on the  {\tt make(1)} program.  All actions are
triggered by invoking {\tt make} with the proper target. The following sections
document          the           most          useful          ones          and
\xref{section:using:targets}{section~\ref{section:using:targets}} recaps a more
comprehensive list.


\subsection{Building packages from source} % -------------------------------

The  first step for  building  a  package  is  downloading the {\em  distfiles}
(i.e. the unmodified  source). If they have not  yet been downloaded, \robotpkg
will  fetch them automatically  and place them  in the {\tt robotpkg/distfiles}
directory.

Once the software  has  been downloaded,  any  patches will be applied  and the
package will  be compiled for  you.  This may  take some time depending on your
computer, and how many other packages the software depends on and their compile
time.

For  example,  type the following  commands  at the shell   prompt to build the
robotpkg documentation package:

\begin{verbatim}
% cd /opt/openrobots/robotpkg
% cd doc/robotpkg
% make
\end{verbatim}

The next  stage is  to  actually install the newly   compiled package onto your
system. While you   are still in  the directory  for whatever package  you  are
installing, you can do this by entering:

\begin{verbatim}
% make install
\end{verbatim}

Installing the package on your system does  not require you  to be root (except
for a few specific  packages). However, if   you bootstraped with a  prefix for
which   you  don't   have  writing   permissions,    \robotpkg   has a     {\rm
just-in-time-sudo}  feature,  which allows you to  become  {\tt  root}  for the
actual installation step.

That's it, the software should now be installed   under  the prefix  of the
packages tree --- {\tt /opt/openrobots} by default --- and setup for use.

You can now enter:

\begin{verbatim}
% make clean
\end{verbatim}

to remove the compiled files in the work  directory, as you shouldn't need them
any more. If  other packages were also  added to your system (dependencies)  to
allow your program to compile, you can also tidy these up with the command:

\begin{verbatim}
% make clean-depends
\end{verbatim}

Since  the three tasks of building,  installing and  cleaning correspond to the
typical usage of \robotpkg, a helper target doing all these tasks exists and is
called {\tt update}. Thus,  to intall a package  with a single command, you can
simply run:

\begin{verbatim}
% make update
\end{verbatim}

In addition, {\tt  make update} will  also recompile all the installed packages
that were depending on the package that you are updating. This can be quite
time consuming if you are updating a low-level package. Also, note that all
packages that depend on the package you are updating will be deinstalled
first and unavailable in your system until all packages are recompiled and
reinstalled.

%
%    <para>Some packages look in &mk.conf; to
%    alter some configuration options at build time.  Have a look at
%    <filename>pkgsrc/mk/defaults/mk.conf</filename> to get an overview
%    of what will be set there by default.  Environment variables such
%    as <varname>LOCALBASE</varname> can be set in
%    &mk.conf; to save having to remember to
%    set them each time you want to use pkgsrc.</para>
%

Occasionally, people want to ``look under the covers'' to see  what is going on
when a  package  is building  or being  installed.  This may  be for  debugging
purposes, or  out  of simple curiosity. A  number  of utility values have  been
added to help with this.

\begin{enumerate}

\item If you invoke the {\tt make} command with {\tt PKG\_DEBUG\_LEVEL=1}, then
      a huge amount of information will be displayed. For example,

\begin{verbatim}
% make patch PKG_DEBUG_LEVEL=1
\end{verbatim}

      will show all the commands that are invoked, up to and including the
      ``patch'' stage. Using {\tt PKG\_DEBUG\_LEVEL=2} will give you even
      more details.

\item If you want to know the value of a certain {\tt make} definition, then
   the {\tt VARNAME} variable   should be used,  in  conjunction with the  {\tt
   show-var} target.  e.g.  to show the  expansion  of the  {\tt make} variable
   {\tt LOCALBASE}:

\begin{verbatim}
% make show-var VARNAME=LOCALBASE
\end{verbatim}

\end{enumerate}

%    <para>If you want to install a binary package that you've either
%    created yourself (see next section), that you put into
%    pkgsrc/packages manually or that is located on a remote FTP
%    server, you can use the "bin-install" target. This target will
%    install a binary package - if available - via &man.pkg.add.1;,
%    else do a <command>make package</command>.  The list of remote FTP
%    sites searched is kept in the variable
%    <varname>BINPKG_SITES</varname>, which defaults to
%    ftp.NetBSD.org. Any flags that should be added to &man.pkg.add.1;
%    can be put into <varname>BIN_INSTALL_FLAGS</varname>.  See
%    <filename>pkgsrc/mk/defaults/mk.conf</filename> for more
%    details.</para>


%    <para>A final word of warning: If you set up a system that has a
%    non-standard setting for <varname>LOCALBASE</varname>, be sure to
%    set that before any packages are installed, as you cannot use
%    several directories for the same purpose. Doing so will result in
%    pkgsrc not being able to properly detect your installed packages,
%    and fail miserably. Note also that precompiled binary packages are
%    usually built with the default <varname>LOCALBASE</varname> of
%    <filename>/usr/pkg</filename>, and that you should
%    <emphasis>not</emphasis> install any if you use a non-standard
%    <varname>LOCALBASE</varname>.</para>


\subsection{Building packages from a repository checkout} % ----------------
\label{section:using:checkout}

Before building a  package, \robotpkg fetches the sources  from the official(s)
download  location(s),  as  instructed  by the  {\tt  MASTER\_SITES}  variable.
This is the standard and expected behaviour when you work with stable packages.

Occasionally, though,  it is useful to fetch  a snapshot of the  sources from a
development repository. For instance, one  might want to quickly test a release
candidate of a  package, or fix a simple  bug and create a patch  from the fix.
Whenever a package defines  the {\tt MASTER\_REPOSITORY} variable, \robotpkg is
able to temporarily  work with the repository defined in  this variable. At the
moment, {\tt cvs}, {\tt svn} and {\tt git} repositories are supported.

To enable this feature for a given package,  you have to first instruct
\robotpkg to work from a '{\tt checkout}' (instead of the stable releases) by
doing '{\tt make checkout}' in the package directory. For instance:

\begin{verbatim}
% cd robotpkg/foo/bar
% make checkout
\end{verbatim}

This sets  a permanent flag in the  {\em working} directory of  the package and
the {\em checkout}  configuration option will be retained  until the next '{\tt
make clean}'. After a '{\tt make  clean}', the configuration option is set back
to its default and \robotpkg will  work again with stable releases. This option
is set on a {\em per} package  basis only: configuring one package to work with
checkouts does not affect the behaviour of other packages.

After a '{\tt make checkout}' (and until a '{\tt make clean}'), the package has
a regular  checkout in its {\em  working} subdirectory.  You  can thus manually
edit, commit, switch branches, etc.  in  the package sources, like in any other
repository, by  first {\tt  cd}ing into the  working directory, then  using the
usual repository commands ({\tt cvs}, {\tt svn} or {\tt git}).

Of  course, the  individual  \robotpkg  targets are  still  available from  the
package  entry in  the robotpkg  hierarchy.  You  can for  instance  {\tt 'make
patch'}, {\tt 'configure'}, {\tt 'build'}, {\tt 'install'} or {\tt 'update'} as
usual. Note that  \robotpkg is not exactly stateless, and  this is most visible
when  working with  checkouts:  for  instance, after  a  successful {\tt  'make
build'}, you  have to do {\tt 'make  rebuild'} to force rebuilding  if you have
modified  the  sources.   The  same  holds  for  {\tt   'configure'}  (do  {\tt
'reconfigure'})  or {\tt  'install'} (do  {'reinstall'}, but  since  you cannot
install a package  twice, you normally have to use {\tt  'make replace'} in the
particular case of reinstalling a package).

The  {\tt  'clean'}  target  is  special,  in  that  it  removes  the  checkout
configuration  option and  all checkouted  sources, including  locally modified
sources. In order to prevent accidental deletion of precious files, you have to
confirm the cleanign with {\tt 'clean confirm'}, as in:

\begin{verbatim}
% make clean confirm
\end{verbatim}

A  final  remark:  we {\em  STRONGLY  DISCOURAGE}  the  use  of robotpkg  as  a
development tool  (i.e. using the {\tt  'checkout'} feature on  a {\em regular}
basis), for at least two reasons:

\begin{itemize}
   \item \robotpkg  is not designed  for this: it  will not really help  you in
   your  daily   development  work,   compared  to  the   manual  configuration
   installation of the software. It will sometimes create even more trouble, by
   ensuring  that all  the software  depending  on the  checkouted software  is
   up-to-date, which is not necessarily something you want to do every time you
   compile.

   \item  A checkout  breaks the  notion  of 'release'  and you  loose all  the
   benefits from working with packages.  In particular, you have no clear state
   of what is installed: you cannot easily reproduce the situation of time T at
   time T+n and don't know precisely  who requires which version of what. It is
   much  more  efficient and  robust  to release  frequently  a  software in  a
   development phase, than using a {\em rolling release} approach.
\end{itemize}

In our opinion, the {\tt 'checkout'}  target use should be limited to testing a
release candidate or  quickly fix a bug  and create a patch from  the fix, that
you commit upstream and put in the patches/ directory until the next release.


\subsection{Installing binary packages} % ----------------------------------

At the moment, installing binary packages is not documented.

\subsection{Removing packages} % -------------------------------------------

To deinstall a package, it does not matter whether it was installed from source
code or  from a  binary package.  The  {\tt robotpkg\_delete} command  does not
know it  anyway.  To delete a  package, you can just  run {\tt robotpkg\_delete
<package-name>}.  The package name can be given with or without version number.
Wildcards can  also be used  to deinstall a  set of packages, for  example {\tt
*genom*} all  packages whose  name contain  the word {\tt  genom}.  Be  sure to
include them  in quotes,  so that the  shell does  not expand them  before {\tt
robotpkg\_delete} sees them.

The {\tt -r} option is very powerful: it  removes all the packages that require
the package in question and then removes the package itself. For example:

\begin{verbatim}
% robotpkg_delete -r genom
\end{verbatim}

will remove genom and all the packages that used it; this allows
upgrading the {\tt genom} package.

\subsection{Getting information about installed packages} % ----------------

The {\tt  robotpkg\_info} shows information about installed  packages or binary
package files.


\subsection{Other administrative functions} % ------------------------------

The  {\tt robotpkg\_admin}  executes  various administrative  functions on  the
package system.

\subsection{Available {\tt make} targets} % --------------------------------
\label{section:using:targets}

The following targets are available in a package directory. They can be invoked
by   running  {\tt   make  <target>}   after   {\tt  cd}ing   into  some   {\tt
robotpkg/category/package}.

\subsubsection{Source manipulation}
\begin{description}
   \item[{\tt fetch}] Download the {\tt\$\{DISTFILES\}}.

   \item[{\tt extract}] Extract the contents of {\tt\$\{DISTFILES\}} into the
   work directory {\tt\$\{WRKDIR\}}.

   \item[{\tt patch}] Apply local patches available in {\tt\$\{PATCHDIR\}}
   (usually the {\tt patches} directory in the package).

   \item[{\tt checkout}] Extract the sources in {\tt\$\{WRKDIR\}} from
   {\tt\$\{MASTER\_REPOSITORY\}} instead of {\tt\$\{MASTER\_SITES\}}. This can
   be used to fetch a not yet released version instead of the latest
   release. This is mutually exclusive with the {\tt fetch} and {\tt extract}
   targets. See
   \xref{section:using:checkout}{section~\ref{section:using:checkout}} for
   details.

   \item[{\tt configure}] Perform the necessary actions to configure the
   sources. This may for instance involve running {\tt configure} or {\tt
   cmake}. If no configuration is required, this step simply does nothing.

   \item[{\tt build}] Or  just {\tt make}, the default  target. It compiles the
   package locally in {\tt\$\{WRKDIR\}}.

   \item[{\tt install}] Install the package into {\tt\$\{PREFIX\}}. The package
   is then available to the rest of the system. If an older version of the
   package is installed and required by other packages, this target requires
   confirmation. Otherwise, any older version of the package is first
   deinstalled.

   \item[{\tt replace}] Same as {\tt install}, but does not remove packages
   that depend on the replaced package. This saves some time, since already
   installed package are not touched, but if the replaced package is
   incompatible with the older version, you will run into trouble. Use with
   care and when you know what you are doing.

   \item[{\tt clean}] Tidy the work directory and removes {\tt\$\{WRKDIR\}}. If
   the package was extracted using {\tt checkout}, this target requires
   confirmation as it may delete locally modified files that will be lost.

   \item[{\tt update}] This is a shortcut target for {\tt fetch}, {\tt
   extract}, {\tt configure}, {\tt build}, {\tt install} and {\tt clean}. If
   the package is already installed and up-to-date, the target asks for
   confirmation.

\end{description}

\subsubsection{Introspection}
\begin{description}
   \item[{\tt show-options}] Display the list of available alternatives (see
   \xref{section:configuring:alternatives}%
   {section~\ref{section:configuring:alternatives}})
   and build options (see
   \xref{section:configuring:build_options}%
   {section~\ref{section:configuring:build_options}}).

   \item[{\tt show-depends}] Recursively display all the required dependencies
   of a package. The results are splitted between system and \robotpkg
   dependencies, and missing dependencies are indicated.

   \item[{\tt show-var}] Display the contents of a variable. This must be
   invoked as {\tt make show-var VARNAME=foo}, where {\tt foo} is the name of
   the variable to be displayed.
\end{description}

\subsubsection{Package sets}

\begin{description}
   \item[{\tt fetch-depends}, {\tt replace-depends}, {\tt update-depends}, {\tt
   clean-depends}]
   This runs the same action as {\tt fetch}, {\tt replace}, {\tt update} or
   {\tt clean} (respectively), but on all dependencies of the package,
   including the package itself. Useful to update a meta-packages, for instance.

   \item[{\tt fetch-<set>}, {\tt replace-<set>}, {\tt update-<set>}, {\tt
   clean-<set>}]
   This runs the same action as {\tt fetch}, {\tt replace}, {\tt update} or
   {\tt clean} (respectively), but on all members of the package set named {\tt
   <set>}. See
   \xref{section:configuring:sets}{section~\ref{section:configuring:sets}} 
   for an explanation of package sets and how to configure them.

\end{description}

% $LAAS: configuring.tex 2010/09/09 15:46:20 mallet $
%
% Copyright (c) 2009-2010 LAAS/CNRS
% All rights reserved.
%
% Permission to use, copy, modify, and distribute this software for any purpose
% with or without   fee is hereby granted, provided   that the above  copyright
% notice and this permission notice appear in all copies.
%
% THE SOFTWARE IS PROVIDED "AS IS" AND THE AUTHOR DISCLAIMS ALL WARRANTIES WITH
% REGARD TO THIS  SOFTWARE INCLUDING ALL  IMPLIED WARRANTIES OF MERCHANTABILITY
% AND FITNESS. IN NO EVENT SHALL THE AUTHOR  BE LIABLE FOR ANY SPECIAL, DIRECT,
% INDIRECT, OR CONSEQUENTIAL DAMAGES OR  ANY DAMAGES WHATSOEVER RESULTING  FROM
% LOSS OF USE, DATA OR PROFITS, WHETHER IN AN ACTION OF CONTRACT, NEGLIGENCE OR
% OTHER TORTIOUS ACTION,   ARISING OUT OF OR IN    CONNECTION WITH THE USE   OR
% PERFORMANCE OF THIS SOFTWARE.
%
%                                             Anthony Mallet on Wed Mar  4 2009
%

\section{Configuring robotpkg} % -------------------------------------------

The whole \robotpkg system is configured  {\em via} a single, centralized file,
called {\tt   robotpkg.conf}  and  placed  in  the   {\tt  /opt/openrobots/etc}
directory  by default.  This location  might be redefined  during the bootstrap
phase,  see  \xref{section:bootstrapping}{Section~\ref{section:bootstrapping}}.
During    the  bootstrap, an initial configuration     file is created with the
settings you provided to {\tt bootstrap}.

The  format of  the configuration file   is that of   the usual GNU style  {\tt
Makefile}s. The whole \robotpkg configuration  is done by setting variables  in
this  file. Note that  you can  define all  kinds of  variables, and no special
error checking (for example for spelling mistakes)  takes place, so you have to
try it out to see if it works.


\subsection{Selecting build options} % -------------------------------------

Some packages have   build time options, usually   to select between  different
dependencies,  enable  optional   support for    big   dependencies or   enable
experimental features.

To see   which options, if  any, a   package supports,  and  which  options are
mutually exclusive, run {\tt make show-options}, for example:

\begin{verbatim}
Any of the following general options may be selected:
    debug   Produce debugging information for binary programs
    doc     Compile documentation material
    lex     Use lex in place of flex
    tcl     Enable support for TCL clients
    yacc    Use yacc in place of bison

These options are enabled by default:
    doc tcl

These options are currently enabled:
    doc tcl
\end{verbatim}

The following variables can be defined  in {\tt robotpkg.conf} to select which
options to enable for a package:

\begin{itemize}
   \item {\tt PKG\_DEFAULT\_OPTIONS}, which can be used to select or  disable
   options  for  all packages  that  support them,

   \item {\tt PKG\_OPTIONS.<pkgbase>}, which can  be   used  to select  or
   disable   options specifically for package {\tt pkgbase}. Options listed
   in these variables are selected, options preceded by {\tt -} are disabled.
\end{itemize}

A few examples:

\begin{verbatim}
PKG_DEFAULT_OPTIONS=    debug
PKG_OPTIONS.genom=      doc -tcl
\end{verbatim}

It is important to note  that options that were  specifically suggested by  the
package  maintainer must be  explicitely removed if you  do not wish to include
the option.  If you  are unsure you  can view the current  state with {\tt make
show-options}.

The following settings are  consulted in the order  given, and the last setting
that selects or disables an option is used:

\begin{enumerate}
   \item the default options as suggested by the package maintainer,

   \item {\tt PKG\_DEFAULT\_OPTIONS},

   \item {\tt PKG\_OPTIONS.<pkgbase>}
\end{enumerate}

For groups of mutually exclusive options, the last option selected is used, all
others are automatically  disabled.  If  an option of  the  group is explicitly
disabled, the previously selected option,  if any, is used.   It is an error if
no option from  a  required group  of  options is  selected, and  building  the
package will fail.


\subsection{Defining collections of packages} % ----------------------------

Instead of installing, removing or updating packages one-by-one, you can define
collections  of  packages  in  your  {\tt  robotpkg.conf}.  Once  one  or  more
collections  are defined,  they enable  special targets  that work  on  all the
packages of a collection.

To define a collection, you have to give it a name and list the set of packages
forming  the  collection   in  the  special  {\tt  PKGSET}   variable  in  {\tt
robotpkg.conf}. The syntax is the following:

\begin{verbatim}
PKGSET.<name> = <list>
\end{verbatim}

where {\tt <name>} is the name of the collection (any string is valid) and {\tt
<list>}  is  the  list  of  packages  in  the  collection,  in  the  form  {\tt
<category>/<name>}. For instance,

\begin{verbatim}
PKGSET.myset = architecture/genom devel/pocolibs
\end{verbatim}

defines  a collection named  {\tt myset}  that contains  the two  packages {\tt
genom} and {\tt pocolibs}.

For each collection {\tt <name>}  defined in {\tt robotpkg.conf}, the following
targets  are available:  {\tt clean-<name>},  {\tt  clean-depends-<name>}, {\tt
fetch-<name>},    {\tt     extract-<name>},    {\tt    install-<name>},    {\tt
replace-<name>},  {\tt update-<name>}. They  perform the  same action  as their
respective counterpart without  {\tt -<name>} suffix, expect that  they work on
all packages  of the set. In addition,  for the {\tt replace}  and {\tt update}
targets, they sort the packages in  the order of their dependencies so that the
job is done a sensible order.

For  the user  convenience,  A  special {\tt  installed}  collection is  always
defined  and represents all  currently installed  packages. Thus,  invoking the
{\tt update-installed} target will update all currently installed packages.

Two {\tt robotpkg.conf} variables affect the behaviour of {\tt robotpkg}
regarding packages sets:

\begin{description}
   \item[PKGSET\_FAILSAFE] When working on a set, and this variable is set to
   yes, robotpkg will continue with further packages instead of stopping on an
   error. If set to 'no', stop on first error. Default: no.

   \item[PKGSET\_STRICT] Specify if package sets should be considered as
   'strict' or include dependencies of packages defined in the set. If set to
   'yes', only package strictly defined in sets are considered. If set to 'no',
   dependencies of packages listed in sets are added to their respective
   sets. Default: no.
\end{description}

\subsection{General configuration variables} % -----------------------------

In  this  section,  you can   find some  variables   that apply  to  all \robotpkg
packages.

% A complete  list of the variables that  can be configured by the user
% is  available in <filename>mk/defaults/mk.conf</filename>,  together with  some
% comments that describe each variable's intent.</para>

\begin{description}
   \item[ACCEPTABLE\_LICENSES] List of acceptable licenses. Whenever you try to
   build a package  whose license is  not in this  list, you will get  an error
   message that includes instructions on how to change this variable.

   \item[DISTDIR] Where to store the downloaded copies of the original source
   distributions used for building \robotpkg packages. The default is
   {\tt \${ROBOTPKG\_DIR}/distfiles}.

   \item[PACKAGES] The top level directory for the binary packages. The default
   is  {\tt \${ROBOTPKG\_DIR}/packages}.

%   \item[PKG\_DBDIR] Where the database  about  installed packages  is  stored.
%   The default is {\tt /opt/openrobots/var/db/pkg}.

   \item[MASTER\_SITE\_BACKUP] List  of backup locations for distribution files
   if not found locally  or  in {\tt \${MASTER\_SITES}}.  The default  is\\
   {\tt http://softs.laas.fr/openrobots/robotpkg/distfiles/}.

   \item[PKG\_DEBUG\_LEVEL] The  level of debugging  output  which is displayed
   whilst making and installing the package.  The  default value for this is 0,
   which will not  display the commands as they  are executed (normal, default,
   quiet  operation); the value 1 will  display all shell commands before their
   invocation,  and  the value  2 will  display both the  shell commands before
   their invocation, and their actual execution progress with {\tt set -x}.
\end{description}


\subsection{Variables affecting the build process} % -----------------------

\begin{description}
   \item[WRKOBJDIR] The top level   directory where, if defined,  the  separate
   working directories will get created.  This is useful for building  packages
   on a different filesystem than the \robotpkg sources.

   \item[DEPENDS\_TARGET] By default,  dependencies are only installed,  and no
   binary package is  created  for them. You  can  set  this variable  to  {\tt
   package}   to   automatically  create    binary  packages   after installing
   dependencies.

   \item[LOCALBASE] Where packages will be installed. The default value is {\tt
   /opt/openrobots}.  Do not  mix     binary  packages with    different values
   of {\tt LOCALBASE}s!


%   \item[GCC\_REQUIRED]  This specifies requirements  on  the version of GCC to
%   use  when  building  packages.   This  variable  should contain   a  list of
%   constraints in the form {\tt  \{<=,<,-,>,>=\}n}. E.g.  to specifiy a minimum
%   version of 4.2  use ``{\tt >=4.2}'', or to  specifiy gcc version 4  only use
%   ``{\tt >=4 <5}''.

\end{description}


\subsection{Additional flags to the compiler} % ----------------------------

If you wish  to set compiler variables   such as {\tt CFLAGS},  {\tt CXXFLAGS},
{\tt FFLAGS} ... please make sure to use  the {\tt +=}  operator instead of the
{\tt {=}} operator:

\begin{verbatim}
CFLAGS+= -your -flags
\end{verbatim}

Using {\tt CFLAGS=} (i.e.  without the ``{\tt +}'') may  lead to  problems with
packages that need to add their own flags.

If you want  to pass flags  to the linker, both in  the configure  step and the
build step, you  can do this  in  two ways.   Either set {\tt  LDFLAGS} or {\tt
LIBS}.  The difference between  the two is that  {\tt LIBS} will be appended to
the command line, while {\tt LDFLAGS} come earlier. {\tt LDFLAGS} is pre-loaded
with rpath settings   for machines that support  it.  As with {\tt CFLAGS}  you
should use the {\tt +=} operator:

\begin{verbatim}
LDFLAGS+= -your -linkerflags
\end{verbatim}


\chapter{The robotpkg developer's guide}
\label{chapter:developer}

This part of the documentation deals with creating and modifying packages.

% $LAAS: pkgvars.tex 2010/10/28 15:08:35 mallet $
%
% Copyright (c) 2010 LAAS/CNRS
% All rights reserved.
%
% Permission to use, copy, modify, and distribute this software for any purpose
% with or without   fee is hereby granted, provided   that the above  copyright
% notice and this permission notice appear in all copies.
%
% THE SOFTWARE IS PROVIDED "AS IS" AND THE AUTHOR DISCLAIMS ALL WARRANTIES WITH
% REGARD TO THIS  SOFTWARE INCLUDING ALL  IMPLIED WARRANTIES OF MERCHANTABILITY
% AND FITNESS. IN NO EVENT SHALL THE AUTHOR  BE LIABLE FOR ANY SPECIAL, DIRECT,
% INDIRECT, OR CONSEQUENTIAL DAMAGES OR  ANY DAMAGES WHATSOEVER RESULTING  FROM
% LOSS OF USE, DATA OR PROFITS, WHETHER IN AN ACTION OF CONTRACT, NEGLIGENCE OR
% OTHER TORTIOUS ACTION,   ARISING OUT OF OR IN    CONNECTION WITH THE USE   OR
% PERFORMANCE OF THIS SOFTWARE.
%
%                                             Anthony Mallet on Wed Oct  6 2010
%
\section{Creating a new package} % -----------------------------------------
\label{section:pkgvars}

Whenever you're preparing a package, there are a number of files involved which
are described in the following sections.

\subsection{Makefile} % ----------------------------------------------------
\label{subsection:makefile}

Building, installation and creation of a package are all controlled by the
package's Makefile. The Makefile describes various things about a package,
for example from where to get it, how to configure, build, and install it.

A package Makefile contains several sections that describe the package.

In the first section there are the following variables, which should appear
exactly in the order given here. The order and grouping of the variables is
mostly historical and has no further meaning.

\begin{description}
   \item[MASTER\_SITES] In simple cases, {\tt MASTER\_SITES}  defines all URLs
   from where the distfile, whose name is derived from the {\tt DISTNAME}
   variable, is fetched.

   When actually fetching the distfiles, each item from {\tt MASTER\_SITES}
   gets the name of each distfile appended to it, without an intermediate
   slash. Therefore, all site values have to end with a slash or other
   separator character. This allows for example to set {\tt MASTER\_SITES} to a
   URL of a CGI script that gets the name of the distfile as a parameter. In
   this case, the definition would look like:
   \begin{quote}
      {\tt MASTER\_SITES=   http://www.example.com/download.cgi?file=}
   \end{quote}

   There are some predefined values for {\tt MASTER\_SITES}, which can be used
   in packages. The names of the variables should speak for themselves.
   \begin{quote}\tt
      \$\{MASTER\_SITE\_SOURCEFORGE\}\\
      \$\{MASTER\_SITE\_GNU\}\\
      \$\{MASTER\_SITE\_OPENROBOTS\}
   \end{quote}

   If you choose one of these predefined sites, you may want to specify a
   subdirectory of that site. Since these macros may expand to more than one
   actual site, {\em you must} use the following construct to specify a
   subdirectory:
   \begin{quote}\tt
      MASTER\_SITES=~\$\{MASTER\_SITE\_SOURCEFORGE:=project\_name/\}
   \end{quote}
   Note the trailing slash after the subdirectory name.

   \smallbreak
   \item[FETCH\_METHOD] This is the method used to download the distfile from
   {\tt MASTER\_SITES}. It defaults to '{\tt archive}' which corresponds to the
   normal situation where distfile is an archive available from {\tt
   MASTER\_SITES}, so it normally needs not to be set.

   However, it can happen that a software provider does not provide any archive
   available for download but has only a public repository. In this case, {\tt
   FETCH\_METHOD} can be set to {\tt cvs}, {\tt git} or {\tt svn} according to
   the kind of repository available. {\tt MASTER\_SITES} is then interpreted as
   a repository of the form {\tt url[@revision[+module]]}, where the bits
   between square brackets are optional and refer to a particular revision and
   module in the repository located at {\tt url}. {\tt url} can take any form
   supported by the underlying fetch tool ({\tt cvs}, {\tt git} or {\tt
   svn}). It is {\em strongly} advised to define at least a specific revision
   to be checked out, so that the package can be reproducibly installed in a
   known state.

\end{description}

The second section contains information about separately downloaded patches, if any.

\begin{description}

   \item[PATCHFILES] Name(s) of additional files that contain distribution
   patches distributed by the author or other maintainers. There is no
   default. robotpkg will look for them at {\tt    PATCH\_SITES}. They will
   automatically be uncompressed before patching if    the names end with .gz
   or .Z.

   \item[PATCH\_SITES] Primary location(s) for distribution patch files (see
   {\tt PATCHFILES} above) if not found locally.

\end{description}

The third section contains the following variables.

\begin{description}

   \item[MAINTAINER] is the email address of the person who feels responsible
   for this package, and who is most likely to look at problems or questions
   regarding this package. Other developers may contact the {\tt MAINTAINER}
   before making changes to the package, but are not required to do so. When
   packaging a new program, set {\tt MAINTAINER} to yourself. If you really
   can't maintain the package for future updates, set it to
   {\tt \string<robotpkg@laas.fr\string>}.

   \item[HOMEPAGE] is a URL where users can find more information about the
   package.

   \item[COMMENT] is a one-line description of the package (should not include
   the package name).

   \item[LICENSE] Denoting that a package may be installed and used according
   to a particular license is done by placing the license in {\tt
   robotpkg/licenses} and setting the LICENSE variable to a string identifying
   the license file, e.g. in {\tt shell/eltclsh}:
   \begin{quote}
      LICENSE=		2-clause-bsd
   \end{quote}

   The license tag mechanism is intended to address copyright-related issues
   surrounding building, installing and using a package, and not to address
   redistribution issues (see RESTRICTED and NO\_PUBLIC\_SRC, etc.). Packages
   with redistribution restrictions should set these tags.

\end{description}


Other variables affecting the build process may be gathered in their own
section:

\begin{description}

   \item[MAKE\_JOBS\_SAFE] Whether the package supports parallel builds. If set
   to yes, at most {\tt MAKE\_JOBS} jobs are carried out in parallel. The
   default value is ``yes'', and packages that don't support it must explicitly
   set it to ``no''.

\end{description}


\subsection{distinfo} % ----------------------------------------------------
\label{subsection:distinfo}

The distinfo file contains the message digest, or checksum, of each distfile
needed for the package. This ensures that the distfiles retrieved from the
Internet have not been corrupted during transfer or altered by a malign force
to introduce a security hole. Due to recent rumor about weaknesses of digest
algorithms, all distfiles are protected using both SHA1 and RMD160 message
digests, as well as the file size.

The distinfo file also contains the checksums for all the patches found in the
patches directory (see
\xref{subsection:patches}{Section~\ref{subsection:patches}}).

To regenerate the distinfo file, use the {\tt make distinfo} or {\tt make mdi}
command.


\subsection{PLIST}
\label{subsection:PLIST}

This  file  governs the  files  that  are installed  on  your  system: all  the
binaries, manual pages, etc. There are other directives which may be entered in
this  file,  to control  the  creation and  deletion  of  directories, and  the
location of inserted files.

The  names used  in the  PLIST are  relative to  the installation  prefix ({\tt
\$\{PREFIX\}}),  which  means  that  it  cannot  register  files  outside  this
directory  (absolute path names  are not  allowed). As  a general  sanity rule,
robotpkg must  not alter  any files outside  {\tt \$\{PREFIX\}} anyway  and, in
particular, not modify automatically existing configuration files. If a package
needs  to install  files  outside {\tt  \$\{PREFIX\}},  the best  option is  to
install   them   with   robotpkg   inside  {\tt   \$\{PREFIX\}}   (e.g.    {\tt
\$\{PREFIX\}/etc} or  {\tt \$\{PREFIX\}/var}) and  create a {\tt  MESSAGE} file
that will instruct the  user to manually link or copy the  files in question to
their final location. See the package {\tt hardware/ieee1394-kmod} for an
example of such package.

In  order to  create  or  update a  {\tt  PLIST}, you  can  use  the {\tt  make
print-PLIST} command  to output  a PLIST that  matches any new  installed files
since  the  package   was  extracted.   This  command  will   generate  a  {\tt
PLIST.guess} file which  you must move manually to  {\tt PLIST} after reviewing
the result of the semi-automatic generation.


\subsection{patches/*} % ----------------------------------------------------
\label{subsection:patches}

Some packages may not work out-of-the box with robotpkg. Therefore, a number of
custom patch  files may be needed to  make the package work.  These patch files
are found in the {\tt patches/} directory. If you want to share patches between
multiple packages  in robotpkg, e.g. because  they use the  same distfiles, set
{\tt PATCHDIR} to the path where the patch files can be found, e.g.:
\begin{quote}
   PATCHDIR= ../../devel/boost/patches
\end{quote}

The file names of the patch files must be of the form {\tt patch-*}, and they
are usually named {\tt patch-[a-z][a-z]}. In  the {\em  patch} phase,  these
patches  are automatically applied  to the  files  in {\tt \$\{WRKSRC\}}
directory after extracting them, in alphabetic order.

The {\tt patch-*} files should be in {\tt diff -bu} format, and apply without a
fuzz to avoid problems.  (To force patches to apply with fuzz  you can set {\tt
PATCH\_FUZZ\_FACTOR=-F2} in a package's {\tt Makefile}).

Each patch file should be commented so that any developer who knows the code of
the application  can make some use of  the patch. Special care  should be taken
for the upstream developers, since  we generally want that they accept robotpkg
patches, so there is less work in the future. When adding a patch that corrects
a problem in the distfile (rather than e.g. enforcing robotpkg's view of where
man pages should go), send the patch as a bug report to the maintainer. This
benefits non-robotpkg users of the package, and usually makes it possible to
remove the patch in future version.

When you add or modify existing patch files, remember to generate the checksums
for the patch files by using the {\tt make mdi} command, see
\xref{subsection:distinfo}{Section~\ref{subsection:distinfo}}.


\chapter{The robotpkg infrastructure internals}
\label{chapter:internal}

\end{document} % -----------------------------------------------------------

\else
   \usepackage[T1]{fontenc}
   \usepackage{robotpkg}
\fi

\title{A guide to robotpkg}
\author{
   Anthony Mallet --- {\tt anthony.mallet@laas.fr}\\[1em]
   Copyright 2006-2010 \copyright LAAS/CNRS
}
\date{\today}

\def\robotpkg{{\tt robotpkg} }

\begin{document} % ---------------------------------------------------------

\frontmatter
\maketitle
\tableofcontents
\mainmatter

\chapter{Introduction}
\label{chapter:introduction}
% $LAAS: introduction.tex 2010/11/17 17:56:41 mallet $
%
% Copyright (c) 2009-2010 LAAS/CNRS
% All rights reserved.
%
% Permission to use, copy, modify, and distribute this software for any purpose
% with or without   fee is hereby granted, provided   that the above  copyright
% notice and this permission notice appear in all copies.
%
% THE SOFTWARE IS PROVIDED "AS IS" AND THE AUTHOR DISCLAIMS ALL WARRANTIES WITH
% REGARD TO THIS  SOFTWARE INCLUDING ALL  IMPLIED WARRANTIES OF MERCHANTABILITY
% AND FITNESS. IN NO EVENT SHALL THE AUTHOR  BE LIABLE FOR ANY SPECIAL, DIRECT,
% INDIRECT, OR CONSEQUENTIAL DAMAGES OR  ANY DAMAGES WHATSOEVER RESULTING  FROM
% LOSS OF USE, DATA OR PROFITS, WHETHER IN AN ACTION OF CONTRACT, NEGLIGENCE OR
% OTHER TORTIOUS ACTION,   ARISING OUT OF OR IN    CONNECTION WITH THE USE   OR
% PERFORMANCE OF THIS SOFTWARE.
%
%                                             Anthony Mallet on Sat Jan 10 2009
%

\section{What is robotpkg?} % ----------------------------------------------

The robotics research  community has always been developing  a lot of software,
in order  to illustrate theoretical concepts and  validate algorithms  on board
real robots.  A great amount of this software was  made freely available to the
community, especially for Unix-based systems,  and is usually available in form
of the source code. Therefore, before such software can be used, it needs to be
configured to  the local system, compiled and  installed.  This is exactly what
The Robotics Packages Collection (robotpkg) does.  robotpkg also has some basic
commands  to handle binary packages,  so that not  every user  has to build the
packages for himself, which is a time-costly, cumbersome and error-prone task.

The robotpkg project was initiated in the \href{http://www.laas.fr/}{Laboratory
for Analysis and Architecture of  Systems} (CNRS/LAAS), France.  The motivation
was, on the one hand,  to ease the software   maintenance tasks for the  robots
that are used there.   On the other  hand, roboticists at CNRS/LAAS have always
fostered  an  open-source  development   model  for   the   software they  were
developing.  In order to  help people  working with the  laboratory to  get the
LAAS software  running outside the laboratory,  a package management system was
necessary.

Although  robotpkg was an  innovative   project in  the robotics community  (it
started in 2006), a lot of general-purpose software packages management systems
were readily available at this time for  a great variety of Unix-based systems.
The main requirements that we wanted  robotpkg to fullfill  were listed and the
best existing package management system  was chosen as  a starting point.   The
biggest requirement was the  capacity of the system to  adapt to the  nature of
the robotic software,  being available mostly in form  of source code  only (no
binary packages),  with unfrequent stable  releases.  robotpkg had thus to deal
mostly with  source code  and automate the  compilation of  the  packages.  The
system chosen  as a starting  point was \href{http://www.pkgsrc.org}{The NetBSD
Packages  Collection} (pkgsrc).  robotpkg  can be considered as  a fork of this
project and  it is still very similar  to pkgsrc in  many points, although some
simplifications were made in order to provide  a tool geared toward people that
are not computer scientists but roboticists.

Due to its  origins, robotpkg provides many packages  developed at LAAS.  It is
however not  limited to such  packages and contains, in  fact, quite some other
software useful to  roboticists.  Of  course, robotpkg  is  not meant to  be  a
general purpose  packaging system   (although  there  would  be   no  technical
restriction to this) and will never  contain widely available packages that can
be found  on  any modern  Unix  distribution. Yet, robotpkg currently  contains
roughly one hundred and fifty packages, including:

\begin{itemize}
   \item architecture/genom - The LAAS Generator of Robotic Components

   \item simulation/openhrp - The Open Architecture Humanoid Robotics
   Platform from AIST, Japan

   \item architecture/openrtm - The robotic distributed middleware from AIST, Japan

   \item middleware/yarp - The ``other'', yet famous, robot platform

   \item ...just to name a few.
\end{itemize}


\section{Why robotpkg?} % --------------------------------------------------

robotpkg provides the following key features:

\begin{itemize}

   \item Easy building of software  from  source as well   as the creation  and
   installation of binary packages. The source and latest patches are retrieved
   from a master download site, checksum verified, then built on your system.

   \item All  packages are installed in a  consistent directory tree, including
   binaries, libraries, man pages and other documentation.

   \item  Package dependencies, including  when performing package updates, are
   handled automatically.

   \item The installation prefix, acceptable  software licenses and  build-time
   options  for a large  number of packages  are all set  in  a simple, central
   configuration file.

   \item The  entire framework source  (not including the  package distribution
   files themselves) is freely available under a BSD license, so you may extend
   and adapt robotpkg to your needs, like robotpkg was adapted from pkgsrc.

\end{itemize}


One question often asked by people is ``why was robotpkg forked from pkgsrc
instead of integrating the packages into pkgsrc?''. This is indeed a very good
question and the following paragraphs try to answer it.

First,  robotpkg is  not meant  to be  a replacement  for the  system's package
management tool (it does not  superseeds pkgsrc, dpkg, macports etc.). The goal
is to package software that is not widely available on a platform, and which is
mostly  "lab  software" (generally  of  lesser  quality  than widely  available
software).    Those   packages   change   (a   lot)  more   often,   and   more
drastically. Thus, robotpkg is a little bit closer to a "development" tool than
pkgsrc.  Other  ``system  packages''  are  correctly handled  by  a  number  of
packaging tools, and there is no need for a new tool.

Currently, pkgsrc mixes both infrastructure and packages descriptions
themselves. For someone working on e.g. Linux, checking-out
the whole pkgsrc tree would be cumbersome: it would be redundant with the base
Linux package system, plus it would be difficult to isolate the specific
robotic packages from the rest (the rest usually being available in the base
system). robotpkg currently suffers from the same symptom: this may change in
the future if the need for several package repositories becomes blatant.

robotpkg provides a number of features not available in pkgsrc (and probably
not really useful to pkgsrc either). The most important feature is to be able
to detect "system packages", that are considered as "external software not in
robotpkg but usually available on a unix system". pkgsrc has a similar system
but much more limited -- to a few base packages only. This is so because pkgsrc
is a full-fledged package system. Thus, it aims at being self contained, while
robotpkg does not.

Finally, there are a number of additions/changes to the pkgsrc infrastructure
that correspond to legitimate users requests and the specifc workflow in which
robotpkg is used. For instance, robotpkg provides the possibility to generate
an archive of a package from a specific tag in a source repository ``on the
fly'' or just bypass the archive generation and work directly from the source
repository to install the software. This later workflow is not encouraged, but
it is convenient to quickly test a -current version of some software to see if
it causes any problem. Those features could be ported back to pkgsrc if the
pkgsrc team would find them useful. In the meantime, robotpkg provides a
good testbed for them.

Still, robotpkg directly uses many of the pkgsrc tools unchanged and the binary
packages are fully compatible.


\section{Supported platforms} % --------------------------------------------

robotpkg consists of  a   source distribution. After retrieving    the required
source, you can be up and running with robotpkg in just minutes!

robotpkg  does not have much requirements  by itself and it  can work on a wide
variety of systems  as  long as they   provide a  GNU-make utility, a   working
C-compiler and a small, reasonably standard subset  of Unix commands (like sed,
awk, find,  grep ...).  However, individual packages  might have their specific
requirements.  The   following platforms  have been  reported  to  be supported
reasonably well:

\begin{center}\begin{tabular}{|c|c|}
\hline
Platform & Version
\doublehline
Fedora & 5 -- 13\\
Ubuntu & 7.10 -- 9.10\\
Debian & 5.03\\
CentOS & 5\\
NetBSD & 4 -- 5\\
Darwin & Partial support - infrastructure works, individual packages may not\\
\hline
\end{tabular}\end{center}


\section{Overview} % -------------------------------------------------------

This document is divided  into three parts.  \xref{chapter:user}{The first one}
describes how  one  can  use  one of   the  packages  in the  Robotics  Package
Collection, either  by installing a precompiled binary  package, or by building
one's own  copy  using  robotpkg.   \xref{chapter:developer}{The  second  part}
explains how  to prepare a package so  it can be  easily  built by  other users
without     knowing     about     the     package's    building        details.
\xref{chapter:internal}{The   third part} is  intended for  those   who want to
understand how robotpkg is implemented.


\section{Terminology} % ----------------------------------------------------

Here is a description of all the terminology used within this document.

\begin{description}
   \item[Package] A set of files and building instructions that describe what's
   necessary to build a certain piece  of software using robotpkg. Packages are
   traditionally stored under {\tt /opt/robotpkg}.

   \item[robotpkg]  This is  the The Robotics   Package Collection.  It handles
   building (compiling), installing, and removing of packages.

   \item[Distfile] This  term describes the file  or files that are provided by
   the author of the piece of software to distribute  his work. All the changes
   necessary to  build are reflected  in the corresponding package. Usually the
   distfile is in  the form of a  compressed  tar-archive, but other  types are
   possible,     too.    Distfiles   are      usually   stored    below    {\tt
   /opt/robotpkg/distfiles}.

   \item[Precompiled/binary package] A set of binaries built with robotpkg from
   a distfile  and stuffed together in a  single {\tt .tgz} file   so it can be
   installed  on machines of the same  machine architecture without the need to
   recompile. Packages are usually generated in {\tt /opt/robotpkg/packages}.

   Sometimes, this is  referred to by the  term ``package''  too, especially in
   the context of precompiled packages.

   \item[Program]  The  piece  of  software to  be  installed  which  will   be
   constructed from all the files in the distfile by the actions defined in the
   corresponding package.

\end{description}


\section{Roles involved in robotpkg} % -------------------------------------

\begin{description}
   \item[robotpkg users] The  robotpkg users  are people  who  use the packages
   provided by robotpkg.  Typically they are student  working  in robotics. The
   usage  of the software  that is {\em inside} the  packages is not covered by
   the robotpkg guide.

   There are two  kinds of robotpkg users:  Some only want to install pre-built
   binary packages.  Others build the robotpkg packages from source, either for
   installing them  directly or for building binary   packages themselves.  For
   robotpkg users, \xref{chapter:user}{Part~\ref{chapter:user}}  should provide
   all necessary documentation.

   \item[package  maintainers]   A   package maintainer  creates  packages   as
   described in \xref{chapter:developer}{Part~\ref{chapter:developer}}.

   \item[infrastructure  developers]  These people are    involved in all those
   files that live  in the {\tt mk/} directory   and below.  Only  these people
   should             need          to               read               through
   \xref{chapter:internal}{Part~\ref{chapter:internal}}, though others might be
   curious, too.

\end{description}


\section{Typography} % -----------------------------------------------------

When giving examples for  commands,  shell prompts  are  used  to show if   the
command  should/can be issued  as  root, or if  ``normal''  user privileges are
sufficient. We use  a {\tt \#}  for  root's shell  prompt, and  a {\tt \%}  for
users' shell prompt, assuming they use the C-shell or tcsh.


\chapter{The robotpkg user's guide}
\label{chapter:user}

Basically, there are two ways of using robotpkg.  The  first is to only install
the  package tools and to  use binary packages that  someone else has prepared.
The second way is  to install the  programs from source. Then  you are  able to
build your own packages,  and you can  still use  binary packages from  someone
else. Sections in this document will detail both approaches where appropriate.

% $LAAS: getting.tex 2009/01/11 11:43:51 tho $
%
% Copyright (c) 2009 LAAS/CNRS
% All rights reserved.
%
% Permission to use, copy, modify, and distribute this software for any purpose
% with or without   fee is hereby granted, provided   that the above  copyright
% notice and this permission notice appear in all copies.
%
% THE SOFTWARE IS PROVIDED "AS IS" AND THE AUTHOR DISCLAIMS ALL WARRANTIES WITH
% REGARD TO THIS  SOFTWARE INCLUDING ALL  IMPLIED WARRANTIES OF MERCHANTABILITY
% AND FITNESS. IN NO EVENT SHALL THE AUTHOR  BE LIABLE FOR ANY SPECIAL, DIRECT,
% INDIRECT, OR CONSEQUENTIAL DAMAGES OR  ANY DAMAGES WHATSOEVER RESULTING  FROM
% LOSS OF USE, DATA OR PROFITS, WHETHER IN AN ACTION OF CONTRACT, NEGLIGENCE OR
% OTHER TORTIOUS ACTION,   ARISING OUT OF OR IN    CONNECTION WITH THE USE   OR
% PERFORMANCE OF THIS SOFTWARE.
%
%                                             Anthony Mallet on Sat Jan 10 2009
%

\section{Where to get robotpkg and how to keep it up-to-date} % ------------
\label{section:getting}

Before you download and extract the files, you need to decide where you want to
extract  them.  robotpkg is  usually installed  in  {\tt /opt/openrobots},  but
creating this directory will probably   require administration privileges.   If
you don't have such privileges, you are free to  install the sources and binary
packages wherever you want in your filesystem, provided  that the pathname does
not contain white-space or other  characters that are interpreted specially  by
the shell and some other programs.  A safe bet is to  use only letters, digits,
underscores  and dashes. The rest of  this  document will  assume  that you are
using {\tt  /opt/openrobots}.  You should adapt  this path to whatever location
you choosed.


\subsection{Getting robotpkg for the first time} % -------------------------

robotpkg  will {\em never} require administration  privileges by itself.  So we
recommend that you set up the {\tt  /opt/openrobots} directory with read, write
and execute permissions for your regular user  name and then  only work as this
user afterwards. If something ever goes really  wrong, you might thank yourself
later that you did so\ldots This  can be done with  the following commands in a
shell:

\begin{verbatim}
% sudo mkdir -p /opt/openrobots
% sudo chown `id -u` /opt/openrobots
\end{verbatim}

At  the    moment,  robotpkg   is      only   distributed    {\em  via}     the
\href{http://git-scm.com/}{\tt git}  software content  management  system. {\tt
git} will probably be available on your system but if you don't have it readily
installed   or if  you  are   unsure  about  it,   contact your  local   system
administrator.

There are two download methods: the anonymous one and the authenticated
one. The two methods are described here.


\subsubsection{The anonymous download}

Anonymous  download is perfect  if  you don't intend  to  work on  the robotpkg
infrastructure itself, nor commit any changes or packages additions back to the
robotpkg main repository.  This is the recommended way  to go: it will fit most
users' usage while still leaving the possibility to send feedback via patches.

As your regular user, simply run in a shell:

\begin{verbatim}
% cd /opt/openrobots
% git clone http://softs.laas.fr/git/robots/robotpkg.git
\end{verbatim}


\subsubsection{The authenticated download}

Authenticated download requires a valid login  on the main robotpkg repository,
and  will give you  full commit access to this   repository. Assuming your user
name is ``{\tt user}'', run the following:

\begin{verbatim}
% cd /opt/openrobots
% git clone ssh://user@softs.laas.fr/git/robots/robotpkg
\end{verbatim}


\subsection{Keeping robotpkg up-to-date} % ---------------------------------

robotpkg is  a living  thing: updates  to the packages  are made  perdiodicaly,
problems are fixed,  enhancements are developed\ldots  In order to get the most
recent packages descriptions, you should keep your robotpkg copy up-to-date by
regularly running {\tt git pull}:

\begin{verbatim}
% cd /opt/openrobots/robotpkg
% git pull
\end{verbatim}

When you update robotpkg, the git program will only  touch those files that are
registered in the git repository. That means that any packages that you created
on your own will stay unmodified. If you change files that  are managed by git,
later updates will try to merge your changes with  those that have been done by
others. See the git-pull manual for details.

If you want  to be informed  of package additions  and other  updates, a public
mailing    list  is   available    for   your    reading   pleasure.  Go     to
\url{https://sympa.laas.fr/sympa/info/robotpkg}    for   more  information  and
subscription.

% $LAAS: bootstrapping.tex 2010/06/23 15:04:49 mallet $
%
% Copyright (c) 2009-2010 LAAS/CNRS
% All rights reserved.
%
% Permission to use, copy, modify, and distribute this software for any purpose
% with or without   fee is hereby granted, provided   that the above  copyright
% notice and this permission notice appear in all copies.
%
% THE SOFTWARE IS PROVIDED "AS IS" AND THE AUTHOR DISCLAIMS ALL WARRANTIES WITH
% REGARD TO THIS  SOFTWARE INCLUDING ALL  IMPLIED WARRANTIES OF MERCHANTABILITY
% AND FITNESS. IN NO EVENT SHALL THE AUTHOR  BE LIABLE FOR ANY SPECIAL, DIRECT,
% INDIRECT, OR CONSEQUENTIAL DAMAGES OR  ANY DAMAGES WHATSOEVER RESULTING  FROM
% LOSS OF USE, DATA OR PROFITS, WHETHER IN AN ACTION OF CONTRACT, NEGLIGENCE OR
% OTHER TORTIOUS ACTION,   ARISING OUT OF OR IN    CONNECTION WITH THE USE   OR
% PERFORMANCE OF THIS SOFTWARE.
%
%                                             Anthony Mallet on Sun Jan 11 2009
%

\section{Bootstrapping robotpkg} % -----------------------------------------
\label{section:bootstrapping}

Once you have  downloaded the robotpkg sources  or the binary bootstrap kit  as
described  in  \xref{section:getting}{Section~\ref{section:getting}}, a minimal
set  of  the administrative package management  utilities  must be installed on
your system  before you  can  use robotpkg.   This  is  called the  ``bootstrap
phase'' and  should   be done only   once,  the very  first  time you  download
robotpkg.


\subsection{Bootstrapping via the binary kit} % ----------------------------

At the moment, the binary bootstrap kit is not available. Please bootstrap {\tt
robotpkg} as described in the next section.


\subsection{Bootstrapping from source} % -----------------------------------

You will  need a working C compiler  and the GNU-make   utility version 3.81 or
later.    If you have  extracted  the  robotpkg  archive  into  the standard {\tt
/opt/openrobots/robotpkg} location, installing the   bootstrap kit from  source
should then be as simple as:

\begin{verbatim}
% cd /opt/openrobots/robotpkg/bootstrap
% ./bootstrap
\end{verbatim}

This will  install various utilities   into {\tt /opt/openrobots/sbin}.

Should you prefer another installation path, you could use the {\tt -{}-prefix}
option to  change the default  installation prefix.  For  instance, configuring
robotpkg  to  install programs  into  the  openrobots  directory in  your  home
directory can be done like this:

\begin{verbatim}
% cd robotpkg/bootstrap
% ./bootstrap --prefix=${HOME}/openrobots
\end{verbatim}

{\bf  After the  bootstrap script  has run,  a message  indicating  the success
should be  displayed.  If  you choosed a  non-standard installation  path, read
this message carefuly}, as it contains  instructions that you have to follow in
order  to  setup your  shell  environment  correctly.   These instructions  are
described in the next section.


\subsubsection{Configuring your environment} % -----------------------------

If  you configured robotpkg,   during the bootstrap  phase,  to install to some
other location   than {\tt /opt/openrobots}, you  have   to setup manually your
shell environment so that it contains a few  variables holding the installation
path.  Assuming  you invoked bootstrap with {\tt --prefix=/path/to/openrobots},
you have two options that are compatible with each other:

\begin{itemize}
   \item Add  the directory {\tt  /path/to/openrobots/sbin}  to your {\tt PATH}
   variable. robotpkg will    then be able  to find    its administrative tools
   automatically and from that recover other configuration information. This is
   the preferred method.

   \item Create the environment variable {\tt ROBOTPKG\_BASE} and set its value
   to {\tt /path/to/openrobots}.  robotpkg will  look for this variable  first,
   so it takes precedence over the  first method.  This is  the method you have
   to choose  if  you have  configured  several instances  of robotpkg  in your
   system. This is ony useful in some circumstances and is not normally needed.
\end{itemize}

If  you  don't know  how  to setup   environment variables  permanently in your
system,  please  refer  to  your shell's  manual  or contact  your local system
administrator.


\subsubsection{The bootstrap script usage} % -------------------------------

The {\tt bootstrap} script will by default install the package administrative
tools in {\tt /opt/openrobots/sbin}, use {\tt gcc} as the C compiler and {\tt
make} as the GNU-make program. This behaviour can be fine-tuned by using the
following options:

\begin{description}
   \item[\tt   -{}-prefix <path>]   will   select the  prefix  location where
   programs will be installed in.

   \item[\tt -{}-sysconfdir <path>] defaults to {\tt <prefix>/etc}. This is the
   path to the robotpkg configuration file.  Other packages configuration files
   (if any) will also be stored in this directory.

   \item[\tt -{}-pkgdbdir  <path>] defaults to {\tt  <prefix>/var/db/pkg}. This
   is the path  to the package database  directory  where robotpkg will  do its
   internal bookkeeping.

   \item[\tt -{}-compiler <program>] defaults to {\tt gcc}.  Use this option if
   you want to use a different C compiler.

   \item[\tt -{}-make <program>] defaults to {\tt make}. Use this option if you
   want to use a different make program. This program should be compatible with
   GNU-make.

   \item[\tt -{}-help]  displays  the {\tt bootstrap} usage.  The comprehensive
   list of recognized options will be displayed.
\end{description}

%
% Copyright (c) 2009-2010,2013 LAAS/CNRS
% All rights reserved.
%
% Permission to use, copy, modify, and distribute this software for any purpose
% with or without   fee is hereby granted, provided   that the above  copyright
% notice and this permission notice appear in all copies.
%
% THE SOFTWARE IS PROVIDED "AS IS" AND THE AUTHOR DISCLAIMS ALL WARRANTIES WITH
% REGARD TO THIS  SOFTWARE INCLUDING ALL  IMPLIED WARRANTIES OF MERCHANTABILITY
% AND FITNESS. IN NO EVENT SHALL THE AUTHOR  BE LIABLE FOR ANY SPECIAL, DIRECT,
% INDIRECT, OR CONSEQUENTIAL DAMAGES OR  ANY DAMAGES WHATSOEVER RESULTING  FROM
% LOSS OF USE, DATA OR PROFITS, WHETHER IN AN ACTION OF CONTRACT, NEGLIGENCE OR
% OTHER TORTIOUS ACTION,   ARISING OUT OF OR IN    CONNECTION WITH THE USE   OR
% PERFORMANCE OF THIS SOFTWARE.
%
%                                             Anthony Mallet on Sun Jan 11 2009
%

\section{Using robotpkg} % -------------------------------------------------

After obtaining \robotpkg, the  {\tt robotpkg} directory now  contains a set of
packages, organized   into  categories.  You can   browse  the online  index of
packages, or run {\tt  make index} from the {\tt  robotpkg} directory to  build
local {\tt index.html}  files for all  packages, viewable with any web  browser
such as {\tt lynx} or {\tt firefox}.

\robotpkg is  essentially based on the  {\tt make(1)} program.  All actions are
triggered by invoking {\tt make} with the proper target. The following sections
document          the           most          useful          ones          and
\xref{section:using:targets}{section~\ref{section:using:targets}} recaps a more
comprehensive list.


\subsection{Building packages from source} % -------------------------------

The  first step for  building  a  package  is  downloading the {\em  distfiles}
(i.e. the unmodified  source). If they have not  yet been downloaded, \robotpkg
will  fetch them automatically  and place them  in the {\tt robotpkg/distfiles}
directory.

Once the software  has  been downloaded,  any  patches will be applied  and the
package will  be compiled for  you.  This may  take some time depending on your
computer, and how many other packages the software depends on and their compile
time.

For  example,  type the following  commands  at the shell   prompt to build the
robotpkg documentation package:

\begin{verbatim}
% cd /opt/openrobots/robotpkg
% cd doc/robotpkg
% make
\end{verbatim}

The next  stage is  to  actually install the newly   compiled package onto your
system. While you   are still in  the directory  for whatever package  you  are
installing, you can do this by entering:

\begin{verbatim}
% make install
\end{verbatim}

Installing the package on your system does  not require you  to be root (except
for a few specific  packages). However, if   you bootstraped with a  prefix for
which   you  don't   have  writing   permissions,    \robotpkg   has a     {\rm
just-in-time-sudo}  feature,  which allows you to  become  {\tt  root}  for the
actual installation step.

That's it, the software should now be installed   under  the prefix  of the
packages tree --- {\tt /opt/openrobots} by default --- and setup for use.

You can now enter:

\begin{verbatim}
% make clean
\end{verbatim}

to remove the compiled files in the work  directory, as you shouldn't need them
any more. If  other packages were also  added to your system (dependencies)  to
allow your program to compile, you can also tidy these up with the command:

\begin{verbatim}
% make clean-depends
\end{verbatim}

Since  the three tasks of building,  installing and  cleaning correspond to the
typical usage of \robotpkg, a helper target doing all these tasks exists and is
called {\tt update}. Thus,  to intall a package  with a single command, you can
simply run:

\begin{verbatim}
% make update
\end{verbatim}

In addition, {\tt  make update} will  also recompile all the installed packages
that were depending on the package that you are updating. This can be quite
time consuming if you are updating a low-level package. Also, note that all
packages that depend on the package you are updating will be deinstalled
first and unavailable in your system until all packages are recompiled and
reinstalled.

%
%    <para>Some packages look in &mk.conf; to
%    alter some configuration options at build time.  Have a look at
%    <filename>pkgsrc/mk/defaults/mk.conf</filename> to get an overview
%    of what will be set there by default.  Environment variables such
%    as <varname>LOCALBASE</varname> can be set in
%    &mk.conf; to save having to remember to
%    set them each time you want to use pkgsrc.</para>
%

Occasionally, people want to ``look under the covers'' to see  what is going on
when a  package  is building  or being  installed.  This may  be for  debugging
purposes, or  out  of simple curiosity. A  number  of utility values have  been
added to help with this.

\begin{enumerate}

\item If you invoke the {\tt make} command with {\tt PKG\_DEBUG\_LEVEL=1}, then
      a huge amount of information will be displayed. For example,

\begin{verbatim}
% make patch PKG_DEBUG_LEVEL=1
\end{verbatim}

      will show all the commands that are invoked, up to and including the
      ``patch'' stage. Using {\tt PKG\_DEBUG\_LEVEL=2} will give you even
      more details.

\item If you want to know the value of a certain {\tt make} definition, then
   the {\tt VARNAME} variable   should be used,  in  conjunction with the  {\tt
   show-var} target.  e.g.  to show the  expansion  of the  {\tt make} variable
   {\tt LOCALBASE}:

\begin{verbatim}
% make show-var VARNAME=LOCALBASE
\end{verbatim}

\end{enumerate}

%    <para>If you want to install a binary package that you've either
%    created yourself (see next section), that you put into
%    pkgsrc/packages manually or that is located on a remote FTP
%    server, you can use the "bin-install" target. This target will
%    install a binary package - if available - via &man.pkg.add.1;,
%    else do a <command>make package</command>.  The list of remote FTP
%    sites searched is kept in the variable
%    <varname>BINPKG_SITES</varname>, which defaults to
%    ftp.NetBSD.org. Any flags that should be added to &man.pkg.add.1;
%    can be put into <varname>BIN_INSTALL_FLAGS</varname>.  See
%    <filename>pkgsrc/mk/defaults/mk.conf</filename> for more
%    details.</para>


%    <para>A final word of warning: If you set up a system that has a
%    non-standard setting for <varname>LOCALBASE</varname>, be sure to
%    set that before any packages are installed, as you cannot use
%    several directories for the same purpose. Doing so will result in
%    pkgsrc not being able to properly detect your installed packages,
%    and fail miserably. Note also that precompiled binary packages are
%    usually built with the default <varname>LOCALBASE</varname> of
%    <filename>/usr/pkg</filename>, and that you should
%    <emphasis>not</emphasis> install any if you use a non-standard
%    <varname>LOCALBASE</varname>.</para>


\subsection{Building packages from a repository checkout} % ----------------
\label{section:using:checkout}

Before building a  package, \robotpkg fetches the sources  from the official(s)
download  location(s),  as  instructed  by the  {\tt  MASTER\_SITES}  variable.
This is the standard and expected behaviour when you work with stable packages.

Occasionally, though,  it is useful to fetch  a snapshot of the  sources from a
development repository. For instance, one  might want to quickly test a release
candidate of a  package, or fix a simple  bug and create a patch  from the fix.
Whenever a package defines  the {\tt MASTER\_REPOSITORY} variable, \robotpkg is
able to temporarily  work with the repository defined in  this variable. At the
moment, {\tt cvs}, {\tt svn} and {\tt git} repositories are supported.

To enable this feature for a given package,  you have to first instruct
\robotpkg to work from a '{\tt checkout}' (instead of the stable releases) by
doing '{\tt make checkout}' in the package directory. For instance:

\begin{verbatim}
% cd robotpkg/foo/bar
% make checkout
\end{verbatim}

This sets  a permanent flag in the  {\em working} directory of  the package and
the {\em checkout}  configuration option will be retained  until the next '{\tt
make clean}'. After a '{\tt make  clean}', the configuration option is set back
to its default and \robotpkg will  work again with stable releases. This option
is set on a {\em per} package  basis only: configuring one package to work with
checkouts does not affect the behaviour of other packages.

After a '{\tt make checkout}' (and until a '{\tt make clean}'), the package has
a regular  checkout in its {\em  working} subdirectory.  You  can thus manually
edit, commit, switch branches, etc.  in  the package sources, like in any other
repository, by  first {\tt  cd}ing into the  working directory, then  using the
usual repository commands ({\tt cvs}, {\tt svn} or {\tt git}).

Of  course, the  individual  \robotpkg  targets are  still  available from  the
package  entry in  the robotpkg  hierarchy.  You  can for  instance  {\tt 'make
patch'}, {\tt 'configure'}, {\tt 'build'}, {\tt 'install'} or {\tt 'update'} as
usual. Note that  \robotpkg is not exactly stateless, and  this is most visible
when  working with  checkouts:  for  instance, after  a  successful {\tt  'make
build'}, you  have to do {\tt 'make  rebuild'} to force rebuilding  if you have
modified  the  sources.   The  same  holds  for  {\tt   'configure'}  (do  {\tt
'reconfigure'})  or {\tt  'install'} (do  {'reinstall'}, but  since  you cannot
install a package  twice, you normally have to use {\tt  'make replace'} in the
particular case of reinstalling a package).

The  {\tt  'clean'}  target  is  special,  in  that  it  removes  the  checkout
configuration  option and  all checkouted  sources, including  locally modified
sources. In order to prevent accidental deletion of precious files, you have to
confirm the cleanign with {\tt 'clean confirm'}, as in:

\begin{verbatim}
% make clean confirm
\end{verbatim}

A  final  remark:  we {\em  STRONGLY  DISCOURAGE}  the  use  of robotpkg  as  a
development tool  (i.e. using the {\tt  'checkout'} feature on  a {\em regular}
basis), for at least two reasons:

\begin{itemize}
   \item \robotpkg  is not designed  for this: it  will not really help  you in
   your  daily   development  work,   compared  to  the   manual  configuration
   installation of the software. It will sometimes create even more trouble, by
   ensuring  that all  the software  depending  on the  checkouted software  is
   up-to-date, which is not necessarily something you want to do every time you
   compile.

   \item  A checkout  breaks the  notion  of 'release'  and you  loose all  the
   benefits from working with packages.  In particular, you have no clear state
   of what is installed: you cannot easily reproduce the situation of time T at
   time T+n and don't know precisely  who requires which version of what. It is
   much  more  efficient and  robust  to release  frequently  a  software in  a
   development phase, than using a {\em rolling release} approach.
\end{itemize}

In our opinion, the {\tt 'checkout'}  target use should be limited to testing a
release candidate or  quickly fix a bug  and create a patch from  the fix, that
you commit upstream and put in the patches/ directory until the next release.


\subsection{Installing binary packages} % ----------------------------------

At the moment, installing binary packages is not documented.

\subsection{Removing packages} % -------------------------------------------

To deinstall a package, it does not matter whether it was installed from source
code or  from a  binary package.  The  {\tt robotpkg\_delete} command  does not
know it  anyway.  To delete a  package, you can just  run {\tt robotpkg\_delete
<package-name>}.  The package name can be given with or without version number.
Wildcards can  also be used  to deinstall a  set of packages, for  example {\tt
*genom*} all  packages whose  name contain  the word {\tt  genom}.  Be  sure to
include them  in quotes,  so that the  shell does  not expand them  before {\tt
robotpkg\_delete} sees them.

The {\tt -r} option is very powerful: it  removes all the packages that require
the package in question and then removes the package itself. For example:

\begin{verbatim}
% robotpkg_delete -r genom
\end{verbatim}

will remove genom and all the packages that used it; this allows
upgrading the {\tt genom} package.

\subsection{Getting information about installed packages} % ----------------

The {\tt  robotpkg\_info} shows information about installed  packages or binary
package files.


\subsection{Other administrative functions} % ------------------------------

The  {\tt robotpkg\_admin}  executes  various administrative  functions on  the
package system.

\subsection{Available {\tt make} targets} % --------------------------------
\label{section:using:targets}

The following targets are available in a package directory. They can be invoked
by   running  {\tt   make  <target>}   after   {\tt  cd}ing   into  some   {\tt
robotpkg/category/package}.

\subsubsection{Source manipulation}
\begin{description}
   \item[{\tt fetch}] Download the {\tt\$\{DISTFILES\}}.

   \item[{\tt extract}] Extract the contents of {\tt\$\{DISTFILES\}} into the
   work directory {\tt\$\{WRKDIR\}}.

   \item[{\tt patch}] Apply local patches available in {\tt\$\{PATCHDIR\}}
   (usually the {\tt patches} directory in the package).

   \item[{\tt checkout}] Extract the sources in {\tt\$\{WRKDIR\}} from
   {\tt\$\{MASTER\_REPOSITORY\}} instead of {\tt\$\{MASTER\_SITES\}}. This can
   be used to fetch a not yet released version instead of the latest
   release. This is mutually exclusive with the {\tt fetch} and {\tt extract}
   targets. See
   \xref{section:using:checkout}{section~\ref{section:using:checkout}} for
   details.

   \item[{\tt configure}] Perform the necessary actions to configure the
   sources. This may for instance involve running {\tt configure} or {\tt
   cmake}. If no configuration is required, this step simply does nothing.

   \item[{\tt build}] Or  just {\tt make}, the default  target. It compiles the
   package locally in {\tt\$\{WRKDIR\}}.

   \item[{\tt install}] Install the package into {\tt\$\{PREFIX\}}. The package
   is then available to the rest of the system. If an older version of the
   package is installed and required by other packages, this target requires
   confirmation. Otherwise, any older version of the package is first
   deinstalled.

   \item[{\tt replace}] Same as {\tt install}, but does not remove packages
   that depend on the replaced package. This saves some time, since already
   installed package are not touched, but if the replaced package is
   incompatible with the older version, you will run into trouble. Use with
   care and when you know what you are doing.

   \item[{\tt clean}] Tidy the work directory and removes {\tt\$\{WRKDIR\}}. If
   the package was extracted using {\tt checkout}, this target requires
   confirmation as it may delete locally modified files that will be lost.

   \item[{\tt update}] This is a shortcut target for {\tt fetch}, {\tt
   extract}, {\tt configure}, {\tt build}, {\tt install} and {\tt clean}. If
   the package is already installed and up-to-date, the target asks for
   confirmation.

\end{description}

\subsubsection{Introspection}
\begin{description}
   \item[{\tt show-options}] Display the list of available alternatives (see
   \xref{section:configuring:alternatives}%
   {section~\ref{section:configuring:alternatives}})
   and build options (see
   \xref{section:configuring:build_options}%
   {section~\ref{section:configuring:build_options}}).

   \item[{\tt show-depends}] Recursively display all the required dependencies
   of a package. The results are splitted between system and \robotpkg
   dependencies, and missing dependencies are indicated.

   \item[{\tt show-var}] Display the contents of a variable. This must be
   invoked as {\tt make show-var VARNAME=foo}, where {\tt foo} is the name of
   the variable to be displayed.
\end{description}

\subsubsection{Package sets}

\begin{description}
   \item[{\tt fetch-depends}, {\tt replace-depends}, {\tt update-depends}, {\tt
   clean-depends}]
   This runs the same action as {\tt fetch}, {\tt replace}, {\tt update} or
   {\tt clean} (respectively), but on all dependencies of the package,
   including the package itself. Useful to update a meta-packages, for instance.

   \item[{\tt fetch-<set>}, {\tt replace-<set>}, {\tt update-<set>}, {\tt
   clean-<set>}]
   This runs the same action as {\tt fetch}, {\tt replace}, {\tt update} or
   {\tt clean} (respectively), but on all members of the package set named {\tt
   <set>}. See
   \xref{section:configuring:sets}{section~\ref{section:configuring:sets}} 
   for an explanation of package sets and how to configure them.

\end{description}

% $LAAS: configuring.tex 2010/09/09 15:46:20 mallet $
%
% Copyright (c) 2009-2010 LAAS/CNRS
% All rights reserved.
%
% Permission to use, copy, modify, and distribute this software for any purpose
% with or without   fee is hereby granted, provided   that the above  copyright
% notice and this permission notice appear in all copies.
%
% THE SOFTWARE IS PROVIDED "AS IS" AND THE AUTHOR DISCLAIMS ALL WARRANTIES WITH
% REGARD TO THIS  SOFTWARE INCLUDING ALL  IMPLIED WARRANTIES OF MERCHANTABILITY
% AND FITNESS. IN NO EVENT SHALL THE AUTHOR  BE LIABLE FOR ANY SPECIAL, DIRECT,
% INDIRECT, OR CONSEQUENTIAL DAMAGES OR  ANY DAMAGES WHATSOEVER RESULTING  FROM
% LOSS OF USE, DATA OR PROFITS, WHETHER IN AN ACTION OF CONTRACT, NEGLIGENCE OR
% OTHER TORTIOUS ACTION,   ARISING OUT OF OR IN    CONNECTION WITH THE USE   OR
% PERFORMANCE OF THIS SOFTWARE.
%
%                                             Anthony Mallet on Wed Mar  4 2009
%

\section{Configuring robotpkg} % -------------------------------------------

The whole \robotpkg system is configured  {\em via} a single, centralized file,
called {\tt   robotpkg.conf}  and  placed  in  the   {\tt  /opt/openrobots/etc}
directory  by default.  This location  might be redefined  during the bootstrap
phase,  see  \xref{section:bootstrapping}{Section~\ref{section:bootstrapping}}.
During    the  bootstrap, an initial configuration     file is created with the
settings you provided to {\tt bootstrap}.

The  format of  the configuration file   is that of   the usual GNU style  {\tt
Makefile}s. The whole \robotpkg configuration  is done by setting variables  in
this  file. Note that  you can  define all  kinds of  variables, and no special
error checking (for example for spelling mistakes)  takes place, so you have to
try it out to see if it works.


\subsection{Selecting build options} % -------------------------------------

Some packages have   build time options, usually   to select between  different
dependencies,  enable  optional   support for    big   dependencies or   enable
experimental features.

To see   which options, if  any, a   package supports,  and  which  options are
mutually exclusive, run {\tt make show-options}, for example:

\begin{verbatim}
Any of the following general options may be selected:
    debug   Produce debugging information for binary programs
    doc     Compile documentation material
    lex     Use lex in place of flex
    tcl     Enable support for TCL clients
    yacc    Use yacc in place of bison

These options are enabled by default:
    doc tcl

These options are currently enabled:
    doc tcl
\end{verbatim}

The following variables can be defined  in {\tt robotpkg.conf} to select which
options to enable for a package:

\begin{itemize}
   \item {\tt PKG\_DEFAULT\_OPTIONS}, which can be used to select or  disable
   options  for  all packages  that  support them,

   \item {\tt PKG\_OPTIONS.<pkgbase>}, which can  be   used  to select  or
   disable   options specifically for package {\tt pkgbase}. Options listed
   in these variables are selected, options preceded by {\tt -} are disabled.
\end{itemize}

A few examples:

\begin{verbatim}
PKG_DEFAULT_OPTIONS=    debug
PKG_OPTIONS.genom=      doc -tcl
\end{verbatim}

It is important to note  that options that were  specifically suggested by  the
package  maintainer must be  explicitely removed if you  do not wish to include
the option.  If you  are unsure you  can view the current  state with {\tt make
show-options}.

The following settings are  consulted in the order  given, and the last setting
that selects or disables an option is used:

\begin{enumerate}
   \item the default options as suggested by the package maintainer,

   \item {\tt PKG\_DEFAULT\_OPTIONS},

   \item {\tt PKG\_OPTIONS.<pkgbase>}
\end{enumerate}

For groups of mutually exclusive options, the last option selected is used, all
others are automatically  disabled.  If  an option of  the  group is explicitly
disabled, the previously selected option,  if any, is used.   It is an error if
no option from  a  required group  of  options is  selected, and  building  the
package will fail.


\subsection{Defining collections of packages} % ----------------------------

Instead of installing, removing or updating packages one-by-one, you can define
collections  of  packages  in  your  {\tt  robotpkg.conf}.  Once  one  or  more
collections  are defined,  they enable  special targets  that work  on  all the
packages of a collection.

To define a collection, you have to give it a name and list the set of packages
forming  the  collection   in  the  special  {\tt  PKGSET}   variable  in  {\tt
robotpkg.conf}. The syntax is the following:

\begin{verbatim}
PKGSET.<name> = <list>
\end{verbatim}

where {\tt <name>} is the name of the collection (any string is valid) and {\tt
<list>}  is  the  list  of  packages  in  the  collection,  in  the  form  {\tt
<category>/<name>}. For instance,

\begin{verbatim}
PKGSET.myset = architecture/genom devel/pocolibs
\end{verbatim}

defines  a collection named  {\tt myset}  that contains  the two  packages {\tt
genom} and {\tt pocolibs}.

For each collection {\tt <name>}  defined in {\tt robotpkg.conf}, the following
targets  are available:  {\tt clean-<name>},  {\tt  clean-depends-<name>}, {\tt
fetch-<name>},    {\tt     extract-<name>},    {\tt    install-<name>},    {\tt
replace-<name>},  {\tt update-<name>}. They  perform the  same action  as their
respective counterpart without  {\tt -<name>} suffix, expect that  they work on
all packages  of the set. In addition,  for the {\tt replace}  and {\tt update}
targets, they sort the packages in  the order of their dependencies so that the
job is done a sensible order.

For  the user  convenience,  A  special {\tt  installed}  collection is  always
defined  and represents all  currently installed  packages. Thus,  invoking the
{\tt update-installed} target will update all currently installed packages.

Two {\tt robotpkg.conf} variables affect the behaviour of {\tt robotpkg}
regarding packages sets:

\begin{description}
   \item[PKGSET\_FAILSAFE] When working on a set, and this variable is set to
   yes, robotpkg will continue with further packages instead of stopping on an
   error. If set to 'no', stop on first error. Default: no.

   \item[PKGSET\_STRICT] Specify if package sets should be considered as
   'strict' or include dependencies of packages defined in the set. If set to
   'yes', only package strictly defined in sets are considered. If set to 'no',
   dependencies of packages listed in sets are added to their respective
   sets. Default: no.
\end{description}

\subsection{General configuration variables} % -----------------------------

In  this  section,  you can   find some  variables   that apply  to  all \robotpkg
packages.

% A complete  list of the variables that  can be configured by the user
% is  available in <filename>mk/defaults/mk.conf</filename>,  together with  some
% comments that describe each variable's intent.</para>

\begin{description}
   \item[ACCEPTABLE\_LICENSES] List of acceptable licenses. Whenever you try to
   build a package  whose license is  not in this  list, you will get  an error
   message that includes instructions on how to change this variable.

   \item[DISTDIR] Where to store the downloaded copies of the original source
   distributions used for building \robotpkg packages. The default is
   {\tt \${ROBOTPKG\_DIR}/distfiles}.

   \item[PACKAGES] The top level directory for the binary packages. The default
   is  {\tt \${ROBOTPKG\_DIR}/packages}.

%   \item[PKG\_DBDIR] Where the database  about  installed packages  is  stored.
%   The default is {\tt /opt/openrobots/var/db/pkg}.

   \item[MASTER\_SITE\_BACKUP] List  of backup locations for distribution files
   if not found locally  or  in {\tt \${MASTER\_SITES}}.  The default  is\\
   {\tt http://softs.laas.fr/openrobots/robotpkg/distfiles/}.

   \item[PKG\_DEBUG\_LEVEL] The  level of debugging  output  which is displayed
   whilst making and installing the package.  The  default value for this is 0,
   which will not  display the commands as they  are executed (normal, default,
   quiet  operation); the value 1 will  display all shell commands before their
   invocation,  and  the value  2 will  display both the  shell commands before
   their invocation, and their actual execution progress with {\tt set -x}.
\end{description}


\subsection{Variables affecting the build process} % -----------------------

\begin{description}
   \item[WRKOBJDIR] The top level   directory where, if defined,  the  separate
   working directories will get created.  This is useful for building  packages
   on a different filesystem than the \robotpkg sources.

   \item[DEPENDS\_TARGET] By default,  dependencies are only installed,  and no
   binary package is  created  for them. You  can  set  this variable  to  {\tt
   package}   to   automatically  create    binary  packages   after installing
   dependencies.

   \item[LOCALBASE] Where packages will be installed. The default value is {\tt
   /opt/openrobots}.  Do not  mix     binary  packages with    different values
   of {\tt LOCALBASE}s!


%   \item[GCC\_REQUIRED]  This specifies requirements  on  the version of GCC to
%   use  when  building  packages.   This  variable  should contain   a  list of
%   constraints in the form {\tt  \{<=,<,-,>,>=\}n}. E.g.  to specifiy a minimum
%   version of 4.2  use ``{\tt >=4.2}'', or to  specifiy gcc version 4  only use
%   ``{\tt >=4 <5}''.

\end{description}


\subsection{Additional flags to the compiler} % ----------------------------

If you wish  to set compiler variables   such as {\tt CFLAGS},  {\tt CXXFLAGS},
{\tt FFLAGS} ... please make sure to use  the {\tt +=}  operator instead of the
{\tt {=}} operator:

\begin{verbatim}
CFLAGS+= -your -flags
\end{verbatim}

Using {\tt CFLAGS=} (i.e.  without the ``{\tt +}'') may  lead to  problems with
packages that need to add their own flags.

If you want  to pass flags  to the linker, both in  the configure  step and the
build step, you  can do this  in  two ways.   Either set {\tt  LDFLAGS} or {\tt
LIBS}.  The difference between  the two is that  {\tt LIBS} will be appended to
the command line, while {\tt LDFLAGS} come earlier. {\tt LDFLAGS} is pre-loaded
with rpath settings   for machines that support  it.  As with {\tt CFLAGS}  you
should use the {\tt +=} operator:

\begin{verbatim}
LDFLAGS+= -your -linkerflags
\end{verbatim}


\chapter{The robotpkg developer's guide}
\label{chapter:developer}

This part of the documentation deals with creating and modifying packages.

% $LAAS: pkgvars.tex 2010/10/28 15:08:35 mallet $
%
% Copyright (c) 2010 LAAS/CNRS
% All rights reserved.
%
% Permission to use, copy, modify, and distribute this software for any purpose
% with or without   fee is hereby granted, provided   that the above  copyright
% notice and this permission notice appear in all copies.
%
% THE SOFTWARE IS PROVIDED "AS IS" AND THE AUTHOR DISCLAIMS ALL WARRANTIES WITH
% REGARD TO THIS  SOFTWARE INCLUDING ALL  IMPLIED WARRANTIES OF MERCHANTABILITY
% AND FITNESS. IN NO EVENT SHALL THE AUTHOR  BE LIABLE FOR ANY SPECIAL, DIRECT,
% INDIRECT, OR CONSEQUENTIAL DAMAGES OR  ANY DAMAGES WHATSOEVER RESULTING  FROM
% LOSS OF USE, DATA OR PROFITS, WHETHER IN AN ACTION OF CONTRACT, NEGLIGENCE OR
% OTHER TORTIOUS ACTION,   ARISING OUT OF OR IN    CONNECTION WITH THE USE   OR
% PERFORMANCE OF THIS SOFTWARE.
%
%                                             Anthony Mallet on Wed Oct  6 2010
%
\section{Creating a new package} % -----------------------------------------
\label{section:pkgvars}

Whenever you're preparing a package, there are a number of files involved which
are described in the following sections.

\subsection{Makefile} % ----------------------------------------------------
\label{subsection:makefile}

Building, installation and creation of a package are all controlled by the
package's Makefile. The Makefile describes various things about a package,
for example from where to get it, how to configure, build, and install it.

A package Makefile contains several sections that describe the package.

In the first section there are the following variables, which should appear
exactly in the order given here. The order and grouping of the variables is
mostly historical and has no further meaning.

\begin{description}
   \item[MASTER\_SITES] In simple cases, {\tt MASTER\_SITES}  defines all URLs
   from where the distfile, whose name is derived from the {\tt DISTNAME}
   variable, is fetched.

   When actually fetching the distfiles, each item from {\tt MASTER\_SITES}
   gets the name of each distfile appended to it, without an intermediate
   slash. Therefore, all site values have to end with a slash or other
   separator character. This allows for example to set {\tt MASTER\_SITES} to a
   URL of a CGI script that gets the name of the distfile as a parameter. In
   this case, the definition would look like:
   \begin{quote}
      {\tt MASTER\_SITES=   http://www.example.com/download.cgi?file=}
   \end{quote}

   There are some predefined values for {\tt MASTER\_SITES}, which can be used
   in packages. The names of the variables should speak for themselves.
   \begin{quote}\tt
      \$\{MASTER\_SITE\_SOURCEFORGE\}\\
      \$\{MASTER\_SITE\_GNU\}\\
      \$\{MASTER\_SITE\_OPENROBOTS\}
   \end{quote}

   If you choose one of these predefined sites, you may want to specify a
   subdirectory of that site. Since these macros may expand to more than one
   actual site, {\em you must} use the following construct to specify a
   subdirectory:
   \begin{quote}\tt
      MASTER\_SITES=~\$\{MASTER\_SITE\_SOURCEFORGE:=project\_name/\}
   \end{quote}
   Note the trailing slash after the subdirectory name.

   \smallbreak
   \item[FETCH\_METHOD] This is the method used to download the distfile from
   {\tt MASTER\_SITES}. It defaults to '{\tt archive}' which corresponds to the
   normal situation where distfile is an archive available from {\tt
   MASTER\_SITES}, so it normally needs not to be set.

   However, it can happen that a software provider does not provide any archive
   available for download but has only a public repository. In this case, {\tt
   FETCH\_METHOD} can be set to {\tt cvs}, {\tt git} or {\tt svn} according to
   the kind of repository available. {\tt MASTER\_SITES} is then interpreted as
   a repository of the form {\tt url[@revision[+module]]}, where the bits
   between square brackets are optional and refer to a particular revision and
   module in the repository located at {\tt url}. {\tt url} can take any form
   supported by the underlying fetch tool ({\tt cvs}, {\tt git} or {\tt
   svn}). It is {\em strongly} advised to define at least a specific revision
   to be checked out, so that the package can be reproducibly installed in a
   known state.

\end{description}

The second section contains information about separately downloaded patches, if any.

\begin{description}

   \item[PATCHFILES] Name(s) of additional files that contain distribution
   patches distributed by the author or other maintainers. There is no
   default. robotpkg will look for them at {\tt    PATCH\_SITES}. They will
   automatically be uncompressed before patching if    the names end with .gz
   or .Z.

   \item[PATCH\_SITES] Primary location(s) for distribution patch files (see
   {\tt PATCHFILES} above) if not found locally.

\end{description}

The third section contains the following variables.

\begin{description}

   \item[MAINTAINER] is the email address of the person who feels responsible
   for this package, and who is most likely to look at problems or questions
   regarding this package. Other developers may contact the {\tt MAINTAINER}
   before making changes to the package, but are not required to do so. When
   packaging a new program, set {\tt MAINTAINER} to yourself. If you really
   can't maintain the package for future updates, set it to
   {\tt \string<robotpkg@laas.fr\string>}.

   \item[HOMEPAGE] is a URL where users can find more information about the
   package.

   \item[COMMENT] is a one-line description of the package (should not include
   the package name).

   \item[LICENSE] Denoting that a package may be installed and used according
   to a particular license is done by placing the license in {\tt
   robotpkg/licenses} and setting the LICENSE variable to a string identifying
   the license file, e.g. in {\tt shell/eltclsh}:
   \begin{quote}
      LICENSE=		2-clause-bsd
   \end{quote}

   The license tag mechanism is intended to address copyright-related issues
   surrounding building, installing and using a package, and not to address
   redistribution issues (see RESTRICTED and NO\_PUBLIC\_SRC, etc.). Packages
   with redistribution restrictions should set these tags.

\end{description}


Other variables affecting the build process may be gathered in their own
section:

\begin{description}

   \item[MAKE\_JOBS\_SAFE] Whether the package supports parallel builds. If set
   to yes, at most {\tt MAKE\_JOBS} jobs are carried out in parallel. The
   default value is ``yes'', and packages that don't support it must explicitly
   set it to ``no''.

\end{description}


\subsection{distinfo} % ----------------------------------------------------
\label{subsection:distinfo}

The distinfo file contains the message digest, or checksum, of each distfile
needed for the package. This ensures that the distfiles retrieved from the
Internet have not been corrupted during transfer or altered by a malign force
to introduce a security hole. Due to recent rumor about weaknesses of digest
algorithms, all distfiles are protected using both SHA1 and RMD160 message
digests, as well as the file size.

The distinfo file also contains the checksums for all the patches found in the
patches directory (see
\xref{subsection:patches}{Section~\ref{subsection:patches}}).

To regenerate the distinfo file, use the {\tt make distinfo} or {\tt make mdi}
command.


\subsection{PLIST}
\label{subsection:PLIST}

This  file  governs the  files  that  are installed  on  your  system: all  the
binaries, manual pages, etc. There are other directives which may be entered in
this  file,  to control  the  creation and  deletion  of  directories, and  the
location of inserted files.

The  names used  in the  PLIST are  relative to  the installation  prefix ({\tt
\$\{PREFIX\}}),  which  means  that  it  cannot  register  files  outside  this
directory  (absolute path names  are not  allowed). As  a general  sanity rule,
robotpkg must  not alter  any files outside  {\tt \$\{PREFIX\}} anyway  and, in
particular, not modify automatically existing configuration files. If a package
needs  to install  files  outside {\tt  \$\{PREFIX\}},  the best  option is  to
install   them   with   robotpkg   inside  {\tt   \$\{PREFIX\}}   (e.g.    {\tt
\$\{PREFIX\}/etc} or  {\tt \$\{PREFIX\}/var}) and  create a {\tt  MESSAGE} file
that will instruct the  user to manually link or copy the  files in question to
their final location. See the package {\tt hardware/ieee1394-kmod} for an
example of such package.

In  order to  create  or  update a  {\tt  PLIST}, you  can  use  the {\tt  make
print-PLIST} command  to output  a PLIST that  matches any new  installed files
since  the  package   was  extracted.   This  command  will   generate  a  {\tt
PLIST.guess} file which  you must move manually to  {\tt PLIST} after reviewing
the result of the semi-automatic generation.


\subsection{patches/*} % ----------------------------------------------------
\label{subsection:patches}

Some packages may not work out-of-the box with robotpkg. Therefore, a number of
custom patch  files may be needed to  make the package work.  These patch files
are found in the {\tt patches/} directory. If you want to share patches between
multiple packages  in robotpkg, e.g. because  they use the  same distfiles, set
{\tt PATCHDIR} to the path where the patch files can be found, e.g.:
\begin{quote}
   PATCHDIR= ../../devel/boost/patches
\end{quote}

The file names of the patch files must be of the form {\tt patch-*}, and they
are usually named {\tt patch-[a-z][a-z]}. In  the {\em  patch} phase,  these
patches  are automatically applied  to the  files  in {\tt \$\{WRKSRC\}}
directory after extracting them, in alphabetic order.

The {\tt patch-*} files should be in {\tt diff -bu} format, and apply without a
fuzz to avoid problems.  (To force patches to apply with fuzz  you can set {\tt
PATCH\_FUZZ\_FACTOR=-F2} in a package's {\tt Makefile}).

Each patch file should be commented so that any developer who knows the code of
the application  can make some use of  the patch. Special care  should be taken
for the upstream developers, since  we generally want that they accept robotpkg
patches, so there is less work in the future. When adding a patch that corrects
a problem in the distfile (rather than e.g. enforcing robotpkg's view of where
man pages should go), send the patch as a bug report to the maintainer. This
benefits non-robotpkg users of the package, and usually makes it possible to
remove the patch in future version.

When you add or modify existing patch files, remember to generate the checksums
for the patch files by using the {\tt make mdi} command, see
\xref{subsection:distinfo}{Section~\ref{subsection:distinfo}}.


\chapter{The robotpkg infrastructure internals}
\label{chapter:internal}

\end{document} % -----------------------------------------------------------

\else
   \usepackage[T1]{fontenc}
   \usepackage{share/robotpkg}
\fi

\title{A guide to robotpkg}
\author{
   Anthony Mallet --- {\tt anthony.mallet@laas.fr}\\[1em]
   Copyright 2006-2009 \copyright LAAS/CNRS
}

\begin{document} % ---------------------------------------------------------

\frontmatter
\maketitle
\tableofcontents
\mainmatter

\chapter{Introduction}
\label{chapter:introduction}
% $LAAS: introduction.tex 2010/11/17 17:56:41 mallet $
%
% Copyright (c) 2009-2010 LAAS/CNRS
% All rights reserved.
%
% Permission to use, copy, modify, and distribute this software for any purpose
% with or without   fee is hereby granted, provided   that the above  copyright
% notice and this permission notice appear in all copies.
%
% THE SOFTWARE IS PROVIDED "AS IS" AND THE AUTHOR DISCLAIMS ALL WARRANTIES WITH
% REGARD TO THIS  SOFTWARE INCLUDING ALL  IMPLIED WARRANTIES OF MERCHANTABILITY
% AND FITNESS. IN NO EVENT SHALL THE AUTHOR  BE LIABLE FOR ANY SPECIAL, DIRECT,
% INDIRECT, OR CONSEQUENTIAL DAMAGES OR  ANY DAMAGES WHATSOEVER RESULTING  FROM
% LOSS OF USE, DATA OR PROFITS, WHETHER IN AN ACTION OF CONTRACT, NEGLIGENCE OR
% OTHER TORTIOUS ACTION,   ARISING OUT OF OR IN    CONNECTION WITH THE USE   OR
% PERFORMANCE OF THIS SOFTWARE.
%
%                                             Anthony Mallet on Sat Jan 10 2009
%

\section{What is robotpkg?} % ----------------------------------------------

The robotics research  community has always been developing  a lot of software,
in order  to illustrate theoretical concepts and  validate algorithms  on board
real robots.  A great amount of this software was  made freely available to the
community, especially for Unix-based systems,  and is usually available in form
of the source code. Therefore, before such software can be used, it needs to be
configured to  the local system, compiled and  installed.  This is exactly what
The Robotics Packages Collection (robotpkg) does.  robotpkg also has some basic
commands  to handle binary packages,  so that not  every user  has to build the
packages for himself, which is a time-costly, cumbersome and error-prone task.

The robotpkg project was initiated in the \href{http://www.laas.fr/}{Laboratory
for Analysis and Architecture of  Systems} (CNRS/LAAS), France.  The motivation
was, on the one hand,  to ease the software   maintenance tasks for the  robots
that are used there.   On the other  hand, roboticists at CNRS/LAAS have always
fostered  an  open-source  development   model  for   the   software they  were
developing.  In order to  help people  working with the  laboratory to  get the
LAAS software  running outside the laboratory,  a package management system was
necessary.

Although  robotpkg was an  innovative   project in  the robotics community  (it
started in 2006), a lot of general-purpose software packages management systems
were readily available at this time for  a great variety of Unix-based systems.
The main requirements that we wanted  robotpkg to fullfill  were listed and the
best existing package management system  was chosen as  a starting point.   The
biggest requirement was the  capacity of the system to  adapt to the  nature of
the robotic software,  being available mostly in form  of source code  only (no
binary packages),  with unfrequent stable  releases.  robotpkg had thus to deal
mostly with  source code  and automate the  compilation of  the  packages.  The
system chosen  as a starting  point was \href{http://www.pkgsrc.org}{The NetBSD
Packages  Collection} (pkgsrc).  robotpkg  can be considered as  a fork of this
project and  it is still very similar  to pkgsrc in  many points, although some
simplifications were made in order to provide  a tool geared toward people that
are not computer scientists but roboticists.

Due to its  origins, robotpkg provides many packages  developed at LAAS.  It is
however not  limited to such  packages and contains, in  fact, quite some other
software useful to  roboticists.  Of  course, robotpkg  is  not meant to  be  a
general purpose  packaging system   (although  there  would  be   no  technical
restriction to this) and will never  contain widely available packages that can
be found  on  any modern  Unix  distribution. Yet, robotpkg currently  contains
roughly one hundred and fifty packages, including:

\begin{itemize}
   \item architecture/genom - The LAAS Generator of Robotic Components

   \item simulation/openhrp - The Open Architecture Humanoid Robotics
   Platform from AIST, Japan

   \item architecture/openrtm - The robotic distributed middleware from AIST, Japan

   \item middleware/yarp - The ``other'', yet famous, robot platform

   \item ...just to name a few.
\end{itemize}


\section{Why robotpkg?} % --------------------------------------------------

robotpkg provides the following key features:

\begin{itemize}

   \item Easy building of software  from  source as well   as the creation  and
   installation of binary packages. The source and latest patches are retrieved
   from a master download site, checksum verified, then built on your system.

   \item All  packages are installed in a  consistent directory tree, including
   binaries, libraries, man pages and other documentation.

   \item  Package dependencies, including  when performing package updates, are
   handled automatically.

   \item The installation prefix, acceptable  software licenses and  build-time
   options  for a large  number of packages  are all set  in  a simple, central
   configuration file.

   \item The  entire framework source  (not including the  package distribution
   files themselves) is freely available under a BSD license, so you may extend
   and adapt robotpkg to your needs, like robotpkg was adapted from pkgsrc.

\end{itemize}


One question often asked by people is ``why was robotpkg forked from pkgsrc
instead of integrating the packages into pkgsrc?''. This is indeed a very good
question and the following paragraphs try to answer it.

First,  robotpkg is  not meant  to be  a replacement  for the  system's package
management tool (it does not  superseeds pkgsrc, dpkg, macports etc.). The goal
is to package software that is not widely available on a platform, and which is
mostly  "lab  software" (generally  of  lesser  quality  than widely  available
software).    Those   packages   change   (a   lot)  more   often,   and   more
drastically. Thus, robotpkg is a little bit closer to a "development" tool than
pkgsrc.  Other  ``system  packages''  are  correctly handled  by  a  number  of
packaging tools, and there is no need for a new tool.

Currently, pkgsrc mixes both infrastructure and packages descriptions
themselves. For someone working on e.g. Linux, checking-out
the whole pkgsrc tree would be cumbersome: it would be redundant with the base
Linux package system, plus it would be difficult to isolate the specific
robotic packages from the rest (the rest usually being available in the base
system). robotpkg currently suffers from the same symptom: this may change in
the future if the need for several package repositories becomes blatant.

robotpkg provides a number of features not available in pkgsrc (and probably
not really useful to pkgsrc either). The most important feature is to be able
to detect "system packages", that are considered as "external software not in
robotpkg but usually available on a unix system". pkgsrc has a similar system
but much more limited -- to a few base packages only. This is so because pkgsrc
is a full-fledged package system. Thus, it aims at being self contained, while
robotpkg does not.

Finally, there are a number of additions/changes to the pkgsrc infrastructure
that correspond to legitimate users requests and the specifc workflow in which
robotpkg is used. For instance, robotpkg provides the possibility to generate
an archive of a package from a specific tag in a source repository ``on the
fly'' or just bypass the archive generation and work directly from the source
repository to install the software. This later workflow is not encouraged, but
it is convenient to quickly test a -current version of some software to see if
it causes any problem. Those features could be ported back to pkgsrc if the
pkgsrc team would find them useful. In the meantime, robotpkg provides a
good testbed for them.

Still, robotpkg directly uses many of the pkgsrc tools unchanged and the binary
packages are fully compatible.


\section{Supported platforms} % --------------------------------------------

robotpkg consists of  a   source distribution. After retrieving    the required
source, you can be up and running with robotpkg in just minutes!

robotpkg  does not have much requirements  by itself and it  can work on a wide
variety of systems  as  long as they   provide a  GNU-make utility, a   working
C-compiler and a small, reasonably standard subset  of Unix commands (like sed,
awk, find,  grep ...).  However, individual packages  might have their specific
requirements.  The   following platforms  have been  reported  to  be supported
reasonably well:

\begin{center}\begin{tabular}{|c|c|}
\hline
Platform & Version
\doublehline
Fedora & 5 -- 13\\
Ubuntu & 7.10 -- 9.10\\
Debian & 5.03\\
CentOS & 5\\
NetBSD & 4 -- 5\\
Darwin & Partial support - infrastructure works, individual packages may not\\
\hline
\end{tabular}\end{center}


\section{Overview} % -------------------------------------------------------

This document is divided  into three parts.  \xref{chapter:user}{The first one}
describes how  one  can  use  one of   the  packages  in the  Robotics  Package
Collection, either  by installing a precompiled binary  package, or by building
one's own  copy  using  robotpkg.   \xref{chapter:developer}{The  second  part}
explains how  to prepare a package so  it can be  easily  built by  other users
without     knowing     about     the     package's    building        details.
\xref{chapter:internal}{The   third part} is  intended for  those   who want to
understand how robotpkg is implemented.


\section{Terminology} % ----------------------------------------------------

Here is a description of all the terminology used within this document.

\begin{description}
   \item[Package] A set of files and building instructions that describe what's
   necessary to build a certain piece  of software using robotpkg. Packages are
   traditionally stored under {\tt /opt/robotpkg}.

   \item[robotpkg]  This is  the The Robotics   Package Collection.  It handles
   building (compiling), installing, and removing of packages.

   \item[Distfile] This  term describes the file  or files that are provided by
   the author of the piece of software to distribute  his work. All the changes
   necessary to  build are reflected  in the corresponding package. Usually the
   distfile is in  the form of a  compressed  tar-archive, but other  types are
   possible,     too.    Distfiles   are      usually   stored    below    {\tt
   /opt/robotpkg/distfiles}.

   \item[Precompiled/binary package] A set of binaries built with robotpkg from
   a distfile  and stuffed together in a  single {\tt .tgz} file   so it can be
   installed  on machines of the same  machine architecture without the need to
   recompile. Packages are usually generated in {\tt /opt/robotpkg/packages}.

   Sometimes, this is  referred to by the  term ``package''  too, especially in
   the context of precompiled packages.

   \item[Program]  The  piece  of  software to  be  installed  which  will   be
   constructed from all the files in the distfile by the actions defined in the
   corresponding package.

\end{description}


\section{Roles involved in robotpkg} % -------------------------------------

\begin{description}
   \item[robotpkg users] The  robotpkg users  are people  who  use the packages
   provided by robotpkg.  Typically they are student  working  in robotics. The
   usage  of the software  that is {\em inside} the  packages is not covered by
   the robotpkg guide.

   There are two  kinds of robotpkg users:  Some only want to install pre-built
   binary packages.  Others build the robotpkg packages from source, either for
   installing them  directly or for building binary   packages themselves.  For
   robotpkg users, \xref{chapter:user}{Part~\ref{chapter:user}}  should provide
   all necessary documentation.

   \item[package  maintainers]   A   package maintainer  creates  packages   as
   described in \xref{chapter:developer}{Part~\ref{chapter:developer}}.

   \item[infrastructure  developers]  These people are    involved in all those
   files that live  in the {\tt mk/} directory   and below.  Only  these people
   should             need          to               read               through
   \xref{chapter:internal}{Part~\ref{chapter:internal}}, though others might be
   curious, too.

\end{description}


\section{Typography} % -----------------------------------------------------

When giving examples for  commands,  shell prompts  are  used  to show if   the
command  should/can be issued  as  root, or if  ``normal''  user privileges are
sufficient. We use  a {\tt \#}  for  root's shell  prompt, and  a {\tt \%}  for
users' shell prompt, assuming they use the C-shell or tcsh.


\chapter{The robotpkg user's guide}
\label{chapter:user}
% $LAAS: getting.tex 2009/01/11 11:43:51 tho $
%
% Copyright (c) 2009 LAAS/CNRS
% All rights reserved.
%
% Permission to use, copy, modify, and distribute this software for any purpose
% with or without   fee is hereby granted, provided   that the above  copyright
% notice and this permission notice appear in all copies.
%
% THE SOFTWARE IS PROVIDED "AS IS" AND THE AUTHOR DISCLAIMS ALL WARRANTIES WITH
% REGARD TO THIS  SOFTWARE INCLUDING ALL  IMPLIED WARRANTIES OF MERCHANTABILITY
% AND FITNESS. IN NO EVENT SHALL THE AUTHOR  BE LIABLE FOR ANY SPECIAL, DIRECT,
% INDIRECT, OR CONSEQUENTIAL DAMAGES OR  ANY DAMAGES WHATSOEVER RESULTING  FROM
% LOSS OF USE, DATA OR PROFITS, WHETHER IN AN ACTION OF CONTRACT, NEGLIGENCE OR
% OTHER TORTIOUS ACTION,   ARISING OUT OF OR IN    CONNECTION WITH THE USE   OR
% PERFORMANCE OF THIS SOFTWARE.
%
%                                             Anthony Mallet on Sat Jan 10 2009
%

\section{Where to get robotpkg and how to keep it up-to-date} % ------------
\label{section:getting}

Before you download and extract the files, you need to decide where you want to
extract  them.  robotpkg is  usually installed  in  {\tt /opt/openrobots},  but
creating this directory will probably   require administration privileges.   If
you don't have such privileges, you are free to  install the sources and binary
packages wherever you want in your filesystem, provided  that the pathname does
not contain white-space or other  characters that are interpreted specially  by
the shell and some other programs.  A safe bet is to  use only letters, digits,
underscores  and dashes. The rest of  this  document will  assume  that you are
using {\tt  /opt/openrobots}.  You should adapt  this path to whatever location
you choosed.


\subsection{Getting robotpkg for the first time} % -------------------------

robotpkg  will {\em never} require administration  privileges by itself.  So we
recommend that you set up the {\tt  /opt/openrobots} directory with read, write
and execute permissions for your regular user  name and then  only work as this
user afterwards. If something ever goes really  wrong, you might thank yourself
later that you did so\ldots This  can be done with  the following commands in a
shell:

\begin{verbatim}
% sudo mkdir -p /opt/openrobots
% sudo chown `id -u` /opt/openrobots
\end{verbatim}

At  the    moment,  robotpkg   is      only   distributed    {\em  via}     the
\href{http://git-scm.com/}{\tt git}  software content  management  system. {\tt
git} will probably be available on your system but if you don't have it readily
installed   or if  you  are   unsure  about  it,   contact your  local   system
administrator.

There are two download methods: the anonymous one and the authenticated
one. The two methods are described here.


\subsubsection{The anonymous download}

Anonymous  download is perfect  if  you don't intend  to  work on  the robotpkg
infrastructure itself, nor commit any changes or packages additions back to the
robotpkg main repository.  This is the recommended way  to go: it will fit most
users' usage while still leaving the possibility to send feedback via patches.

As your regular user, simply run in a shell:

\begin{verbatim}
% cd /opt/openrobots
% git clone http://softs.laas.fr/git/robots/robotpkg.git
\end{verbatim}


\subsubsection{The authenticated download}

Authenticated download requires a valid login  on the main robotpkg repository,
and  will give you  full commit access to this   repository. Assuming your user
name is ``{\tt user}'', run the following:

\begin{verbatim}
% cd /opt/openrobots
% git clone ssh://user@softs.laas.fr/git/robots/robotpkg
\end{verbatim}


\subsection{Keeping robotpkg up-to-date} % ---------------------------------

robotpkg is  a living  thing: updates  to the packages  are made  perdiodicaly,
problems are fixed,  enhancements are developed\ldots  In order to get the most
recent packages descriptions, you should keep your robotpkg copy up-to-date by
regularly running {\tt git pull}:

\begin{verbatim}
% cd /opt/openrobots/robotpkg
% git pull
\end{verbatim}

When you update robotpkg, the git program will only  touch those files that are
registered in the git repository. That means that any packages that you created
on your own will stay unmodified. If you change files that  are managed by git,
later updates will try to merge your changes with  those that have been done by
others. See the git-pull manual for details.

If you want  to be informed  of package additions  and other  updates, a public
mailing    list  is   available    for   your    reading   pleasure.  Go     to
\url{https://sympa.laas.fr/sympa/info/robotpkg}    for   more  information  and
subscription.

% $LAAS: bootstrapping.tex 2010/06/23 15:04:49 mallet $
%
% Copyright (c) 2009-2010 LAAS/CNRS
% All rights reserved.
%
% Permission to use, copy, modify, and distribute this software for any purpose
% with or without   fee is hereby granted, provided   that the above  copyright
% notice and this permission notice appear in all copies.
%
% THE SOFTWARE IS PROVIDED "AS IS" AND THE AUTHOR DISCLAIMS ALL WARRANTIES WITH
% REGARD TO THIS  SOFTWARE INCLUDING ALL  IMPLIED WARRANTIES OF MERCHANTABILITY
% AND FITNESS. IN NO EVENT SHALL THE AUTHOR  BE LIABLE FOR ANY SPECIAL, DIRECT,
% INDIRECT, OR CONSEQUENTIAL DAMAGES OR  ANY DAMAGES WHATSOEVER RESULTING  FROM
% LOSS OF USE, DATA OR PROFITS, WHETHER IN AN ACTION OF CONTRACT, NEGLIGENCE OR
% OTHER TORTIOUS ACTION,   ARISING OUT OF OR IN    CONNECTION WITH THE USE   OR
% PERFORMANCE OF THIS SOFTWARE.
%
%                                             Anthony Mallet on Sun Jan 11 2009
%

\section{Bootstrapping robotpkg} % -----------------------------------------
\label{section:bootstrapping}

Once you have  downloaded the robotpkg sources  or the binary bootstrap kit  as
described  in  \xref{section:getting}{Section~\ref{section:getting}}, a minimal
set  of  the administrative package management  utilities  must be installed on
your system  before you  can  use robotpkg.   This  is  called the  ``bootstrap
phase'' and  should   be done only   once,  the very  first  time you  download
robotpkg.


\subsection{Bootstrapping via the binary kit} % ----------------------------

At the moment, the binary bootstrap kit is not available. Please bootstrap {\tt
robotpkg} as described in the next section.


\subsection{Bootstrapping from source} % -----------------------------------

You will  need a working C compiler  and the GNU-make   utility version 3.81 or
later.    If you have  extracted  the  robotpkg  archive  into  the standard {\tt
/opt/openrobots/robotpkg} location, installing the   bootstrap kit from  source
should then be as simple as:

\begin{verbatim}
% cd /opt/openrobots/robotpkg/bootstrap
% ./bootstrap
\end{verbatim}

This will  install various utilities   into {\tt /opt/openrobots/sbin}.

Should you prefer another installation path, you could use the {\tt -{}-prefix}
option to  change the default  installation prefix.  For  instance, configuring
robotpkg  to  install programs  into  the  openrobots  directory in  your  home
directory can be done like this:

\begin{verbatim}
% cd robotpkg/bootstrap
% ./bootstrap --prefix=${HOME}/openrobots
\end{verbatim}

{\bf  After the  bootstrap script  has run,  a message  indicating  the success
should be  displayed.  If  you choosed a  non-standard installation  path, read
this message carefuly}, as it contains  instructions that you have to follow in
order  to  setup your  shell  environment  correctly.   These instructions  are
described in the next section.


\subsubsection{Configuring your environment} % -----------------------------

If  you configured robotpkg,   during the bootstrap  phase,  to install to some
other location   than {\tt /opt/openrobots}, you  have   to setup manually your
shell environment so that it contains a few  variables holding the installation
path.  Assuming  you invoked bootstrap with {\tt --prefix=/path/to/openrobots},
you have two options that are compatible with each other:

\begin{itemize}
   \item Add  the directory {\tt  /path/to/openrobots/sbin}  to your {\tt PATH}
   variable. robotpkg will    then be able  to find    its administrative tools
   automatically and from that recover other configuration information. This is
   the preferred method.

   \item Create the environment variable {\tt ROBOTPKG\_BASE} and set its value
   to {\tt /path/to/openrobots}.  robotpkg will  look for this variable  first,
   so it takes precedence over the  first method.  This is  the method you have
   to choose  if  you have  configured  several instances  of robotpkg  in your
   system. This is ony useful in some circumstances and is not normally needed.
\end{itemize}

If  you  don't know  how  to setup   environment variables  permanently in your
system,  please  refer  to  your shell's  manual  or contact  your local system
administrator.


\subsubsection{The bootstrap script usage} % -------------------------------

The {\tt bootstrap} script will by default install the package administrative
tools in {\tt /opt/openrobots/sbin}, use {\tt gcc} as the C compiler and {\tt
make} as the GNU-make program. This behaviour can be fine-tuned by using the
following options:

\begin{description}
   \item[\tt   -{}-prefix <path>]   will   select the  prefix  location where
   programs will be installed in.

   \item[\tt -{}-sysconfdir <path>] defaults to {\tt <prefix>/etc}. This is the
   path to the robotpkg configuration file.  Other packages configuration files
   (if any) will also be stored in this directory.

   \item[\tt -{}-pkgdbdir  <path>] defaults to {\tt  <prefix>/var/db/pkg}. This
   is the path  to the package database  directory  where robotpkg will  do its
   internal bookkeeping.

   \item[\tt -{}-compiler <program>] defaults to {\tt gcc}.  Use this option if
   you want to use a different C compiler.

   \item[\tt -{}-make <program>] defaults to {\tt make}. Use this option if you
   want to use a different make program. This program should be compatible with
   GNU-make.

   \item[\tt -{}-help]  displays  the {\tt bootstrap} usage.  The comprehensive
   list of recognized options will be displayed.
\end{description}

%%
% Copyright (c) 2009-2010,2013 LAAS/CNRS
% All rights reserved.
%
% Permission to use, copy, modify, and distribute this software for any purpose
% with or without   fee is hereby granted, provided   that the above  copyright
% notice and this permission notice appear in all copies.
%
% THE SOFTWARE IS PROVIDED "AS IS" AND THE AUTHOR DISCLAIMS ALL WARRANTIES WITH
% REGARD TO THIS  SOFTWARE INCLUDING ALL  IMPLIED WARRANTIES OF MERCHANTABILITY
% AND FITNESS. IN NO EVENT SHALL THE AUTHOR  BE LIABLE FOR ANY SPECIAL, DIRECT,
% INDIRECT, OR CONSEQUENTIAL DAMAGES OR  ANY DAMAGES WHATSOEVER RESULTING  FROM
% LOSS OF USE, DATA OR PROFITS, WHETHER IN AN ACTION OF CONTRACT, NEGLIGENCE OR
% OTHER TORTIOUS ACTION,   ARISING OUT OF OR IN    CONNECTION WITH THE USE   OR
% PERFORMANCE OF THIS SOFTWARE.
%
%                                             Anthony Mallet on Sun Jan 11 2009
%

\section{Using robotpkg} % -------------------------------------------------

After obtaining \robotpkg, the  {\tt robotpkg} directory now  contains a set of
packages, organized   into  categories.  You can   browse  the online  index of
packages, or run {\tt  make index} from the {\tt  robotpkg} directory to  build
local {\tt index.html}  files for all  packages, viewable with any web  browser
such as {\tt lynx} or {\tt firefox}.

\robotpkg is  essentially based on the  {\tt make(1)} program.  All actions are
triggered by invoking {\tt make} with the proper target. The following sections
document          the           most          useful          ones          and
\xref{section:using:targets}{section~\ref{section:using:targets}} recaps a more
comprehensive list.


\subsection{Building packages from source} % -------------------------------

The  first step for  building  a  package  is  downloading the {\em  distfiles}
(i.e. the unmodified  source). If they have not  yet been downloaded, \robotpkg
will  fetch them automatically  and place them  in the {\tt robotpkg/distfiles}
directory.

Once the software  has  been downloaded,  any  patches will be applied  and the
package will  be compiled for  you.  This may  take some time depending on your
computer, and how many other packages the software depends on and their compile
time.

For  example,  type the following  commands  at the shell   prompt to build the
robotpkg documentation package:

\begin{verbatim}
% cd /opt/openrobots/robotpkg
% cd doc/robotpkg
% make
\end{verbatim}

The next  stage is  to  actually install the newly   compiled package onto your
system. While you   are still in  the directory  for whatever package  you  are
installing, you can do this by entering:

\begin{verbatim}
% make install
\end{verbatim}

Installing the package on your system does  not require you  to be root (except
for a few specific  packages). However, if   you bootstraped with a  prefix for
which   you  don't   have  writing   permissions,    \robotpkg   has a     {\rm
just-in-time-sudo}  feature,  which allows you to  become  {\tt  root}  for the
actual installation step.

That's it, the software should now be installed   under  the prefix  of the
packages tree --- {\tt /opt/openrobots} by default --- and setup for use.

You can now enter:

\begin{verbatim}
% make clean
\end{verbatim}

to remove the compiled files in the work  directory, as you shouldn't need them
any more. If  other packages were also  added to your system (dependencies)  to
allow your program to compile, you can also tidy these up with the command:

\begin{verbatim}
% make clean-depends
\end{verbatim}

Since  the three tasks of building,  installing and  cleaning correspond to the
typical usage of \robotpkg, a helper target doing all these tasks exists and is
called {\tt update}. Thus,  to intall a package  with a single command, you can
simply run:

\begin{verbatim}
% make update
\end{verbatim}

In addition, {\tt  make update} will  also recompile all the installed packages
that were depending on the package that you are updating. This can be quite
time consuming if you are updating a low-level package. Also, note that all
packages that depend on the package you are updating will be deinstalled
first and unavailable in your system until all packages are recompiled and
reinstalled.

%
%    <para>Some packages look in &mk.conf; to
%    alter some configuration options at build time.  Have a look at
%    <filename>pkgsrc/mk/defaults/mk.conf</filename> to get an overview
%    of what will be set there by default.  Environment variables such
%    as <varname>LOCALBASE</varname> can be set in
%    &mk.conf; to save having to remember to
%    set them each time you want to use pkgsrc.</para>
%

Occasionally, people want to ``look under the covers'' to see  what is going on
when a  package  is building  or being  installed.  This may  be for  debugging
purposes, or  out  of simple curiosity. A  number  of utility values have  been
added to help with this.

\begin{enumerate}

\item If you invoke the {\tt make} command with {\tt PKG\_DEBUG\_LEVEL=1}, then
      a huge amount of information will be displayed. For example,

\begin{verbatim}
% make patch PKG_DEBUG_LEVEL=1
\end{verbatim}

      will show all the commands that are invoked, up to and including the
      ``patch'' stage. Using {\tt PKG\_DEBUG\_LEVEL=2} will give you even
      more details.

\item If you want to know the value of a certain {\tt make} definition, then
   the {\tt VARNAME} variable   should be used,  in  conjunction with the  {\tt
   show-var} target.  e.g.  to show the  expansion  of the  {\tt make} variable
   {\tt LOCALBASE}:

\begin{verbatim}
% make show-var VARNAME=LOCALBASE
\end{verbatim}

\end{enumerate}

%    <para>If you want to install a binary package that you've either
%    created yourself (see next section), that you put into
%    pkgsrc/packages manually or that is located on a remote FTP
%    server, you can use the "bin-install" target. This target will
%    install a binary package - if available - via &man.pkg.add.1;,
%    else do a <command>make package</command>.  The list of remote FTP
%    sites searched is kept in the variable
%    <varname>BINPKG_SITES</varname>, which defaults to
%    ftp.NetBSD.org. Any flags that should be added to &man.pkg.add.1;
%    can be put into <varname>BIN_INSTALL_FLAGS</varname>.  See
%    <filename>pkgsrc/mk/defaults/mk.conf</filename> for more
%    details.</para>


%    <para>A final word of warning: If you set up a system that has a
%    non-standard setting for <varname>LOCALBASE</varname>, be sure to
%    set that before any packages are installed, as you cannot use
%    several directories for the same purpose. Doing so will result in
%    pkgsrc not being able to properly detect your installed packages,
%    and fail miserably. Note also that precompiled binary packages are
%    usually built with the default <varname>LOCALBASE</varname> of
%    <filename>/usr/pkg</filename>, and that you should
%    <emphasis>not</emphasis> install any if you use a non-standard
%    <varname>LOCALBASE</varname>.</para>


\subsection{Building packages from a repository checkout} % ----------------
\label{section:using:checkout}

Before building a  package, \robotpkg fetches the sources  from the official(s)
download  location(s),  as  instructed  by the  {\tt  MASTER\_SITES}  variable.
This is the standard and expected behaviour when you work with stable packages.

Occasionally, though,  it is useful to fetch  a snapshot of the  sources from a
development repository. For instance, one  might want to quickly test a release
candidate of a  package, or fix a simple  bug and create a patch  from the fix.
Whenever a package defines  the {\tt MASTER\_REPOSITORY} variable, \robotpkg is
able to temporarily  work with the repository defined in  this variable. At the
moment, {\tt cvs}, {\tt svn} and {\tt git} repositories are supported.

To enable this feature for a given package,  you have to first instruct
\robotpkg to work from a '{\tt checkout}' (instead of the stable releases) by
doing '{\tt make checkout}' in the package directory. For instance:

\begin{verbatim}
% cd robotpkg/foo/bar
% make checkout
\end{verbatim}

This sets  a permanent flag in the  {\em working} directory of  the package and
the {\em checkout}  configuration option will be retained  until the next '{\tt
make clean}'. After a '{\tt make  clean}', the configuration option is set back
to its default and \robotpkg will  work again with stable releases. This option
is set on a {\em per} package  basis only: configuring one package to work with
checkouts does not affect the behaviour of other packages.

After a '{\tt make checkout}' (and until a '{\tt make clean}'), the package has
a regular  checkout in its {\em  working} subdirectory.  You  can thus manually
edit, commit, switch branches, etc.  in  the package sources, like in any other
repository, by  first {\tt  cd}ing into the  working directory, then  using the
usual repository commands ({\tt cvs}, {\tt svn} or {\tt git}).

Of  course, the  individual  \robotpkg  targets are  still  available from  the
package  entry in  the robotpkg  hierarchy.  You  can for  instance  {\tt 'make
patch'}, {\tt 'configure'}, {\tt 'build'}, {\tt 'install'} or {\tt 'update'} as
usual. Note that  \robotpkg is not exactly stateless, and  this is most visible
when  working with  checkouts:  for  instance, after  a  successful {\tt  'make
build'}, you  have to do {\tt 'make  rebuild'} to force rebuilding  if you have
modified  the  sources.   The  same  holds  for  {\tt   'configure'}  (do  {\tt
'reconfigure'})  or {\tt  'install'} (do  {'reinstall'}, but  since  you cannot
install a package  twice, you normally have to use {\tt  'make replace'} in the
particular case of reinstalling a package).

The  {\tt  'clean'}  target  is  special,  in  that  it  removes  the  checkout
configuration  option and  all checkouted  sources, including  locally modified
sources. In order to prevent accidental deletion of precious files, you have to
confirm the cleanign with {\tt 'clean confirm'}, as in:

\begin{verbatim}
% make clean confirm
\end{verbatim}

A  final  remark:  we {\em  STRONGLY  DISCOURAGE}  the  use  of robotpkg  as  a
development tool  (i.e. using the {\tt  'checkout'} feature on  a {\em regular}
basis), for at least two reasons:

\begin{itemize}
   \item \robotpkg  is not designed  for this: it  will not really help  you in
   your  daily   development  work,   compared  to  the   manual  configuration
   installation of the software. It will sometimes create even more trouble, by
   ensuring  that all  the software  depending  on the  checkouted software  is
   up-to-date, which is not necessarily something you want to do every time you
   compile.

   \item  A checkout  breaks the  notion  of 'release'  and you  loose all  the
   benefits from working with packages.  In particular, you have no clear state
   of what is installed: you cannot easily reproduce the situation of time T at
   time T+n and don't know precisely  who requires which version of what. It is
   much  more  efficient and  robust  to release  frequently  a  software in  a
   development phase, than using a {\em rolling release} approach.
\end{itemize}

In our opinion, the {\tt 'checkout'}  target use should be limited to testing a
release candidate or  quickly fix a bug  and create a patch from  the fix, that
you commit upstream and put in the patches/ directory until the next release.


\subsection{Installing binary packages} % ----------------------------------

At the moment, installing binary packages is not documented.

\subsection{Removing packages} % -------------------------------------------

To deinstall a package, it does not matter whether it was installed from source
code or  from a  binary package.  The  {\tt robotpkg\_delete} command  does not
know it  anyway.  To delete a  package, you can just  run {\tt robotpkg\_delete
<package-name>}.  The package name can be given with or without version number.
Wildcards can  also be used  to deinstall a  set of packages, for  example {\tt
*genom*} all  packages whose  name contain  the word {\tt  genom}.  Be  sure to
include them  in quotes,  so that the  shell does  not expand them  before {\tt
robotpkg\_delete} sees them.

The {\tt -r} option is very powerful: it  removes all the packages that require
the package in question and then removes the package itself. For example:

\begin{verbatim}
% robotpkg_delete -r genom
\end{verbatim}

will remove genom and all the packages that used it; this allows
upgrading the {\tt genom} package.

\subsection{Getting information about installed packages} % ----------------

The {\tt  robotpkg\_info} shows information about installed  packages or binary
package files.


\subsection{Other administrative functions} % ------------------------------

The  {\tt robotpkg\_admin}  executes  various administrative  functions on  the
package system.

\subsection{Available {\tt make} targets} % --------------------------------
\label{section:using:targets}

The following targets are available in a package directory. They can be invoked
by   running  {\tt   make  <target>}   after   {\tt  cd}ing   into  some   {\tt
robotpkg/category/package}.

\subsubsection{Source manipulation}
\begin{description}
   \item[{\tt fetch}] Download the {\tt\$\{DISTFILES\}}.

   \item[{\tt extract}] Extract the contents of {\tt\$\{DISTFILES\}} into the
   work directory {\tt\$\{WRKDIR\}}.

   \item[{\tt patch}] Apply local patches available in {\tt\$\{PATCHDIR\}}
   (usually the {\tt patches} directory in the package).

   \item[{\tt checkout}] Extract the sources in {\tt\$\{WRKDIR\}} from
   {\tt\$\{MASTER\_REPOSITORY\}} instead of {\tt\$\{MASTER\_SITES\}}. This can
   be used to fetch a not yet released version instead of the latest
   release. This is mutually exclusive with the {\tt fetch} and {\tt extract}
   targets. See
   \xref{section:using:checkout}{section~\ref{section:using:checkout}} for
   details.

   \item[{\tt configure}] Perform the necessary actions to configure the
   sources. This may for instance involve running {\tt configure} or {\tt
   cmake}. If no configuration is required, this step simply does nothing.

   \item[{\tt build}] Or  just {\tt make}, the default  target. It compiles the
   package locally in {\tt\$\{WRKDIR\}}.

   \item[{\tt install}] Install the package into {\tt\$\{PREFIX\}}. The package
   is then available to the rest of the system. If an older version of the
   package is installed and required by other packages, this target requires
   confirmation. Otherwise, any older version of the package is first
   deinstalled.

   \item[{\tt replace}] Same as {\tt install}, but does not remove packages
   that depend on the replaced package. This saves some time, since already
   installed package are not touched, but if the replaced package is
   incompatible with the older version, you will run into trouble. Use with
   care and when you know what you are doing.

   \item[{\tt clean}] Tidy the work directory and removes {\tt\$\{WRKDIR\}}. If
   the package was extracted using {\tt checkout}, this target requires
   confirmation as it may delete locally modified files that will be lost.

   \item[{\tt update}] This is a shortcut target for {\tt fetch}, {\tt
   extract}, {\tt configure}, {\tt build}, {\tt install} and {\tt clean}. If
   the package is already installed and up-to-date, the target asks for
   confirmation.

\end{description}

\subsubsection{Introspection}
\begin{description}
   \item[{\tt show-options}] Display the list of available alternatives (see
   \xref{section:configuring:alternatives}%
   {section~\ref{section:configuring:alternatives}})
   and build options (see
   \xref{section:configuring:build_options}%
   {section~\ref{section:configuring:build_options}}).

   \item[{\tt show-depends}] Recursively display all the required dependencies
   of a package. The results are splitted between system and \robotpkg
   dependencies, and missing dependencies are indicated.

   \item[{\tt show-var}] Display the contents of a variable. This must be
   invoked as {\tt make show-var VARNAME=foo}, where {\tt foo} is the name of
   the variable to be displayed.
\end{description}

\subsubsection{Package sets}

\begin{description}
   \item[{\tt fetch-depends}, {\tt replace-depends}, {\tt update-depends}, {\tt
   clean-depends}]
   This runs the same action as {\tt fetch}, {\tt replace}, {\tt update} or
   {\tt clean} (respectively), but on all dependencies of the package,
   including the package itself. Useful to update a meta-packages, for instance.

   \item[{\tt fetch-<set>}, {\tt replace-<set>}, {\tt update-<set>}, {\tt
   clean-<set>}]
   This runs the same action as {\tt fetch}, {\tt replace}, {\tt update} or
   {\tt clean} (respectively), but on all members of the package set named {\tt
   <set>}. See
   \xref{section:configuring:sets}{section~\ref{section:configuring:sets}} 
   for an explanation of package sets and how to configure them.

\end{description}


\chapter{The robotpkg developer's guide}
\label{chapter:developer}

\chapter{The robotpkg infrastructure internals}
\label{chapter:internal}
%\chapter{test}
% $LAAS: internal.tex 2009/01/16 23:58:51 tho $
%
% Copyright (c) 2009 LAAS/CNRS
% All rights reserved.
%
% Permission to use, copy, modify, and distribute this software for any purpose
% with or without   fee is hereby granted, provided   that the above  copyright
% notice and this permission notice appear in all copies.
%
% THE SOFTWARE IS PROVIDED "AS IS" AND THE AUTHOR DISCLAIMS ALL WARRANTIES WITH
% REGARD TO THIS  SOFTWARE INCLUDING ALL  IMPLIED WARRANTIES OF MERCHANTABILITY
% AND FITNESS. IN NO EVENT SHALL THE AUTHOR  BE LIABLE FOR ANY SPECIAL, DIRECT,
% INDIRECT, OR CONSEQUENTIAL DAMAGES OR  ANY DAMAGES WHATSOEVER RESULTING  FROM
% LOSS OF USE, DATA OR PROFITS, WHETHER IN AN ACTION OF CONTRACT, NEGLIGENCE OR
% OTHER TORTIOUS ACTION,   ARISING OUT OF OR IN    CONNECTION WITH THE USE   OR
% PERFORMANCE OF THIS SOFTWARE.
%
%                                             Florent Lamiraux on Sat Jan 24 2009
%

\subsection*{Using autotools}

The following variables and inclusion are useful for packages managed by {\tt autotools}. They should be defined in the installation Makefile of the package.

\paragraph{${\tt GNU\_CONFIGURE}$} should be set to {\tt yes} to invoke configure for configuration. 

\subsection*{Using cmake}

The following variables and inclusion are useful for packages managed by {\tt cmake}. They should be defined in the installation Makefile of the package.

\paragraph{${\tt USE\_CMAKE}$} should be set to {\tt yes} to invoke cmake for configuration. Moreover, the following inclusion should added after the definition of the above variable in the Makefile (assuming the Makefile is two directories below {\tt robotpkg} root directory):\\
{\tt include ../../mk/sysdep/cmake.mk}



\chapter{Trouble shooting}
\label{chapter:troubleshooting}
% $LAAS: troubleshooting.tex 2009/01/30 23:58:51 tho $
%
% Copyright (c) 2009 LAAS/CNRS
% All rights reserved.
%
% Permission to use, copy, modify, and distribute this software for any purpose
% with or without   fee is hereby granted, provided   that the above  copyright
% notice and this permission notice appear in all copies.
%
% THE SOFTWARE IS PROVIDED "AS IS" AND THE AUTHOR DISCLAIMS ALL WARRANTIES WITH
% REGARD TO THIS  SOFTWARE INCLUDING ALL  IMPLIED WARRANTIES OF MERCHANTABILITY
% AND FITNESS. IN NO EVENT SHALL THE AUTHOR  BE LIABLE FOR ANY SPECIAL, DIRECT,
% INDIRECT, OR CONSEQUENTIAL DAMAGES OR  ANY DAMAGES WHATSOEVER RESULTING  FROM
% LOSS OF USE, DATA OR PROFITS, WHETHER IN AN ACTION OF CONTRACT, NEGLIGENCE OR
% OTHER TORTIOUS ACTION,   ARISING OUT OF OR IN    CONNECTION WITH THE USE   OR
% PERFORMANCE OF THIS SOFTWARE.
%
%                                             Florent Lamiraux on Sat Jan 30 2009
%

In this chapter, we list some common failure cases and explain how to solve them.

\subsubsection*{Dependency not found at installation.}
After typing {\tt make install} in a directory, installation fails at the configuration step of a package because {\tt pkg-config} cannot find a dependency although this dependency is encoded in {\tt robotpkg} Makefiles.

\paragraph{Possible reason.} The missing package has been deleted by {\tt robotpkg\_delete} but {\tt make clean} has not been run in the corresponding directory. As a result, a new {\tt make install} command in this directory has no effect since everything seems up to date. 

\paragraph{Solving the problem.} Before installing or reinstalling packages, it is wise to run {\tt make clean} in {\tt robotpkg} root directory.

\end{document} % -----------------------------------------------------------
