% $LAAS: robotpkg.tex 2010/11/17 18:22:18 mallet $
%
% Copyright (c) 2009-2010 LAAS/CNRS
% All rights reserved.
%
% Permission to use, copy, modify, and distribute this software for any purpose
% with or without   fee is hereby granted, provided   that the above  copyright
% notice and this permission notice appear in all copies.
%
% THE SOFTWARE IS PROVIDED "AS IS" AND THE AUTHOR DISCLAIMS ALL WARRANTIES WITH
% REGARD TO THIS  SOFTWARE INCLUDING ALL  IMPLIED WARRANTIES OF MERCHANTABILITY
% AND FITNESS. IN NO EVENT SHALL THE AUTHOR  BE LIABLE FOR ANY SPECIAL, DIRECT,
% INDIRECT, OR CONSEQUENTIAL DAMAGES OR  ANY DAMAGES WHATSOEVER RESULTING  FROM
% LOSS OF USE, DATA OR PROFITS, WHETHER IN AN ACTION OF CONTRACT, NEGLIGENCE OR
% OTHER TORTIOUS ACTION,   ARISING OUT OF OR IN    CONNECTION WITH THE USE   OR
% PERFORMANCE OF THIS SOFTWARE.
%
%                                             Anthony Mallet on Sat Jan 10 2009
%

\documentclass[a4paper,11pt]{book}
\newif\iftth\iftth
   % $LAAS: robotpkg.tex 2009/01/11 17:52:38 tho $
%
% Copyright (c) 2009 LAAS/CNRS
% All rights reserved.
%
% Permission to use, copy, modify, and distribute this software for any purpose
% with or without   fee is hereby granted, provided   that the above  copyright
% notice and this permission notice appear in all copies.
%
% THE SOFTWARE IS PROVIDED "AS IS" AND THE AUTHOR DISCLAIMS ALL WARRANTIES WITH
% REGARD TO THIS  SOFTWARE INCLUDING ALL  IMPLIED WARRANTIES OF MERCHANTABILITY
% AND FITNESS. IN NO EVENT SHALL THE AUTHOR  BE LIABLE FOR ANY SPECIAL, DIRECT,
% INDIRECT, OR CONSEQUENTIAL DAMAGES OR  ANY DAMAGES WHATSOEVER RESULTING  FROM
% LOSS OF USE, DATA OR PROFITS, WHETHER IN AN ACTION OF CONTRACT, NEGLIGENCE OR
% OTHER TORTIOUS ACTION,   ARISING OUT OF OR IN    CONNECTION WITH THE USE   OR
% PERFORMANCE OF THIS SOFTWARE.
%
%                                             Anthony Mallet on Sat Jan 10 2009
%

\documentclass[11pt]{book}
\usepackage{ifpdf}
\usepackage{hyperref}
\usepackage[T1]{fontenc}

\ifpdf\pdfinfo{
   /Author (Anthony Mallet)
   /Title  (A guide to robotpkg)
   /Subject (robotpkg)
   /Keywords (robotic software package)
}\fi

%%tth: \def\hyperref[#1]#2{\special{html:<a href="\##1">}#2\special{html:</a>}}

\title{A guide to robotpkg}
\author{Anthony Mallet --- {\tt anthony.mallet@laas.fr}}

\begin{document} % ---------------------------------------------------------

\maketitle
\tableofcontents

\chapter{Introduction}
\label{chapter:introduction}
\input{introduction}

\chapter{The robotpkg user's guide}
\label{chapter:user}
% $LAAS: getting.tex 2009/02/02 00:14:34 tho $
%
% Copyright (c) 2009 LAAS/CNRS
% All rights reserved.
%
% Permission to use, copy, modify, and distribute this software for any purpose
% with or without   fee is hereby granted, provided   that the above  copyright
% notice and this permission notice appear in all copies.
%
% THE SOFTWARE IS PROVIDED "AS IS" AND THE AUTHOR DISCLAIMS ALL WARRANTIES WITH
% REGARD TO THIS  SOFTWARE INCLUDING ALL  IMPLIED WARRANTIES OF MERCHANTABILITY
% AND FITNESS. IN NO EVENT SHALL THE AUTHOR  BE LIABLE FOR ANY SPECIAL, DIRECT,
% INDIRECT, OR CONSEQUENTIAL DAMAGES OR  ANY DAMAGES WHATSOEVER RESULTING  FROM
% LOSS OF USE, DATA OR PROFITS, WHETHER IN AN ACTION OF CONTRACT, NEGLIGENCE OR
% OTHER TORTIOUS ACTION,   ARISING OUT OF OR IN    CONNECTION WITH THE USE   OR
% PERFORMANCE OF THIS SOFTWARE.
%
%                                             Anthony Mallet on Sat Jan 10 2009
%

Basically, there are two ways of using robotpkg.  The  first is to only install
the package tools  and to use binary packages  that someone  else has prepared.
The second way is  to install the programs from   source. Then you are able  to
build your  own packages, and  you can still use   binary packages from someone
else.


\section{Where to get robotpkg and how to keep it up-to-date} % ------------
\label{section:getting}

Before you download and extract the files, you need to decide where you want to
extract  them.  robotpkg is  usually installed  in  {\tt /opt/openrobots},  but
creating this directory will probably   require administration privileges.   If
you don't have such privileges, you are free to  install the sources and binary
packages wherever you want in your filesystem, provided  that the pathname does
not contain white-space or other  characters that are interpreted specially  by
the shell and some other programs.  A safe bet is to  use only letters, digits,
underscores  and dashes. The rest of  this  document will  assume  that you are
using {\tt  /opt/openrobots}.  You should adapt  this path to whatever location
you choosed.


\subsection{Getting robotpkg for the first time} % -------------------------

robotpkg  will {\em never} require administration  privileges by itself.  So we
recommend that you set up the {\tt  /opt/openrobots} directory with read, write
and execute permissions for your regular user  name and then  only work as this
user afterwards. If something ever goes really  wrong, you might thank yourself
later that you did so\ldots This  can be done with  the following commands in a
shell:

\begin{verbatim}
% sudo mkdir -p /opt/openrobots
% sudo chown `id -u` /opt/openrobots
\end{verbatim}

At  the    moment,  robotpkg   is      only   distributed    {\em  via}     the
\href{http://git-scm.com/}{\tt git}  software content  management  system. {\tt
git} will probably be available on your system but if you don't have it readily
installed   or if  you  are   unsure  about  it,   contact your  local   system
administrator.

There are two download methods: the anonymous one and the authenticated
one. The two methods are described here.


\subsubsection{Anonymous download}

Anonymous  download is perfect  if  you don't intend  to  work on  the robotpkg
infrastructure itself, nor commit any changes or packages additions back to the
robotpkg main repository.  This is the recommended way  to go: it will fit most
users' usage while still leaving the possibility to send feedback via patches.

As your regular user, simply run in a shell:

\begin{verbatim}
% cd /opt/openrobots
% git clone http://softs.laas.fr/git/robots/robotpkg.git
\end{verbatim}


\subsubsection{Authenticated download}

Authenticated download requires a valid login  on the main robotpkg repository,
and  will give you  full commit access to this   repository. Assuming your user
name is ``{\tt user}'', run the following:

\begin{verbatim}
% cd /opt/openrobots
% git clone ssh://user@softs.laas.fr/git/robots/robotpkg
\end{verbatim}


\subsection{Keeping robotpkg up-to-date} % ---------------------------------

robotpkg is  a living  thing: updates  to the packages  are made  perdiodicaly,
problems are fixed,  enhancements are developed\ldots  In order to get the most
recent packages descriptions, you should keep your robotpkg copy up-to-date by
regularly running {\tt git pull}:

\begin{verbatim}
% cd /opt/openrobots/robotpkg
% git pull
\end{verbatim}

When you update robotpkg, the git program will only  touch those files that are
registered in the git repository. That means that any packages that you created
on your own will stay unmodified. If you change files that  are managed by git,
later updates will try to merge your changes with  those that have been done by
others. See the git-pull manual for details.

If you want  to be informed  of package additions  and other  updates, a public
mailing    list  is   available    for   your    reading   pleasure.  Go     to
\url{https://sympa.laas.fr/sympa/info/robotpkg}    for   more  information  and
subscription.

\input{bootstrapping}
% $LAAS: using.tex 2010/08/19 11:54:35 mallet $
%
% Copyright (c) 2009-2010 LAAS/CNRS
% All rights reserved.
%
% Permission to use, copy, modify, and distribute this software for any purpose
% with or without   fee is hereby granted, provided   that the above  copyright
% notice and this permission notice appear in all copies.
%
% THE SOFTWARE IS PROVIDED "AS IS" AND THE AUTHOR DISCLAIMS ALL WARRANTIES WITH
% REGARD TO THIS  SOFTWARE INCLUDING ALL  IMPLIED WARRANTIES OF MERCHANTABILITY
% AND FITNESS. IN NO EVENT SHALL THE AUTHOR  BE LIABLE FOR ANY SPECIAL, DIRECT,
% INDIRECT, OR CONSEQUENTIAL DAMAGES OR  ANY DAMAGES WHATSOEVER RESULTING  FROM
% LOSS OF USE, DATA OR PROFITS, WHETHER IN AN ACTION OF CONTRACT, NEGLIGENCE OR
% OTHER TORTIOUS ACTION,   ARISING OUT OF OR IN    CONNECTION WITH THE USE   OR
% PERFORMANCE OF THIS SOFTWARE.
%
%                                             Anthony Mallet on Sun Jan 11 2009
%

\section{Using robotpkg} % -------------------------------------------------

After obtaining \robotpkg, the  {\tt robotpkg} directory now  contains a set of
packages, organized   into  categories.  You can   browse  the online  index of
packages, or run {\tt  make index} from the {\tt  robotpkg} directory to  build
local {\tt index.html}  files for all  packages, viewable with any web  browser
such as {\tt lynx} or {\tt firefox}.


\subsection{Building packages from source} % -------------------------------

The  first step for  building  a  package  is  downloading the {\em  distfiles}
(i.e. the unmodified  source). If they have not  yet been downloaded, \robotpkg
will  fetch them automatically  and place them  in the {\tt robotpkg/distfiles}
directory.

Once the software  has  been downloaded,  any  patches will be applied  and the
package will  be compiled for  you.  This may  take some time depending on your
computer, and how many other packages the software depends on and their compile
time.

For  example,  type the following  commands  at the shell   prompt to build the
robotpkg documentation package:

\begin{verbatim}
% cd /opt/openrobots/robotpkg
% cd doc/robotpkg
% make
\end{verbatim}

The next  stage is  to  actually install the newly   compiled package onto your
system. While you   are still in  the directory  for whatever package  you  are
installing, you can do this by entering:

\begin{verbatim}
% make install
\end{verbatim}

Installing the package on your system does  not require you  to be root (except
for a few specific  packages). However, if   you bootstraped with a  prefix for
which   you  don't   have  writing   permissions,    \robotpkg   has a     {\rm
just-in-time-sudo}  feature,  which allows you to  become  {\tt  root}  for the
actual installation step.

That's it, the software should now be installed   under  the prefix  of the
packages tree --- {\tt /opt/openrobots} by default --- and setup for use.

You can now enter:

\begin{verbatim}
% make clean
\end{verbatim}

to remove the compiled files in the work  directory, as you shouldn't need them
any more. If  other packages were also  added to your system (dependencies)  to
allow your program to compile, you can also tidy these up with the command:

\begin{verbatim}
% make clean-depends
\end{verbatim}

Since  the three tasks of building,  installing and  cleaning correspond to the
typical usage of \robotpkg, a helper target doing all these tasks exists and is
called {\tt update}. Thus,  to intall a package  with a single command, you can
simply run:

\begin{verbatim}
% make update
\end{verbatim}

In addition, {\tt  make update} will  also recompile all the installed packages
that were depending on the package that you are updating. This can be quite
time consuming if you are updating a low-level package. Also, note that all
packages that depend on the package you are updating will be deinstalled
first and unavailable in your system until all packages are recompiled and
reinstalled.

%
%    <para>Some packages look in &mk.conf; to
%    alter some configuration options at build time.  Have a look at
%    <filename>pkgsrc/mk/defaults/mk.conf</filename> to get an overview
%    of what will be set there by default.  Environment variables such
%    as <varname>LOCALBASE</varname> can be set in
%    &mk.conf; to save having to remember to
%    set them each time you want to use pkgsrc.</para>
%

Occasionally, people want to ``look under the covers'' to see  what is going on
when a  package  is building  or being  installed.  This may  be for  debugging
purposes, or  out  of simple curiosity. A  number  of utility values have  been
added to help with this.

\begin{enumerate}

\item If you invoke the {\tt make} command with {\tt PKG\_DEBUG\_LEVEL=1}, then
      a huge amount of information will be displayed. For example,

\begin{verbatim}
% make patch PKG_DEBUG_LEVEL=1
\end{verbatim}

      will show all the commands that are invoked, up to and including the
      ``patch'' stage. Using {\tt PKG\_DEBUG\_LEVEL=2} will give you even
      more details.

\item If you want to know the value of a certain {\tt make} definition, then
   the {\tt VARNAME} variable   should be used,  in  conjunction with the  {\tt
   show-var} target.  e.g.  to show the  expansion  of the  {\tt make} variable
   {\tt LOCALBASE}:

\begin{verbatim}
% make show-var VARNAME=LOCALBASE
\end{verbatim}

\end{enumerate}

%    <para>If you want to install a binary package that you've either
%    created yourself (see next section), that you put into
%    pkgsrc/packages manually or that is located on a remote FTP
%    server, you can use the "bin-install" target. This target will
%    install a binary package - if available - via &man.pkg.add.1;,
%    else do a <command>make package</command>.  The list of remote FTP
%    sites searched is kept in the variable
%    <varname>BINPKG_SITES</varname>, which defaults to
%    ftp.NetBSD.org. Any flags that should be added to &man.pkg.add.1;
%    can be put into <varname>BIN_INSTALL_FLAGS</varname>.  See
%    <filename>pkgsrc/mk/defaults/mk.conf</filename> for more
%    details.</para>


%    <para>A final word of warning: If you set up a system that has a
%    non-standard setting for <varname>LOCALBASE</varname>, be sure to
%    set that before any packages are installed, as you cannot use
%    several directories for the same purpose. Doing so will result in
%    pkgsrc not being able to properly detect your installed packages,
%    and fail miserably. Note also that precompiled binary packages are
%    usually built with the default <varname>LOCALBASE</varname> of
%    <filename>/usr/pkg</filename>, and that you should
%    <emphasis>not</emphasis> install any if you use a non-standard
%    <varname>LOCALBASE</varname>.</para>


\subsection{Building packages from a repository checkout} % ----------------

Before building a  package, \robotpkg fetches the sources  from the official(s)
download  location(s),  as  instructed  by the  {\tt  MASTER\_SITES}  variable.
This is the standard and expected behaviour when you work with stable packages.

Occasionally, though,  it is useful to fetch  a snapshot of the  sources from a
development repository. For instance, one  might want to quickly test a release
candidate of a  package, or fix a simple  bug and create a patch  from the fix.
Whenever a package defines  the {\tt MASTER\_REPOSITORY} variable, \robotpkg is
able to temporarily  work with the repository defined in  this variable. At the
moment, {\tt cvs}, {\tt svn} and {\tt git} repositories are supported.

To enable this feature for a given package,  you have to first instruct
\robotpkg to work from a '{\tt checkout}' (instead of the stable releases) by
doing '{\tt make checkout}' in the package directory. For instance:

\begin{verbatim}
% cd robotpkg/foo/bar
% make checkout
\end{verbatim}

This sets  a permanent flag in the  {\em working} directory of  the package and
the {\em checkout}  configuration option will be retained  until the next '{\tt
make clean}'. After a '{\tt make  clean}', the configuration option is set back
to its default and \robotpkg will  work again with stable releases. This option
is set on a {\em per} package  basis only: configuring one package to work with
checkouts does not affect the behaviour of other packages.

After a '{\tt make checkout}' (and until a '{\tt make clean}'), the package has
a regular  checkout in its {\em  working} subdirectory.  You  can thus manually
edit, commit, switch branches, etc.  in  the package sources, like in any other
repository, by  first {\tt  cd}ing into the  working directory, then  using the
usual repository commands ({\tt cvs}, {\tt svn} or {\tt git}).

Of  course, the  individual  \robotpkg  targets are  still  available from  the
package  entry in  the robotpkg  hierarchy.  You  can for  instance  {\tt 'make
patch'}, {\tt 'configure'}, {\tt 'build'}, {\tt 'install'} or {\tt 'update'} as
usual. Note that  \robotpkg is not exactly stateless, and  this is most visible
when  working with  checkouts:  for  instance, after  a  successful {\tt  'make
build'}, you  have to do {\tt 'make  rebuild'} to force rebuilding  if you have
modified  the  sources.   The  same  holds  for  {\tt   'configure'}  (do  {\tt
'reconfigure'})  or {\tt  'install'} (do  {'reinstall'}, but  since  you cannot
install a package  twice, you normally have to use {\tt  'make replace'} in the
particular case of reinstalling a package).

The  {\tt  'clean'}  target  is  special,  in  that  it  removes  the  checkout
configuration  option and  all checkouted  sources, including  locally modified
sources. In order to prevent accidental deletion of precious files, you have to
confirm the cleanign with {\tt 'clean confirm'}, as in:

\begin{verbatim}
% make clean confirm
\end{verbatim}

A  final  remark:  we {\em  STRONGLY  DISCOURAGE}  the  use  of robotpkg  as  a
development tool  (i.e. using the {\tt  'checkout'} feature on  a {\em regular}
basis), for at least two reasons:

\begin{itemize}
   \item \robotpkg  is not designed  for this: it  will not really help  you in
   your  daily   development  work,   compared  to  the   manual  configuration
   installation of the software. It will sometimes create even more trouble, by
   ensuring  that all  the software  depending  on the  checkouted software  is
   up-to-date, which is not necessarily something you want to do every time you
   compile.

   \item  A checkout  breaks the  notion  of 'release'  and you  loose all  the
   benefits from working with packages.  In particular, you have no clear state
   of what is installed: you cannot easily reproduce the situation of time T at
   time T+n and don't know precisely  who requires which version of what. It is
   much  more  efficient and  robust  to release  frequently  a  software in  a
   development phase, than using a {\em rolling release} approach.
\end{itemize}

In our opinion, the {\tt 'checkout'}  target use should be limited to testing a
release candidate or  quickly fix a bug  and create a patch from  the fix, that
you commit upstream and put in the patches/ directory until the next release.


\subsection{Installing binary packages} % ----------------------------------

At the moment, installing binary packages is not documented.


\subsection{Removing packages} % -------------------------------------------

To deinstall a package, it does not matter whether it was installed from source
code or  from a  binary package.  The  {\tt robotpkg\_delete} command  does not
know it  anyway.  To delete a  package, you can just  run {\tt robotpkg\_delete
<package-name>}.  The package name can be given with or without version number.
Wildcards can  also be used  to deinstall a  set of packages, for  example {\tt
*genom*} all  packages whose  name contain  the word {\tt  genom}.  Be  sure to
include them  in quotes,  so that the  shell does  not expand them  before {\tt
robotpkg\_delete} sees them.

The {\tt -r} option is very powerful: it  removes all the packages that require
the package in question and then removes the package itself. For example:

\begin{verbatim}
% robotpkg_delete -r genom
\end{verbatim}

will remove genom and all the packages that used it; this allows
upgrading the {\tt genom} package.


\subsection{Getting information about installed packages} % ----------------

The {\tt  robotpkg\_info} shows information about installed  packages or binary
package files.


\subsection{Other administrative functions} % ------------------------------

The  {\tt robotpkg\_admin}  executes  various administrative  functions on  the
package system.


\chapter{The robotpkg developer's guide}
\label{chapter:developer}

\chapter{The robotpkg infrastructure internals}
\label{chapter:internal}

\end{document} % -----------------------------------------------------------

\else
   \usepackage[T1]{fontenc}
   \usepackage{robotpkg}
\fi

\title{A guide to robotpkg}
\author{Anthony Mallet --- {\tt anthony.mallet@laas.fr}}
\date{\today}

\def\robotpkg{{\tt robotpkg} }

\begin{document} % ---------------------------------------------------------

\frontmatter
\maketitle

{\small\parindent0pt
Copyright \copyright 2006-2010 LAAS/CNRS.\\
Copyright \copyright 1997-2010 The NetBSD Foundation, Inc.\\
All rights reserved.\\

Redistribution and use in source and binary forms, with or without
modification, are permitted provided that the following conditions
are met:

\begin{enumerate}
\item Redistributions of source code must retain the above copyright
      notice, this list of conditions and the following disclaimer.

\item Redistributions in binary form must reproduce the above copyright
      notice, this list of conditions and the following disclaimer in the
      documentation and/or other materials provided with the distribution.
\end{enumerate}

THIS SOFTWARE IS PROVIDED BY THE NETBSD FOUNDATION, INC. AND CONTRIBUTORS
``AS IS'' AND ANY EXPRESS OR IMPLIED WARRANTIES, INCLUDING, BUT NOT LIMITED
TO, THE IMPLIED WARRANTIES OF MERCHANTABILITY AND FITNESS FOR A PARTICULAR
PURPOSE ARE DISCLAIMED.  IN NO EVENT SHALL THE FOUNDATION OR CONTRIBUTORS
BE LIABLE FOR ANY DIRECT, INDIRECT, INCIDENTAL, SPECIAL, EXEMPLARY, OR
CONSEQUENTIAL DAMAGES (INCLUDING, BUT NOT LIMITED TO, PROCUREMENT OF
SUBSTITUTE GOODS OR SERVICES; LOSS OF USE, DATA, OR PROFITS; OR BUSINESS
INTERRUPTION) HOWEVER CAUSED AND ON ANY THEORY OF LIABILITY, WHETHER IN
CONTRACT, STRICT LIABILITY, OR TORT (INCLUDING NEGLIGENCE OR OTHERWISE)
ARISING IN ANY WAY OUT OF THE USE OF THIS SOFTWARE, EVEN IF ADVISED OF THE
POSSIBILITY OF SUCH DAMAGE.
}


\tableofcontents
\mainmatter

\chapter{Introduction}
\label{chapter:introduction}
\input{introduction}

\chapter{The robotpkg user's guide}
\label{chapter:user}

Basically, there are two ways of using robotpkg.  The  first is to only install
the  package tools and to  use binary packages that  someone else has prepared.
The second way is  to install the  programs from source. Then  you are  able to
build your own packages,  and you can  still use  binary packages from  someone
else. Sections in this document will detail both approaches where appropriate.

% $LAAS: getting.tex 2009/02/02 00:14:34 tho $
%
% Copyright (c) 2009 LAAS/CNRS
% All rights reserved.
%
% Permission to use, copy, modify, and distribute this software for any purpose
% with or without   fee is hereby granted, provided   that the above  copyright
% notice and this permission notice appear in all copies.
%
% THE SOFTWARE IS PROVIDED "AS IS" AND THE AUTHOR DISCLAIMS ALL WARRANTIES WITH
% REGARD TO THIS  SOFTWARE INCLUDING ALL  IMPLIED WARRANTIES OF MERCHANTABILITY
% AND FITNESS. IN NO EVENT SHALL THE AUTHOR  BE LIABLE FOR ANY SPECIAL, DIRECT,
% INDIRECT, OR CONSEQUENTIAL DAMAGES OR  ANY DAMAGES WHATSOEVER RESULTING  FROM
% LOSS OF USE, DATA OR PROFITS, WHETHER IN AN ACTION OF CONTRACT, NEGLIGENCE OR
% OTHER TORTIOUS ACTION,   ARISING OUT OF OR IN    CONNECTION WITH THE USE   OR
% PERFORMANCE OF THIS SOFTWARE.
%
%                                             Anthony Mallet on Sat Jan 10 2009
%

Basically, there are two ways of using robotpkg.  The  first is to only install
the package tools  and to use binary packages  that someone  else has prepared.
The second way is  to install the programs from   source. Then you are able  to
build your  own packages, and  you can still use   binary packages from someone
else.


\section{Where to get robotpkg and how to keep it up-to-date} % ------------
\label{section:getting}

Before you download and extract the files, you need to decide where you want to
extract  them.  robotpkg is  usually installed  in  {\tt /opt/openrobots},  but
creating this directory will probably   require administration privileges.   If
you don't have such privileges, you are free to  install the sources and binary
packages wherever you want in your filesystem, provided  that the pathname does
not contain white-space or other  characters that are interpreted specially  by
the shell and some other programs.  A safe bet is to  use only letters, digits,
underscores  and dashes. The rest of  this  document will  assume  that you are
using {\tt  /opt/openrobots}.  You should adapt  this path to whatever location
you choosed.


\subsection{Getting robotpkg for the first time} % -------------------------

robotpkg  will {\em never} require administration  privileges by itself.  So we
recommend that you set up the {\tt  /opt/openrobots} directory with read, write
and execute permissions for your regular user  name and then  only work as this
user afterwards. If something ever goes really  wrong, you might thank yourself
later that you did so\ldots This  can be done with  the following commands in a
shell:

\begin{verbatim}
% sudo mkdir -p /opt/openrobots
% sudo chown `id -u` /opt/openrobots
\end{verbatim}

At  the    moment,  robotpkg   is      only   distributed    {\em  via}     the
\href{http://git-scm.com/}{\tt git}  software content  management  system. {\tt
git} will probably be available on your system but if you don't have it readily
installed   or if  you  are   unsure  about  it,   contact your  local   system
administrator.

There are two download methods: the anonymous one and the authenticated
one. The two methods are described here.


\subsubsection{Anonymous download}

Anonymous  download is perfect  if  you don't intend  to  work on  the robotpkg
infrastructure itself, nor commit any changes or packages additions back to the
robotpkg main repository.  This is the recommended way  to go: it will fit most
users' usage while still leaving the possibility to send feedback via patches.

As your regular user, simply run in a shell:

\begin{verbatim}
% cd /opt/openrobots
% git clone http://softs.laas.fr/git/robots/robotpkg.git
\end{verbatim}


\subsubsection{Authenticated download}

Authenticated download requires a valid login  on the main robotpkg repository,
and  will give you  full commit access to this   repository. Assuming your user
name is ``{\tt user}'', run the following:

\begin{verbatim}
% cd /opt/openrobots
% git clone ssh://user@softs.laas.fr/git/robots/robotpkg
\end{verbatim}


\subsection{Keeping robotpkg up-to-date} % ---------------------------------

robotpkg is  a living  thing: updates  to the packages  are made  perdiodicaly,
problems are fixed,  enhancements are developed\ldots  In order to get the most
recent packages descriptions, you should keep your robotpkg copy up-to-date by
regularly running {\tt git pull}:

\begin{verbatim}
% cd /opt/openrobots/robotpkg
% git pull
\end{verbatim}

When you update robotpkg, the git program will only  touch those files that are
registered in the git repository. That means that any packages that you created
on your own will stay unmodified. If you change files that  are managed by git,
later updates will try to merge your changes with  those that have been done by
others. See the git-pull manual for details.

If you want  to be informed  of package additions  and other  updates, a public
mailing    list  is   available    for   your    reading   pleasure.  Go     to
\url{https://sympa.laas.fr/sympa/info/robotpkg}    for   more  information  and
subscription.

\input{bootstrapping}
% $LAAS: using.tex 2010/08/19 11:54:35 mallet $
%
% Copyright (c) 2009-2010 LAAS/CNRS
% All rights reserved.
%
% Permission to use, copy, modify, and distribute this software for any purpose
% with or without   fee is hereby granted, provided   that the above  copyright
% notice and this permission notice appear in all copies.
%
% THE SOFTWARE IS PROVIDED "AS IS" AND THE AUTHOR DISCLAIMS ALL WARRANTIES WITH
% REGARD TO THIS  SOFTWARE INCLUDING ALL  IMPLIED WARRANTIES OF MERCHANTABILITY
% AND FITNESS. IN NO EVENT SHALL THE AUTHOR  BE LIABLE FOR ANY SPECIAL, DIRECT,
% INDIRECT, OR CONSEQUENTIAL DAMAGES OR  ANY DAMAGES WHATSOEVER RESULTING  FROM
% LOSS OF USE, DATA OR PROFITS, WHETHER IN AN ACTION OF CONTRACT, NEGLIGENCE OR
% OTHER TORTIOUS ACTION,   ARISING OUT OF OR IN    CONNECTION WITH THE USE   OR
% PERFORMANCE OF THIS SOFTWARE.
%
%                                             Anthony Mallet on Sun Jan 11 2009
%

\section{Using robotpkg} % -------------------------------------------------

After obtaining \robotpkg, the  {\tt robotpkg} directory now  contains a set of
packages, organized   into  categories.  You can   browse  the online  index of
packages, or run {\tt  make index} from the {\tt  robotpkg} directory to  build
local {\tt index.html}  files for all  packages, viewable with any web  browser
such as {\tt lynx} or {\tt firefox}.


\subsection{Building packages from source} % -------------------------------

The  first step for  building  a  package  is  downloading the {\em  distfiles}
(i.e. the unmodified  source). If they have not  yet been downloaded, \robotpkg
will  fetch them automatically  and place them  in the {\tt robotpkg/distfiles}
directory.

Once the software  has  been downloaded,  any  patches will be applied  and the
package will  be compiled for  you.  This may  take some time depending on your
computer, and how many other packages the software depends on and their compile
time.

For  example,  type the following  commands  at the shell   prompt to build the
robotpkg documentation package:

\begin{verbatim}
% cd /opt/openrobots/robotpkg
% cd doc/robotpkg
% make
\end{verbatim}

The next  stage is  to  actually install the newly   compiled package onto your
system. While you   are still in  the directory  for whatever package  you  are
installing, you can do this by entering:

\begin{verbatim}
% make install
\end{verbatim}

Installing the package on your system does  not require you  to be root (except
for a few specific  packages). However, if   you bootstraped with a  prefix for
which   you  don't   have  writing   permissions,    \robotpkg   has a     {\rm
just-in-time-sudo}  feature,  which allows you to  become  {\tt  root}  for the
actual installation step.

That's it, the software should now be installed   under  the prefix  of the
packages tree --- {\tt /opt/openrobots} by default --- and setup for use.

You can now enter:

\begin{verbatim}
% make clean
\end{verbatim}

to remove the compiled files in the work  directory, as you shouldn't need them
any more. If  other packages were also  added to your system (dependencies)  to
allow your program to compile, you can also tidy these up with the command:

\begin{verbatim}
% make clean-depends
\end{verbatim}

Since  the three tasks of building,  installing and  cleaning correspond to the
typical usage of \robotpkg, a helper target doing all these tasks exists and is
called {\tt update}. Thus,  to intall a package  with a single command, you can
simply run:

\begin{verbatim}
% make update
\end{verbatim}

In addition, {\tt  make update} will  also recompile all the installed packages
that were depending on the package that you are updating. This can be quite
time consuming if you are updating a low-level package. Also, note that all
packages that depend on the package you are updating will be deinstalled
first and unavailable in your system until all packages are recompiled and
reinstalled.

%
%    <para>Some packages look in &mk.conf; to
%    alter some configuration options at build time.  Have a look at
%    <filename>pkgsrc/mk/defaults/mk.conf</filename> to get an overview
%    of what will be set there by default.  Environment variables such
%    as <varname>LOCALBASE</varname> can be set in
%    &mk.conf; to save having to remember to
%    set them each time you want to use pkgsrc.</para>
%

Occasionally, people want to ``look under the covers'' to see  what is going on
when a  package  is building  or being  installed.  This may  be for  debugging
purposes, or  out  of simple curiosity. A  number  of utility values have  been
added to help with this.

\begin{enumerate}

\item If you invoke the {\tt make} command with {\tt PKG\_DEBUG\_LEVEL=1}, then
      a huge amount of information will be displayed. For example,

\begin{verbatim}
% make patch PKG_DEBUG_LEVEL=1
\end{verbatim}

      will show all the commands that are invoked, up to and including the
      ``patch'' stage. Using {\tt PKG\_DEBUG\_LEVEL=2} will give you even
      more details.

\item If you want to know the value of a certain {\tt make} definition, then
   the {\tt VARNAME} variable   should be used,  in  conjunction with the  {\tt
   show-var} target.  e.g.  to show the  expansion  of the  {\tt make} variable
   {\tt LOCALBASE}:

\begin{verbatim}
% make show-var VARNAME=LOCALBASE
\end{verbatim}

\end{enumerate}

%    <para>If you want to install a binary package that you've either
%    created yourself (see next section), that you put into
%    pkgsrc/packages manually or that is located on a remote FTP
%    server, you can use the "bin-install" target. This target will
%    install a binary package - if available - via &man.pkg.add.1;,
%    else do a <command>make package</command>.  The list of remote FTP
%    sites searched is kept in the variable
%    <varname>BINPKG_SITES</varname>, which defaults to
%    ftp.NetBSD.org. Any flags that should be added to &man.pkg.add.1;
%    can be put into <varname>BIN_INSTALL_FLAGS</varname>.  See
%    <filename>pkgsrc/mk/defaults/mk.conf</filename> for more
%    details.</para>


%    <para>A final word of warning: If you set up a system that has a
%    non-standard setting for <varname>LOCALBASE</varname>, be sure to
%    set that before any packages are installed, as you cannot use
%    several directories for the same purpose. Doing so will result in
%    pkgsrc not being able to properly detect your installed packages,
%    and fail miserably. Note also that precompiled binary packages are
%    usually built with the default <varname>LOCALBASE</varname> of
%    <filename>/usr/pkg</filename>, and that you should
%    <emphasis>not</emphasis> install any if you use a non-standard
%    <varname>LOCALBASE</varname>.</para>


\subsection{Building packages from a repository checkout} % ----------------

Before building a  package, \robotpkg fetches the sources  from the official(s)
download  location(s),  as  instructed  by the  {\tt  MASTER\_SITES}  variable.
This is the standard and expected behaviour when you work with stable packages.

Occasionally, though,  it is useful to fetch  a snapshot of the  sources from a
development repository. For instance, one  might want to quickly test a release
candidate of a  package, or fix a simple  bug and create a patch  from the fix.
Whenever a package defines  the {\tt MASTER\_REPOSITORY} variable, \robotpkg is
able to temporarily  work with the repository defined in  this variable. At the
moment, {\tt cvs}, {\tt svn} and {\tt git} repositories are supported.

To enable this feature for a given package,  you have to first instruct
\robotpkg to work from a '{\tt checkout}' (instead of the stable releases) by
doing '{\tt make checkout}' in the package directory. For instance:

\begin{verbatim}
% cd robotpkg/foo/bar
% make checkout
\end{verbatim}

This sets  a permanent flag in the  {\em working} directory of  the package and
the {\em checkout}  configuration option will be retained  until the next '{\tt
make clean}'. After a '{\tt make  clean}', the configuration option is set back
to its default and \robotpkg will  work again with stable releases. This option
is set on a {\em per} package  basis only: configuring one package to work with
checkouts does not affect the behaviour of other packages.

After a '{\tt make checkout}' (and until a '{\tt make clean}'), the package has
a regular  checkout in its {\em  working} subdirectory.  You  can thus manually
edit, commit, switch branches, etc.  in  the package sources, like in any other
repository, by  first {\tt  cd}ing into the  working directory, then  using the
usual repository commands ({\tt cvs}, {\tt svn} or {\tt git}).

Of  course, the  individual  \robotpkg  targets are  still  available from  the
package  entry in  the robotpkg  hierarchy.  You  can for  instance  {\tt 'make
patch'}, {\tt 'configure'}, {\tt 'build'}, {\tt 'install'} or {\tt 'update'} as
usual. Note that  \robotpkg is not exactly stateless, and  this is most visible
when  working with  checkouts:  for  instance, after  a  successful {\tt  'make
build'}, you  have to do {\tt 'make  rebuild'} to force rebuilding  if you have
modified  the  sources.   The  same  holds  for  {\tt   'configure'}  (do  {\tt
'reconfigure'})  or {\tt  'install'} (do  {'reinstall'}, but  since  you cannot
install a package  twice, you normally have to use {\tt  'make replace'} in the
particular case of reinstalling a package).

The  {\tt  'clean'}  target  is  special,  in  that  it  removes  the  checkout
configuration  option and  all checkouted  sources, including  locally modified
sources. In order to prevent accidental deletion of precious files, you have to
confirm the cleanign with {\tt 'clean confirm'}, as in:

\begin{verbatim}
% make clean confirm
\end{verbatim}

A  final  remark:  we {\em  STRONGLY  DISCOURAGE}  the  use  of robotpkg  as  a
development tool  (i.e. using the {\tt  'checkout'} feature on  a {\em regular}
basis), for at least two reasons:

\begin{itemize}
   \item \robotpkg  is not designed  for this: it  will not really help  you in
   your  daily   development  work,   compared  to  the   manual  configuration
   installation of the software. It will sometimes create even more trouble, by
   ensuring  that all  the software  depending  on the  checkouted software  is
   up-to-date, which is not necessarily something you want to do every time you
   compile.

   \item  A checkout  breaks the  notion  of 'release'  and you  loose all  the
   benefits from working with packages.  In particular, you have no clear state
   of what is installed: you cannot easily reproduce the situation of time T at
   time T+n and don't know precisely  who requires which version of what. It is
   much  more  efficient and  robust  to release  frequently  a  software in  a
   development phase, than using a {\em rolling release} approach.
\end{itemize}

In our opinion, the {\tt 'checkout'}  target use should be limited to testing a
release candidate or  quickly fix a bug  and create a patch from  the fix, that
you commit upstream and put in the patches/ directory until the next release.


\subsection{Installing binary packages} % ----------------------------------

At the moment, installing binary packages is not documented.


\subsection{Removing packages} % -------------------------------------------

To deinstall a package, it does not matter whether it was installed from source
code or  from a  binary package.  The  {\tt robotpkg\_delete} command  does not
know it  anyway.  To delete a  package, you can just  run {\tt robotpkg\_delete
<package-name>}.  The package name can be given with or without version number.
Wildcards can  also be used  to deinstall a  set of packages, for  example {\tt
*genom*} all  packages whose  name contain  the word {\tt  genom}.  Be  sure to
include them  in quotes,  so that the  shell does  not expand them  before {\tt
robotpkg\_delete} sees them.

The {\tt -r} option is very powerful: it  removes all the packages that require
the package in question and then removes the package itself. For example:

\begin{verbatim}
% robotpkg_delete -r genom
\end{verbatim}

will remove genom and all the packages that used it; this allows
upgrading the {\tt genom} package.


\subsection{Getting information about installed packages} % ----------------

The {\tt  robotpkg\_info} shows information about installed  packages or binary
package files.


\subsection{Other administrative functions} % ------------------------------

The  {\tt robotpkg\_admin}  executes  various administrative  functions on  the
package system.

\input{configuring}

\chapter{The robotpkg developer's guide}
\label{chapter:developer}

This part of the documentation deals with creating and modifying packages.

% $LAAS: pkgvars.tex 2010/10/06 18:45:43 mallet $
%
% Copyright (c) 2010 LAAS/CNRS
% All rights reserved.
%
% Permission to use, copy, modify, and distribute this software for any purpose
% with or without   fee is hereby granted, provided   that the above  copyright
% notice and this permission notice appear in all copies.
%
% THE SOFTWARE IS PROVIDED "AS IS" AND THE AUTHOR DISCLAIMS ALL WARRANTIES WITH
% REGARD TO THIS  SOFTWARE INCLUDING ALL  IMPLIED WARRANTIES OF MERCHANTABILITY
% AND FITNESS. IN NO EVENT SHALL THE AUTHOR  BE LIABLE FOR ANY SPECIAL, DIRECT,
% INDIRECT, OR CONSEQUENTIAL DAMAGES OR  ANY DAMAGES WHATSOEVER RESULTING  FROM
% LOSS OF USE, DATA OR PROFITS, WHETHER IN AN ACTION OF CONTRACT, NEGLIGENCE OR
% OTHER TORTIOUS ACTION,   ARISING OUT OF OR IN    CONNECTION WITH THE USE   OR
% PERFORMANCE OF THIS SOFTWARE.
%
%                                             Anthony Mallet on Wed Oct  6 2010
%
\section{Package files, directories and contents} % ------------------------
\label{section:pkgvars}

Whenever you're preparing a package, there are a number of files involved which
are described in the following sections.

\subsection{Makefile} % ----------------------------------------------------
\label{subsection:makefile}

Building, installation and creation of a package are all controlled by the
package's Makefile. The Makefile describes various things about a package,
for example from where to get it, how to configure, build, and install it.

A package Makefile contains several sections that describe the package.

In the first section there are the following variables, which should appear
exactly in the order given here. The order and grouping of the variables is
mostly historical and has no further meaning.

\begin{description}
   \item[MASTER\_SITES] In simple cases, {\tt MASTER\_SITES}  defines all URLs
   from where the distfile, whose name is derived from the {\tt DISTNAME}
   variable, is fetched.

   When actually fetching the distfiles, each item from {\tt MASTER\_SITES}
   gets the name of each distfile appended to it, without an intermediate
   slash. Therefore, all site values have to end with a slash or other
   separator character. This allows for example to set {\tt MASTER\_SITES} to a
   URL of a CGI script that gets the name of the distfile as a parameter. In
   this case, the definition would look like:
   \begin{quote}
      {\tt MASTER\_SITES=   http://www.example.com/download.cgi?file=}
   \end{quote}

   There are some predefined values for {\tt MASTER\_SITES}, which can be used
   in packages. The names of the variables should speak for themselves.
   \begin{quote}\tt
      \$\{MASTER\_SITE\_SOURCEFORGE\}\\
      \$\{MASTER\_SITE\_GNU\}\\
      \$\{MASTER\_SITE\_OPENROBOTS\}
   \end{quote}

   If you choose one of these predefined sites, you may want to specify a
   subdirectory of that site. Since these macros may expand to more than one
   actual site, {\em you must} use the following construct to specify a
   subdirectory:
   \begin{quote}\tt
      MASTER\_SITES=~\$\{MASTER\_SITE\_SOURCEFORGE:=project\_name/\}
   \end{quote}
   Note the trailing slash after the subdirectory name.

   \smallbreak
   \item[FETCH\_METHOD] This is the method used to download the distfile from
   {\tt MASTER\_SITES}. It defaults to '{\tt archive}' which corresponds to the
   normal situation where distfile is an archive available from {\tt
   MASTER\_SITES}, so it normally needs not to be set.

   However, it can happen that a software provider does not provide any archive
   available for download but has only a public repository. In this case, {\tt
   FETCH\_METHOD} can be set to {\tt cvs}, {\tt git} or {\tt svn} according to
   the kind of repository available. {\tt MASTER\_SITES} is then interpreted as
   a repository of the form {\tt url[@revision[+module]]}, where the bits
   between square brackets are optional and refer to a particular revision and
   module in the repository located at {\tt url}. {\tt url} can take any form
   supported by the underlying fetch tool ({\tt cvs}, {\tt git} or {\tt
   svn}). It is {\em strongly} advised to define at least a specific revision
   to be checked out, so that the package can be reproducibly installed in a
   known state.

\end{description}

% $LAAS: fixing.tex 2010/10/29 17:45:24 mallet $
%
% Copyright (c) 2010 LAAS/CNRS
% All rights reserved.
%
% Permission to use, copy, modify, and distribute this software for any purpose
% with or without   fee is hereby granted, provided   that the above  copyright
% notice and this permission notice appear in all copies.
%
% THE SOFTWARE IS PROVIDED "AS IS" AND THE AUTHOR DISCLAIMS ALL WARRANTIES WITH
% REGARD TO THIS  SOFTWARE INCLUDING ALL  IMPLIED WARRANTIES OF MERCHANTABILITY
% AND FITNESS. IN NO EVENT SHALL THE AUTHOR  BE LIABLE FOR ANY SPECIAL, DIRECT,
% INDIRECT, OR CONSEQUENTIAL DAMAGES OR  ANY DAMAGES WHATSOEVER RESULTING  FROM
% LOSS OF USE, DATA OR PROFITS, WHETHER IN AN ACTION OF CONTRACT, NEGLIGENCE OR
% OTHER TORTIOUS ACTION,   ARISING OUT OF OR IN    CONNECTION WITH THE USE   OR
% PERFORMANCE OF THIS SOFTWARE.
%
%                                             Anthony Mallet on Wed Oct 29 2010
%

\section{Making a package work} % ------------------------------------------
\label{section:fixing}

\subsection{Incrementing versions when fixing an existing package} % -------
\label{section:fixing:PKGREVISION}

When making fixes to an existing package it can be useful to change the version
number in {\tt PKGNAME}. To avoid conflicting with future versions by the
original author, a "r1", "r2", ... suffix can be used on package versions by
setting {\tt PKGREVISION=1} ({\tt 2}, ...) in the package Makefile. E.g.
\begin{quote}
   DISTNAME=             foo-17.42\\
   PKGREVISION=          9
\end{quote}
will result in a {\tt PKGNAME} of "foo-17.42r9". The "r" is treated like a "."
by the package tools.

{\tt PKGREVISION} should be incremented for any non-trivial change in the
resulting binary package. Without a {\tt PKGREVISION} bump, someone with the
previous version installed has no way of knowing that their package is out
of date. Thus, changes without increasing {\tt PKGREVISION} are essentially
labeled "this is so trivial that no reasonable person would want to
upgrade", and this is the rough test for when increasing {\tt PKGREVISION}
is appropriate. Examples of changes that do not merit increasing {\tt
PKGREVISION} are:
\begin{itemize}
   \item Changing {\tt HOMEPAGE}, {\tt MAINTAINER} or comments in Makefile.
   \item Changing build variables if the resulting binary package is the same.
   \item Changing {\tt DESCR}.
   \item Adding {\tt PKG\_OPTIONS} if the default options don't change.
\end{itemize}

Examples of changes that do merit an increase to {\tt PKGREVISION} include:
\begin{itemize}
   \item Security fixes
   \item Changes or additions to a patch file
   \item Changes to the {\tt PLIST}
   \item A dependency is changed or renamed.
\end{itemize}

{\tt PKGREVISION} must also be incremented when dependencies have ABI changes.

When a new release of the package is released, the {\tt PKGREVISION} must be
removed.


\chapter{The robotpkg infrastructure internals}
\label{chapter:internal}

\end{document} % -----------------------------------------------------------
