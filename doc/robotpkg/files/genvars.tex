% $LAAS: genvars.tex 2010/12/14 14:21:56 mallet $
%
% Copyright (c) 2010 LAAS/CNRS
% All rights reserved.
%
% Permission to use, copy, modify, and distribute this software for any purpose
% with or without   fee is hereby granted, provided   that the above  copyright
% notice and this permission notice appear in all copies.
%
% THE SOFTWARE IS PROVIDED "AS IS" AND THE AUTHOR DISCLAIMS ALL WARRANTIES WITH
% REGARD TO THIS  SOFTWARE INCLUDING ALL  IMPLIED WARRANTIES OF MERCHANTABILITY
% AND FITNESS. IN NO EVENT SHALL THE AUTHOR  BE LIABLE FOR ANY SPECIAL, DIRECT,
% INDIRECT, OR CONSEQUENTIAL DAMAGES OR  ANY DAMAGES WHATSOEVER RESULTING  FROM
% LOSS OF USE, DATA OR PROFITS, WHETHER IN AN ACTION OF CONTRACT, NEGLIGENCE OR
% OTHER TORTIOUS ACTION,   ARISING OUT OF OR IN    CONNECTION WITH THE USE   OR
% PERFORMANCE OF THIS SOFTWARE.
%
%                                             Anthony Mallet on Wed Oct 29 2010
%

\section{General operation} % ----------------------------------------------
\label{section:genvars}

\subsection{Incrementing versions when fixing an existing package} % -------
\label{section:genvars:PKGREVISION}

When making fixes to an existing package it can be useful to change the version
number in {\tt PKGNAME}. To avoid conflicting with future versions by the
original author, a "r1", "r2", ... suffix can be used on package versions by
setting {\tt PKGREVISION=1} ({\tt 2}, ...) in the package Makefile. E.g.
\begin{quote}
   DISTNAME=             foo-17.42\\
   PKGREVISION=          9
\end{quote}
will result in a {\tt PKGNAME} of "foo-17.42r9". The "r" is treated like a "."
by the package tools.

{\tt PKGREVISION} should be incremented for any non-trivial change in the
resulting binary package. Without a {\tt PKGREVISION} bump, someone with the
previous version installed has no way of knowing that their package is out
of date. Thus, changes without increasing {\tt PKGREVISION} are essentially
labeled "this is so trivial that no reasonable person would want to
upgrade", and this is the rough test for when increasing {\tt PKGREVISION}
is appropriate. Examples of changes that do not merit increasing {\tt
PKGREVISION} are:
\begin{itemize}
   \item Changing {\tt HOMEPAGE}, {\tt MAINTAINER} or comments in Makefile.
   \item Changing build variables if the resulting binary package is the same.
   \item Changing {\tt DESCR}.
   \item Adding {\tt PKG\_OPTIONS} if the default options don't change.
\end{itemize}

Examples of changes that do merit an increase to {\tt PKGREVISION} include:
\begin{itemize}
   \item Security fixes
   \item Changes or additions to a patch file
   \item Changes to the {\tt PLIST}
   \item A dependency is changed or renamed.
\end{itemize}

{\tt PKGREVISION} must also be incremented when dependencies have ABI changes.

When a new release of the package is released, the {\tt PKGREVISION} must be
removed.


\subsection{Substituting variable text in the package files
(the {\tt SUBST} framework)} % ---------------------------------------------------
\label{section:genvars:SUBST}

When you want to replace the same text in multiple files or when the
replacement text varies, patches alone cannot help. This is where the SUBST
framework comes in. It provides an easy-to-use interface for replacing text in
files. Example:
\begin{verbatim}
   SUBST_CLASSES+=           fix-paths
   SUBST_STAGE.fix-paths=    pre-configure
   SUBST_MESSAGE.fix-paths=  Fixing absolute paths.
   SUBST_FILES.fix-paths=    src/*.c
   SUBST_SED.fix-paths=      -e 's,"/usr/local,"${PREFIX},g'
\end{verbatim}

{\tt SUBST\_CLASSES}  is a list  of identifiers that  are used to  identify the
different {\tt  SUBST} blocks  that are defined.  The {\tt SUBST}  framework is
used by  \robotpkg{}, so it  is important to  always use the {\tt  +=} operator
with this variable. Otherwise some substitutions may be skipped.

The remaining  variables of each {\tt  SUBST} block are  parameterized with the
identifier from the first line ({\tt fix-paths} in this case).

{\tt SUBST\_STAGE.*}  specifies the  stage at which  the replacement  will take
place. All combinations  of pre-, do- and post- together with  a phase name are
possible, though only few are  actually used. Most commonly used are post-patch
and pre-configure. Of these two, pre-configure should be preferred because then
it is possible  to run {\tt make  patch} and have the state  after applying the
patches but before making any other changes. This is especially useful when you
are debugging  a package  in order  to create new  patches for  it.  Similarly,
post-build  is preferred  over pre-install,  because the  install  phase should
generally be kept as simple as  possible. When you use post-build, you have the
same files  in the working directory that  will be installed later,  so you can
check if the substitution has succeeded.

{\tt  SUBST\_MESSAGE.*} is an  optional text  that is  printed just  before the
substitution is  done.

{\tt SUBST\_FILES.*} is the list  of shell globbing patterns that specifies the
files in which the substitution  will take place.  The patterns are interpreted
relatively  to the {\tt WRKSRC}  directory.

{\tt SUBST\_SED.*}  is a  list of  arguments to {\tt  sed(1)} that  specify the
actual substitution.  Every sed command should be prefixed with -e, so that all
{\tt SUBST} blocks look uniform.

There are some more variables, but they are so seldomly used that they are only
documented in the mk/internal/subst.mk file.
