% $LAAS: bootstrapping.tex 2009/03/04 17:01:14 mallet $
%
% Copyright (c) 2009 LAAS/CNRS
% All rights reserved.
%
% Permission to use, copy, modify, and distribute this software for any purpose
% with or without   fee is hereby granted, provided   that the above  copyright
% notice and this permission notice appear in all copies.
%
% THE SOFTWARE IS PROVIDED "AS IS" AND THE AUTHOR DISCLAIMS ALL WARRANTIES WITH
% REGARD TO THIS  SOFTWARE INCLUDING ALL  IMPLIED WARRANTIES OF MERCHANTABILITY
% AND FITNESS. IN NO EVENT SHALL THE AUTHOR  BE LIABLE FOR ANY SPECIAL, DIRECT,
% INDIRECT, OR CONSEQUENTIAL DAMAGES OR  ANY DAMAGES WHATSOEVER RESULTING  FROM
% LOSS OF USE, DATA OR PROFITS, WHETHER IN AN ACTION OF CONTRACT, NEGLIGENCE OR
% OTHER TORTIOUS ACTION,   ARISING OUT OF OR IN    CONNECTION WITH THE USE   OR
% PERFORMANCE OF THIS SOFTWARE.
%
%                                             Anthony Mallet on Sun Jan 11 2009
%

\section{Bootstrapping robotpkg} % -----------------------------------------
\label{section:bootstrapping}

Once you have  downloaded the robotpkg sources  or the binary bootstrap kit  as
described  in  \xref{section:getting}{Section~\ref{section:getting}}, a minimal
set  of  the administrative package management  utilities  must be installed on
your system  before you  can  use robotpkg.   This  is  called the  ``bootstrap
phase'' and  should   be done only   once,  the very  first  time you  download
robotpkg.


\subsection{Bootstrapping via the binary kit} % ----------------------------

At the moment, the binary bootstrap kit is not available. Please bootstrap {\tt
robotpkg} as described in the next section.


\subsection{Bootstrapping from source} % -----------------------------------

You will  need a working C compiler  and the GNU-make   utility version 3.81 or
later.    If you have  extracted  the  robotpkg  archive  into  the standard {\tt
/opt/openrobots/robotpkg} location, installing the   bootstrap kit from  source
should then be as simple as:

\begin{verbatim}
% cd /opt/openrobots/robotpkg/bootstrap
% ./bootstrap
\end{verbatim}

This will  install various utilities   into {\tt /opt/openrobots/sbin}.  Should
you prefer another   location, you could  use the  {\tt -{}-prefix}  option  to
change the default installation  prefix.  For instance, configuring robotpkg to
install programs  into the openrobots directory  in your home directory  can be
done like this:

\begin{verbatim}
% cd robotpkg/bootstrap
% ./bootstrap --prefix=${HOME}/openrobots
\end{verbatim}

After the bootstrap script has run, a message  indicating the success should be
displayed.   {\bf If you choosed  a  non-standard installation path, read  this
message carefuly}, as it contains instructions that you have to follow in order
to setup your shell environment correctly.  These instructions are described in
the next section.


\subsubsection{Configuring your environment} % -----------------------------

If  you configured robotpkg,   during the bootstrap  phase,  to install to some
other location   than {\tt /opt/openrobots}, you  have   to setup manually your
shell environment so that it contains a few  variables holding the installation
path.  Assuming  you invoked bootstrap with {\tt --prefix=/path/to/openrobots},
you have two options that are compatible with each other:

\begin{itemize}
   \item Add  the directory {\tt  /path/to/openrobots/sbin}  to your {\tt PATH}
   variable. robotpkg will    then be able  to find    its administrative tools
   automatically and from that recover other configuration information. This is
   the preferred method.

   \item Create the environment variable {\tt ROBOTPKG\_BASE} and set its value
   to {\tt /path/to/openrobots}.  robotpkg will  look for this variable  first,
   so it takes precedence over the  first method.  This is  the method you have
   to choose  if  you have  configured  several instances  of robotpkg  in your
   system. This is ony useful in some circumstances and is not normally needed.
\end{itemize}

If  you  don't know  how  to setup   environment variables  permanently in your
system,  please  refer  to  your shell's  manual  or contact  your local system
administrator.


\subsubsection{The bootstrap script usage} % -------------------------------

The {\tt bootstrap} script will by default install the package administrative
tools in {\tt /opt/openrobots/sbin}, use {\tt gcc} as the C compiler and {\tt
make} as the GNU-make program. This behaviour can be fine-tuned by using the
following options:

\begin{description}
   \item[\tt   -{}-prefix <path>]   will   select the  prefix  location where
   programs will be installed in.

   \item[\tt -{}-sysconfdir <path>] defaults to {\tt <prefix>/etc}. This is the
   path to the robotpkg configuration file.  Other packages configuration files
   (if any) will also be stored in this directory.

   \item[\tt -{}-pkgdbdir  <path>] defaults to {\tt  <prefix>/var/db/pkg}. This
   is the path  to the package database  directory  where robotpkg will  do its
   internal bookkeeping.

   \item[\tt -{}-compiler <program>] defaults to {\tt gcc}.  Use this option if
   you want to use a different C compiler.

   \item[\tt -{}-make <program>] defaults to {\tt make}. Use this option if you
   want to use a different make program. This program should be compatible with
   GNU-make.

   \item[\tt -{}-help]  displays  the {\tt bootstrap} usage.  The comprehensive
   list of recognized options will be displayed.
\end{description}
